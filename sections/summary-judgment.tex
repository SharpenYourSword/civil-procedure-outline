\section{Summary Judgment}

\begin{enumerate}
    \item \textbf{Motion for dismissal for failure to state a claim}: assuming the plaintiff's allegations are assumed to be true, the court determines whether there is a cause of action. Federal claims fall under FRCP 12(b)(6) and state claims fall under demurrer rules. Motions to dismiss are based on the pleadings themselves; any additional factual allegations will cause the motion to be treated as a motion for summary judgment.
    \item \textbf{Motion for summary judgment}: either side can challenge the legal sufficiency of the other's factual allegations or legal contentions under 12(a) (formerly 12(c)).
    \begin{enumerate}
        \item Exists to decide issues that are so one-sided that a trial would be wasteful.\footnote{Casebook p. 992 n. 1.}
        \item Pleadings do not support motions for summary judgment\footnote{Casebook p. 993 n. 2}---i.e., they must be supported with facts, not allegations.
        \item ``...one of the prime uses of discovery is to gather information that will be useful in supporting and opposing summary judgment.''\footnote{Casebook p. 993.}
        \item \textbf{Burden of production}: plaintiff must produce sufficient evidence on each element of the case for a jury to reasonable rule in its favor. Otherwise, under FRCP 50, the judge may grant judgment as a matter of law.\footnote{Casebook p. 994--95 n. 4.}
        \item \textbf{Burden of persuasion}: the standard by which a plaintiff will have to convince a jury (which, in civil cases, is ``by a preponderance'').\footnote{Casebook p. 994 n. 4}.
        \item Summary judgment tests the whether a party can meet the burden of production.
    \end{enumerate}
\end{enumerate}

\emph{[p. 994 n. 5: how do you determine whether a motion for summary judgment is ``supported''?]}

\subsection{FRCP 56}

\begin{itemize}
    \item (a) Court can grant summary judgment if ``there is not genuine dispute as to any material fact and the movant is entitled to judgment as a matter of law.''
    \item (b) Parties may file motions within 30 days after the close of discovery.
    \item (c) Procedures.
    \begin{itemize}
        \item (1) Supporting factual assertions.
        \item (2) Facts must be admissible.
        \item (3) Court can consider materials not cited in the motion.
        \item (4) Affidavits/declarations.
    \end{itemize}
    \item (d) When facts are unavailable to the non-movant.
    \item (e) Failing to properly support or address a fact.
    \item (f) Judgment independent of the motion. [Courts can grant summary judgments absent motions from any party.]
    \item (g) Failing to grant the requested relief. [If the summary judgment doesn't end the case, its outcome can still come into play during trial.]
    \item (h) Court can impose sanctions for bad faith affidavits/declarations.
\end{itemize}

\subsection{\emph{Adickes v. S.H. Kress \& Co.}}

\begin{enumerate}
    \item Sandra Adickes, a white teacher, took a group of black students to Kress's restaurant in Hattiesburg, Mississippi. She was refused service and then arrested for vagrancy.
    \item She alleged that (1) she was refused service because she was part of a mixed-race group and (2) the refusal of service and subsequent arrest resulted from a conspiracy between Kress and the Hattiesburg police. The District Court for the Southern District of New York directed a verdict for the defendants on the first count [i.e., no cause of action?] and granted summary judgment on the second. The Second Circuit unanimously affirmed.
    \item The Supreme Court reversed on both counts (but the edited opinion in the casebook addresses only the summary judgment on the second count).
    \item Justice Harlan:
    \begin{enumerate}
        \item To show conspiracy as alleged, Adickes must show (1) that an employee of Kress deprived her of her constitutional rights and (2) that the defendant acted ``under color of law'' (which is satisfied if an employee and the policeman ``somehow reached an understanding to deny Miss Adickes service''\footnote{Casebook p. 986.}).
        \item Summary judgment was inappropriate because the respondent, Kress, ``failed to carry its burden of showing the absence of any genuine issue of fact''\footnote{Casebook p. 987.} (and any material it submits ``must be viewed in the light most favorable to the opposing party.''\footnote{Casebook p. 988.}
        \item In this case, the two big factual gaps were that the police officers failed to ``foreclose the possibility'' that (1) they were in the store and (2) influenced the Kress employee to not serve Adickes.\footnote{Casebook p. 990.}
        \item Because respondent did not meet the burden of establishing the police officers' presence, petitioner was not required to file opposing affidavits.
    \end{enumerate}
\end{enumerate}

\subsection{\emph{Celotex Corp. v. Catrett}}

\begin{enumerate}
    \item Catrett died in 1979. In 1980, his wife (respondent) filed suit against 15 asbestos companies, including Celotex (petitioner).
    \item Celotex moved for summary judgment for lack of evidence showing that its product was a proximate cause of Catrett's death, including a lack of witnesses who could testify to that effect. Catrett produced three documents she claimed demonstrated ``a genuine material factual dispute.''\footnote{Casebook p. 996--97.} Celotex argued that the documents were inadmissible hearsay.
    \item The District Court for D.C. granted the motions from Celotex and the other defendants. Catrett appealed only the grant for Celotex. The D.C. Circuit reversed on the grounds that Celotex failed to show any evidence to support its motion.\footnote{Casebook p. 997.}
    \item Justice Rehnquist:
    \begin{enumerate}
        \item The D.C. Circuit's opinion was inconsistent with FRCP 56. The plaintiff bears the burden of proof for its claim, and in this case the defendant's motion contended that the plaintiff failed to establish the existence of an element essential to its case.
        \item The moving party need not negate its opponent's claim.
        \item ``One of the principle purposes of the summary judgment rule is to isolate and dispose of factually unsupported claims or defenses, and we think it should be interpreted in a way that allows it to accomplish this purpose.''\footnote{Casebook p. 998.}

    \end{enumerate}
\end{enumerate}

% \subsection{\emph{Arnstein v. Porter}}
% 
% \begin{enumerate}
%     \item 
% \end{enumerate}
% 
% \subsection{\emph{Scott v. Harris}}
% 
% \begin{enumerate}
%     \item 
% \end{enumerate}
