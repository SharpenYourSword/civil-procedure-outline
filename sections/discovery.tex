\section{Discovery}

\begin{enumerate}
    \item Discovery consists of \textbf{(1) interrogatories, (2) requests for 
    production or inspection, and (3) depositions.}\footnote{Casebook p. 9.}
    \item Fuller disclosure leads to the most favorable case for each 
    party.\footnote{Casebook p. 882.}
    \item Legitimate purposes: promotes settlement, helps determine whether a 
    case can be decided in a summary judgment.
    \item Less legitimate purposes: inflict costs, harass, ``reconstruction'' 
    (i.e., put into the record facts that aren't true).
    \item State discovery rules generally track the federal rules.
    \item The discovery process:
    \begin{enumerate}
        \item \emph{Informal investigation}: happens outside the compulsory 
        structure of formal discovery---interviews, document review, property 
        visits.
        \item \emph{Discovery plan}: FRCP 26(f) requires parties to agree to a 
        discovery plan.
        \item \emph{Initial disclosures}: mandatory disclosures include (1) 
        names and contact details of relevant individuals, (2) copies or 
        descriptions of records, (3) computations of damages, and (4) 
        insurance information. FRCP 26(a)(1)(A). Parties are only required to 
        disclose information that is favorable to their 
        cases.\footnote{Casebook p. 885.}
        \item \emph{Depositions}:
        \begin{enumerate}
            \item Depositions are binding but expensive. Lawyers generally 
            depose all unfriendly witnesses.\footnote{Casebook p. 886.}
            \item FRCP 30 defines the scope of depositions.
            \item Lawyers can instruct defendants not to answer to (1) 
            preserve a privilege, (2) enforce a protect order limiting 
            discovery, or (3) stop abusive behavior.
        \end{enumerate}
        \item \emph{Interrogatories}: written questions that must be answered 
        under oath, often with accompanying requests for documents. FRCP 33 
        governs interrogatories.
        \item \emph{Production}: items can be obtained by subpoena if 
        necessary. FRCP 34 controls.
        \begin{enumerate}
            \item \textbf{Privileges}: privileged information is not subject 
            to discovery. Common law privileged relationships include 
            attorney-client, spouse-spouse, doctor-patient, clergy-penitent, 
            and others that vary by state.
            \item \textbf{Work product}: an attorney's work product is 
            privileged. \emph{Hickman}. See \textbf{Work-Product Doctrine} 
            below.
        \end{enumerate}
        \item \emph{Physical and mental examinations}: only when physical or 
        mental states are issues in the case. FRCP 35 controls.
        \item \emph{Requests for admission}: to determine whether facts are 
        accurate and documents genuine. FRCP 36.
        \item \emph{Motions for protective orders and motions to compel}: 
        court must award attorneys' fees to the winning party. FRCP 26(c), 
        37(a)(1).
        \item \emph{Sanctions}: most commonly an award of costs.
    \end{enumerate}
\end{enumerate}

\subsection{FRCP 26: Duty to Disclose; General Provisions Governing Discovery}

\begin{itemize}
    \item (a) Required disclosures.
    \begin{itemize}
        \item (1) Initial disclosure.
        \begin{itemize}
            \item (A) Generally:
            \begin{itemize}
                \item (i) Names and contact details.
                \item (ii) Copies or descriptions of documents, etc.
                \item (iii) Computation of damages.
                \item (iv) Insurance details.
            \end{itemize}
            \item (E) Parties must supplement disclosures when required under 
            26(e).
        \end{itemize}
    \end{itemize}
    \item (b) Discovery scope and limits.
    \begin{itemize}
        \item (1) Parties can discover all nonprivileged matter relevant to 
        claims or defenses. It need not be admissible at trial if it is 
        \emph{reasonably calculated} lead to the discovery of admissible 
        evidence.
        \item (2) Limitations on frequency and extent.
        \begin{itemize}
            \item (A) Court can alter limits.
            \item (B) Electronic information need not be produced if it 
            carries undue burden or cost.
            \item (C) Court must limit discovery under certain circumstances 
            (enumerated within).
        \end{itemize}
        \item (3) Trial preparation: materials [i.e., work product---see 
        \emph{Hickman}].
        \begin{itemize}
            \item (A) Tangible things prepared for litigation are not 
            discoverable, unless:
            \begin{itemize}
                \item (i) They are discoverable under 26(b)(1).
                \item (ii) There is substantial need or they cannot be 
                obtained without undue hardship.
            \end{itemize}
            \item (B) Mental impressions, etc. are excluded [i.e., only facts 
            are discoverable].
            \begin{itemize}
                \item [How would \emph{Hickman} be decided under this rule?]
            \end{itemize}
            \item (C) People can access their own statements.
        \end{itemize}
    \end{itemize}
    \item (c) With good cause, the court can protect material from discovery.
    \item (d) After the initial discovery meeting under 26(f), discovery can 
    proceed in any sequence.
    \item (e) Supplements and corrections are sometimes required.
    \item (f) Conference of the parties; planning for discovery.
    \begin{itemize}
        \item (1) Parties must confer as soon as practicable.
        \item (2) Parties must submit a discovery plan within 14 days.
        \item (3) Requirements for the discovery plan.
    \end{itemize}
\end{itemize}

\subsection{FRCP 29: Stipulations about Discovery Procedure}

\subsection{FRCP 30: Depositions by Oral Examination}

\begin{itemize}
    \item (a) When a deposition may be taken.
    \item (b) Notice of the deposition; other formal requirements.
    \begin{itemize}
        \item (1) Notice.
        \item (2) Producing documents.
        \item (6) Notice or subpoena directed to an organization.
    \end{itemize}
    \item (c) Examination and cross-examination; record of the examination; 
    written questions.
    \begin{itemize}
        \item (1) Conducted as they would be at trial.
    \end{itemize}
    \item (d) Duration; sanction; motion to terminate or limit.
    \begin{itemize}
        \item Limited to one day of seven hours.
    \end{itemize}
\end{itemize}

\subsection{FRCP 33--36}

\begin{itemize}
    \item 33: Interrogatories to Parties.
    \item 34: Producing Documents, Electronically Stored Information, and 
    Tangible Things, or Entering onto Land, for Inspection and Other Purposes.
    \item 35: Physical and mental examinations.
    \item 36: Requests for admission.
\end{itemize}

\subsection{Privileges and Sanctions}

\begin{enumerate}
    \item There were several privileged relationships under the common law, 
    including attorney-client, spouse-spouse, doctor-patient, clergy-penitent, 
    and others that vary by state. There is also the Fifth Amendment privilege 
    against self-incrimination.
    \item Privileged information is not discoverable even if it is relevant to 
    the litigation.
    \item Privilege protects the source, not the information itself.
    \item Privilege can be waived, even unintentionally.
\end{enumerate}

\subsubsection{Work Product Privilege: \emph{Hickman v. Taylor}}

The attorney's \textbf{work product} is privileged. Work product includes 
unsworn witness statements, internal memoranda, and the attorney's mental 
impressions.

\begin{enumerate}
    \item Enabling broad discovery vs. encouraging adversarial hearings.
    \item ``Work product'' is material prepared in anticipation of trial.
    \item Facts:
    \begin{enumerate}
        \item February 7, 1943: tug J.M. Taylor sank; five of nine crew 
        drowned.
        \item March 4, 1943: public hearing at United States Steamboat 
        Inspectors where the four survivors testified.
        \item March 29, 1943: Fortenbaugh privately interviewed survivors and 
        took signed statements. (He conducted other interviews, too, and in 
        some cases made memoranda.)
        \item November 26, 1943: estate of the fifth dead crew member (here, 
        petitioner) brought suit against the two tug owners and the railroad.
    \end{enumerate}
    \item Petitioner requested via interrogatories copies of all 
    incident-related records, including signed witness statements, unwritten 
    oral statements, and Fortenbaugh's own memoranda. The court in the Eastern 
    District of Pennsylvania held that the materials were not privileged and 
    ordered disclosure. When the respondents refused, the court held them in 
    contempt. (The contempt order allowed Fortenbaugh to appeal immediately, 
    rather than wait for the trial to end.)
    \begin{enumerate}
        \item Under 26(b)(1) alone, these materials would have been 
        discoverable, and they were not protected by attorney-client 
        privilege.
    \end{enumerate}
    \item The Third Circuit reversed, holding that the requested information 
    was privileged because it was part of the \textbf{``work product of the 
    lawyer.''}\footnote{Casebook p. 919.}
    \item Justice Murphy:
    \begin{itemize}
        \item Discovery has two purposes: (1) narrow and clarify the issues in 
        the case and (2) ascertain the facts.
        \item The specific rule in question here is irrelevant; what matters 
        is the question of whether ``materials collected by an adverse party's 
        counsel in the court of preparation for possible litigation'' are 
        privileged.\footnote{Casebook p. 920.}
        \item Hickman (petitioner) argued that the privilege to withhold 
        material must be limited to the attorney-client privilege.
        \item The attorney-client privilege is not an appropriate framework to 
        answer the question in this case because that privilege does not cover 
        witness interviews or subsequent memoranda, etc.
        \item Here, Hickman has requested information that he could have 
        discovered for himself (e.g., testimony of known witnesses). Hickman 
        has failed to show that disclosure of these materials is necessary or 
        that nondisclosure would lead to injustice.
        \item The Supreme Court was generally enthusiastic of the liberal FRCP 
        disclosure regime, which was still new at the time.\footnote{Casebook 
        p. 920.} Requiring disclosure of an attorney's work product would lead 
        to ``[i]nefficiency, unfairness, and sharp 
        practices.''\footnote{Casebook p. 923.} Other arguments against 
        disclosure (in Bradt's framing):
        \begin{enumerate}
            \item It would create a disincentive to create the work product in 
            the first place, which might make for worse lawyering.
            \item ``Sweat of the brow'' argument: it would be unfair to let an 
            attorney freeride on an opposing attorney's work.
            \item Mental impressions are inherently private.
        \end{enumerate}
        \item Affirmed.
    \end{itemize}
    \item Justice Jackson, concurring:
    \begin{itemize}
        \item Restricting discovery preserves adversarial proceedings.
        \item Requiring lawyers to disclose their work product would force 
        them to become witnesses for ``other witnesses' 
        stories.''\footnote{Casebook p. 925--926.}
    \end{itemize}
    \item Parts of \emph{Hickman} codified in FRCP 26(b)(3):
    \begin{enumerate}
        \item The definition of work product and its broad protection.
        \item ``But a common law trial is and always should be an adversary 
        proceeding. Discovery was hardly intended to enable a learned 
        profession to perform its functions either without wits or on wits 
        borrowed from the adversary.''\footnote{Casebook p. 925.}
        \item Work product immunity is limited when there is ``substantial 
        need'' or the threat of ``undue hardship.''
    \end{enumerate}
\end{enumerate}

\subsubsection{Work-Product Doctrine}

\begin{enumerate}
    \item In 1970, the FRCP were amended to include an exception for 
    work-product materials in 26(b)(3): material ``prepared in anticipation of 
    litigation or for trial~.~.~.~''
    \item The work-product doctrine can be overcome only if the information 
    cannot be obtained or can only be obtained with great 
    difficulty.\footnote{Casebook p. 929.}
    \item Courts (including the \emph{Hinckman} court) are divided on whether 
    attorneys must disclose unrecorded recollections of witnesses' statements.
    \item Work product is only protected if it was prepared specifically for 
    the possibility of litigation (though the specific case may not matter).
    \item Witnesses must still be disclosed before trial.
    \item Discovery sanctions:
    \begin{enumerate}
        \item FRCP 26(g): courts must sanction violations of the 26(g) 
        certification requirement.
        \item FRCP 30: failure to attend a deposition or to subpoena a witness 
        for a deposition is sanctionable.
        \item FRCP 37: (1) If a court orders disclosure, it must award costs 
        to the winning party. (2) If a party disobeys a judicial order, the 
        court must order it to pay expenses.
        \item The severity of sanctions are left to district 
        courts.\footnote{Casebook p. 977.}
    \end{enumerate}
\end{enumerate}

\subsubsection{FRCP 37: Failure to Make Disclosures or to Cooperate in 
Discovery; Sanctions.}

\begin{itemize}
    \item (a) Motion for an order compelling disclosure or discovery.
    \begin{itemize}
        \item (1) Must include a showing of a good faith attempt at discovery 
        or obtaining disclosure without court action.
    \end{itemize}
    \item (b) Failure to comply with a court order.
\end{itemize}
