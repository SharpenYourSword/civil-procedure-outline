\section{The Governing Law in the Federal Courts}

\begin{enumerate}
    \item Holmes: ``The common law is not a brooding omnipresence in the sky, 
    but the articulate voice of some sovereign or quasi sovereign that can be 
    identified...''\footnote{\emph{Southern Pac. Co. v. Jensen} 244 U.S. 205, 
    222 (1917).}
    \item ``Rules of Decision Act'': 
    ``\emph{And be it further enacted} That the laws of the several states, 
    except where the constitution, treaties, or statutes of the United States 
    shall otherwise require or provide, shall be regarded as rules of decision 
    in trials at common law in the courts of the United States in cases where 
    they apply.''\footnote{\S\ 34 of the Judicial Act of 1789, now codified in 
    similar form in 28 U.S.C. 1652. See Casebook p. 489.}
\end{enumerate}

\subsection{General Law over State Law: \emph{Swift v. Tyson}}

Where state law deviates from general law, federal courts should follow 
general law.

\begin{enumerate}
    \item Swift owned a bill of exchange that Tyson originally made to two 
    other men, who endorsed it over to Swift. Tyson refused to pay because of 
    a breach of the original contract for which the bill of exchange was 
    originally issued.
    \item Swift sued Tyson in federal court for payment. Under 
    ``local`` law (New York state law), which the state court followed, 
    Tyson's defense was valid. Under ``general'' law, which federal courts 
    followed, the defense was not valid against Swift.
    \item \S\ 34 of the Judiciary Act of 1789 required the application of ``the 
    laws of the several states\ldots in cases where they apply'' unless 
    federal law otherwise required (roughly equivalent to 28 U.S.C. \S 1652).
    \item In a unanimous opinion, Justice Story held that the federal court 
    ``should follow the general law rather than a state's local law in cases 
    where the state law deviated from the general law.''\footnote{Casebook p. 
    491.}
    \item After the Civil War, the economic interests of the states began to 
    diverge (north: finance, manufacturing; south: agriculture, labor). The 
    Supreme Court expanded general law to include torts, so industrial 
    accidents were increasingly litigated in federal courts. Federal courts 
    also grew increasingly sympathetic to creditors and employers. Those who 
    favored the results of state courts became enemies of \emph{Swift}.
\end{enumerate}

\subsection{No Federal General Common Law: \emph{Erie R.R. Co. v. Tompkins}}

``Except in matters governed by the Federal Constitution or by acts of 
Congress, the law to be applied in any case is the law of the state....  
\textbf{There is no federal general common law}.''\footnote{Casebook p. 497.}

\begin{enumerate}
    \item A passing train injured Tompkins was while he was 
    walking on a footpath along a railroad track in Pennsylvania. He brought 
    suit against the Erie Railroad Co. in federal court in New York.
    \item Under Pennsylvania state law, 
    Tompkins would have been considered a trespasser and therefore not 
    entitled to recover damages. Under general law, the railroad might have 
    been held negligent.
    \item The legal circumstances were unusual. At the time, ``general law'' 
    usually benefited corporations, while ``local law'' usually favored 
    individuals. In this case, however, the plaintiff argued for the 
    application of general law.
    \item The trial court and appellate court, following \emph{Swift}, found 
    that since no state statute governed the issue at hand, general law should 
    control. They found for the plaintiff. The issue before the Supreme Court 
    was whether the district court was free to disregard Pennsylvania common 
    law.
    \item Justice Brandeis:
    \begin{enumerate}
        \item The \emph{Swift} court had misinterpreted the intentions 
        of the authors of the Judiciary Act of 1789: ``...the construction 
        given to it by the court was erroneous; and that the purpose of the 
        section was merely to make certain that, in all matters except those 
        in which some federal law is controlling, the federal courts 
        exercising jurisdiction in diversity of citizenship cases would apply 
        as their rules of decision the law of the state, unwritten as well as 
        written.''\footnote{Casebook p. 495.} (The notes point out that the historian on whom the opinion 
        relies, Charles Warren, might have mistakenly interpreted the Rules of 
        Decision Act as requiring federal courts to follow state common law 
        even in areas where federal common law applied.)
        \item \emph{Swift} caused significant ``injustice and 
        confusion''\footnote{Casebook p. 501}---e.g., companies 
        reincorporating in other states in order to establish diversity 
        jurisdiction to have their cases tried in federal court (\emph{Black 
        \& White Taxicab}). ``\emph{Swift v. Tyson} introduced grave 
        discrimination by noncitizens against citizens.''\footnote{Casebook p. 
        496.}
        \item The federal government did not have the power to 
        legislate rules of tort or contract law.\footnote{This quickly became untrue 
        as the Court expanded the federal government's power to regulate these 
        areas under the Commerce Clause.} Federal courts also do not have the 
        power to create rules in these areas.
        \item Reversed and remanded to be decided on the basis of Pennsylvania 
        state law.
        \item The \emph{Swift} rule is overturned.
    \end{enumerate}
    \item ``Except in matters governed by the Federal Constitution or by acts 
    of Congress, the law to be applied in any case is the law of the state.... 
    There is no federal general common law.''\footnote{Casebook p. 497.} 
    Judges often relied on ``general law'' as a way of ignoring state laws that 
    conflicted with their views.
    \item Justice Reed, concurring:
    \begin{enumerate}
        \item It is enough to broaden the \emph{Swift} framework to hold that 
        ``the laws'' in the Rules of Decison Act include state common law, 
        rather than declare the entire \emph{Swift framework} to be 
        unconstitutional.
        \item It's ``questionable'' to say that Congress has no power to 
        declare which substantive laws control in federal courts---moreso 
        because ``[t]he line between procedural and substantive law is 
        hazy.''\footnote{Casebook p. 499.}
    \end{enumerate}
\end{enumerate}

\subsection{28 U.S.C. \S\ 1652} % TODO

\subsection{28 U.S.C. \S\ 2071} % TODO § (a)-(b)

\subsection{``Outcome Determinative'' Test: \emph{Guaranty Trust v. York}}

The \textbf{``outcome determinative'' test}: would it significantly 
affect the outcome of the litigation for a federal court to disregard state 
law? If so, \emph{Erie} holds that the court should follow state law.

\begin{enumerate}
    \item Background:
    \begin{enumerate}
        \item \textbf{Substance vs. procedure}: did \emph{Erie} and the Rules 
        of Decision Act apply to both procedural and substantive law?
        \item Before the FRCP were enacted in 1938, courts were divided into 
        courts of law (in which cases were triable by jury) and courts of 
        equity (where there was no right to a jury).
        \item The \textbf{Conformity Act of 1872} required that in cases 
        \emph{at law}, federal courts must conform to the procedural rules of 
        the states in which they were located. Thus, there were procedural 
        differences between federal courts in different states, but few 
        differences between federal and state courts in the same state.
        \item In cases \emph{in equity}, federal and state courts followed 
        different rules. Federal courts developed their own system of 
        procedural rules for suits brought in equity.\footnote{Casebook p.  
        504.}
        \item The \textbf{Rules Enabling Act of 1934} authorized the Supreme 
        Court (with congressional approval) to develop a national system of 
        procedural rules for federal civil cases. So while \emph{Erie} 
        required federal courts to follow state rules in substantive law, 
        federal courts developed independent procedural rules (codified in the 
        FRCP in 1938).
    \end{enumerate}
    \item York brought suit in 1942 equity in New York federal court for fraud 
    that occurred in 1931. The defendants argued that the New York statute of 
    limitations applied to cases both at law and in equity. The plaintiff 
    argued that the federal rule of laches, which typically applied in equity 
    cases, should apply in this case.
    \item The trial court agreed with the plaintiff and applied the New York 
    statute of limitations. The appellate court reversed, holding that the 
    federal laches doctrine should have applied, and granted summary judgment 
    to the defendants.
    \item Justice Frankfurter:
    \begin{enumerate}
        \item There is not a clear distinction between ``substantive'' and 
        ``procedural'' rights.
        \item This case dealt with a state-created right. When a federal court 
        adjudicates a state-created right solely on the basis of jurisdiction, 
        it becomes ``in effect, only another court of the State.''
        \item Does the statute of limitations affect ``merely the manner and 
        the means'' of the right to recover, or is it ``a matter of 
        substance'' that affects the result of the litigation?
        \item \emph{Erie} did not intend to ``formulate scientific legal 
        terminology'' around the terms ``substantive'' and ``procedural.'' It 
        intended to ensure that outcomes of diversity cases in federal court 
        would be similar to outcomes in state courts.
        \item The \emph{Erie} doctrine does not distinguish between cases at 
        law and cases in equity.
        \item ``The source of substantive rights enforced by a federal court 
        under diversity jurisdiction, it cannot be said too often, is the law 
        of the States. Whenever that law is authoritatively declared by a 
        State, whether its voice be the legislature or its highest court, such 
        law ought to govern in litigation founded on that law, whether the 
        forum of application is a State or a federal court and whether the 
        remedies be sought at law or may be had in equity.''\footnote{Casebook 
        p. 507.}
        \item Reversed and remanded for hearings consistent with the New York 
        statute of limitations.
    \end{enumerate}
    \item Justice Rutledge, dissenting:
    \begin{enumerate}
        \item The distinction between ``substantive'' and ``procedural'' law 
        is arbitrary but important.
        \item Forum states are free to apply their own statutes of 
        limitations, which may be different from those of the state that 
        originally created the substantive right.
    \end{enumerate}
\end{enumerate}

\subsection{Really Regulating Procedure: \emph{Sibbach v. Wilson \& 
Co.}}

According to the Rules Enabling Act, the federal rules ``shall neither 
abridge, enlarge, nor modify the substantive rights of any litigant.'' A rule 
is procedural if it ``really regulates procedure.''

\begin{enumerate}
    \item A federal court required the plaintiff to submit to a mandatory medical 
    examination. The examination would not have been mandatory under Illinois 
    state law. She refused the exam on the grounds that the Rules Enabling Act 
    forbade rules that abridge litigants' substantive rights. The court held 
    that the rule requiring medical examinations was procedural, not 
    substantive.
    \item ``The test must be whether a rule \textbf{really regulates 
    procedure}.''\footnote{Casebook p. 509.}
    \item Justice Frankfurter (and three others) dissented, arguing that the 
    examination rule constituted ``invasion of the person'' and was therefore 
    significantly different from other procedural rules.
\end{enumerate}

\subsection{``Byrd Balancing'': \emph{Byrd v. Blue Ridge Rural Elec. Coop.}}

\textbf{Byrd balancing}: even if a state rule is outcome determinative, 
countervailing federal considerations can tip the balance in favor of applying 
federal rules.

\begin{enumerate}
    \item Byrd sued the defendant in North Carolina district court for 
    negligence for an injury he sustained while connecting power lines.
    \item The defendant argued that the plaintiff was a ``statutory employee'' 
    under the South Carolina Workmen's Compensation Act when the injury 
    occurred, which would mean that the plaintiff was barred from suing and 
    was obliged to accept statutory compensation benefits.
    \item The district court found for Byrd. The Fourth Circuit reversed.
    \item The questions before the Supreme Court were (1) whether the 
    appellate court erred in directing judgment for Blue Ridge without giving 
    Byrd an opportunity to introduce further evidence, and (2) whether Byrd is 
    entitled to a jury to determine factual issues.
    \item Justice Brennan:
    \begin{enumerate}
        \item Blue Ridge argued that a judge, not a jury, should decide the 
        question of immunity. In \emph{Adams v. Davidson-Paxon Co.}, the South 
        Carolina Supreme court held that a judge, not a jury, should determine 
        the question of whether the plaintiff was a statutory employee (and 
        therefore whether the employer is immune from paying damages). The 
        defendant's argument contends that the federal court should follow 
        this state precedent.
        \item \emph{Erie} held that in diversity cases, federal courts 
        must respect state-created rights by state courts. Here, the Supreme 
        Court found that the decision in \emph{Adams} to send the immunity 
        question to a judge was a ``practical consideration''---``merely a 
        form and mode''---``and not a rule intended to be bound up with the 
        definition of the rights and obligations of the parties.''
        \item Mere ``form and mode'' rules can be important if they 
        bear substantially on the outcome of the litigation. It may be in this 
        case that the question of whether immunity should be decided by a 
        judge or jury would bear substantially on the outcome. Here, however, 
        there is a general federal policy of sending factual questions to the 
        jury. Even though it may affect the outcome, the federal rule should 
        outweigh the state rule because a federal procedural rule is ``not in 
        any sense a local matter.''
        \item Third, it is not at all clear that sending the immunity question 
        to a jury would result in a different outcome.
        \item The federal rule applies. Reversed and remanded.
    \end{enumerate}
\end{enumerate}

\subsection{\emph{Hanna v. Plumer}}

\begin{enumerate}
    \item 
\end{enumerate}

% two tests: hanna holding and hanna dictum
% see bradt class notes

% The two-prong Hannah test: from 252 F.3d 1260 (11th Cir. 2001)
% To aid courts in this "challenging endeavor" the Supreme Court developed a 
% two-part test in Hanna v. Plumer [...] Under the Hanna test, "when the federal 
% law sought to be applied is a congressional statute or Federal Rule of Civil 
% Procedure, the district court must first decide whether the statute is 
% 'sufficiently broad to control the issue before the court.' " [...]  If the 
% federal procedural rule is "sufficiently broad to control the issue" and 
% conflicts with the state law, the federal procedural rule applies instead of 
% the state law. [...] If the federal rule does not directly conflict with the 
% state law, then the second prong of the Hanna test requires the district court 
% to evaluate "whether failure to apply the state law would lead to different 
% outcomes in state and federal court and result in inequitable administration 
% of the laws or forum shopping." [...]

\subsection{\emph{Walker v. Armco Steel Corp.}}

\begin{enumerate}
    \item todo
\end{enumerate}

\subsection{\emph{Gasperini v. Center for Humanities}}

\begin{enumerate}
    \item todo
\end{enumerate}

\subsection{\emph{Clearfield Trust Co v. United States}}

\begin{enumerate}
    \item todo
\end{enumerate}
