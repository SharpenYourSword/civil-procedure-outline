\section{Jury Trials}

\subsection{Overview and Background}

\begin{enumerate}
    \item Very few cases go to trial.\footnote{Casebook p. 1057.}
    \item As of 2002, 2/3rds of federal trials were heard before a jury, while 
    97\% of all state trials were heard before a judge.\footnote{Casebook p. 
    1058.}
    \item Pre--jury verdict motions: nonsuit, directed verdict (state) or 
    judgment as a matter of law (federal).
    \item Post--jury verdict motions: judgment n.o.v. (state), judgment as a 
    matter of law (federal), or new trial (both).\footnote{Casebook p. 1060.}
    \item A party waives the right to a jury trial if it does not request it. 
    FRCP 38.
    \item Seventh Amendmend guarantee to a jury trial does not apply to the 
    states.\footnote{Casebook p. 1098.}
    \item Six and twelve are the most common jury sizes. Six is the lower 
    limit.
    \item The Supreme Court has upheld non-unanimous verdicts. FRCP 48 
    requires a unanimous verdict unless the parties stipulate otherwise.
\end{enumerate}

\subsection{Judgment as a Matter of Law: \emph{Simblest v. Maynard}}

\begin{enumerate}
    \item A travelling salesman collided with a fire truck in an intersection 
    in Burlington. He claimed that the power went out when he was in the 
    middle of the intersection, including the traffic lights. All other 
    witnesses testified that the power had gone out at least ten minutes 
    before.
    \item He looked to his right when he was half or three-fourths of the way 
    through the intersection. That was the first time he saw the fire truck. 
    He claimed to have not heard the siren nor seen the flashing lights.
    \item The salesman sued for negligence to recover for injuries he suffered 
    from the collision.
    \item At the close of the plaintiff's case, the defendant moved for a 
    directed verdict. Denied.
    \item The jury found for the plaintiff. The defendant moved for judgment 
    n.o.v., which the judge granted.
    \item The plaintiff argued, first, that the district court granted the 
    judgment n.o.v. in error.
    \begin{enumerate}
        \item The court here explained the standard of review for judgments 
        n.o.v. \textbf{``that, without weighing the credibility of the 
        witnesses or otherwise considering the weight of the evidence, there 
        can be but one conclusion as to the verdict that reasonable men could 
        have reached.''}\footnote{Casebook p. 1165.} Evidence must be viewed 
        in the light most favorable to the non-moving party.
        \item A key question is whether the court con consider \emph{all} 
        evidence or only evidence favorable to the non-moving party and all 
        uncontradicted, unimpeached evidence against him. Vermont law would 
        allow all evidence. The plaintiff argued that the federal rules would 
        allow only favorable evidence. The court here held that the federal 
        standard also allows admission of uncontradicted, unimpeached 
        evidence.
        \item Under either rule, however, the court believed it was clear that 
        the plaintiff was guilty of contributory negligence. Thus, the 
        judgment n.o.v. for the defendant was properly granted.
    \end{enumerate}
    \item The plaintiff argued in the alternative that the district court 
    erred in declining to instruct the jury on last clear chance. The court 
    dispatched with this argument with the conclusion that ``the overwhelming, 
    uncontroverted evidence demonstrates that defendant in the exercise of due 
    care simply could not have avoided the accident.''\footnote{Casebook 
    1169.}
    \item The \emph{Simblest} court developed a three-part test for 
    determining whether a jury could reasonably find for the non-moving 
    party:\footnote{Casebook p. 1170 n. 2.}
    \begin{enumerate}
        \item Review all of the evidence in the record. [---but didn't the 
        court interpret the federal standard to only include unfavorable 
        evidence if it's unimpeached and uncontradicted?]
        \item Draw all reasonable inferences in favor of the non moving party.
        \item Not make credibility determinations or weigh the evidence.
    \end{enumerate}
    \item Courts rarely grant pre-verdict motions for judgment as a matter of 
    law, because if the case is successfully appealed, it will have to be 
    retried entirely from the beginning. However, if the judge grants a motion 
    for judgment after a matter of law after the jury has returned a verdict, 
    a successful appeal will only reinstate that jury's verdict.
\end{enumerate}

\subsection{\emph{Sioux City \& Pac R.R. Co. v. Stout}}

\begin{enumerate}
    \item A six-year-old was injured while playing on an unlocked railyard 
    turntable. His parents brought an action for negligence. The jury returned 
    a verdict of \$7,500 for the plaintiff.
    \item The court reasoned that if the evidence led to a reasonable 
    inference of negligence, it was free to find for the plaintiff. In this 
    case, because it was reasonable to think that an unlocked turntable might 
    lead to injuries for kids playing on it, negligence was a reasonable 
    inference.
    \item ``~.~.~.~although the facts are undisputed it is for the jury and not 
    for the judge to determine whether proper care was given, and whether they 
    establish negligence.''\footnote{Casebook p. 1175.}
    \item What distinguishes this case from \emph{Simblest}? The facts are 
    disputed in both. In both, the court held in substance that the facts were 
    in dispute. In \emph{Simblest}, however, the judge did not believe that 
    the facts could give rise to a reasonable finding in favor of the 
    plaintiff. In this case, however, the judge believed it was reasonable to 
    infer the defendant's negligence.
\end{enumerate}

\subsection{FRCP 50: Judgment as a Matter of Law in a Jury Trial; Related 
Motion for a New Trial; Conditional Ruling}

\begin{itemize}
    \item (a) Judgment as as matter of law.
    \item (b) Renewing the motion after trial; alternative motion for a new 
    trial.
    \item (c) Granting the renewed motion; conditional ruling on a motion for 
    a new trial.
    \item (d) Time for a losing party's new-trial motion.
    \item (e) Denying the motion for judgment as a matter of law; reversal on 
    appeal.
\end{itemize}

\subsection{FRCP 59: New Trial; Altering or Amending a Judgment}

\begin{enumerate}
    \item (a) In general.
    \item (b) Time to file a motion for a new trial.
    \item (c) Time to serve affidavits.
    \item (d) New trial on the court's initiative or for reasons not in the 
    motion.
    \item (e) Motion to alter or amend a judgment.
\end{enumerate}

\subsection{Overview of Post-Trial Motions}

\begin{enumerate}
    \item A court can do two things after a jury returns a 
    verdict:\footnote{Casebook p. 1180.}
    \begin{enumerate}
        \item Enter judgment against the verdict-winner---i.e., judgment as a 
        matter of law (federal law) or judgment n.o.v. (state).
        \item Grant a new trial because of weak evidence or procedural errors.
    \end{enumerate}
\end{enumerate}

\subsection{\emph{Tanner v. United States}}

\begin{enumerate}
    \item Petitioners Conover and Tanner were convicted of conspiring to 
    defraud the United States. The Eleventh Circuit affirmed.
    \item The petitioners argued that the district court erred in refusing to 
    admit juror testimony at a post-verdict hearing on juror intoxication 
    during the trial.
    \item Evidence shows that multiple jurors were drinking, smoking 
    marijuana, and taking cocaine throughout the trial. One of the jurors 
    involved reported the activity to the petitioner's attorney. The district 
    court denied motions for leave to interview jurors or for an evidentiary 
    hearing.
    \item Justice O'Connor:
    \begin{enumerate}
        \item The firm common law rule is that juror testimony cannot impeach 
        a jury verdict. A few exceptions exist for ``external'' influence, 
        e.g., when a juror had applied for a job in the D.A.'s office. Juror 
        testimony is not allowed on ``internal'' influences, including 
        allegations of insanity or lack of understanding of English.
        \item The rule exists to protect jurors from harassment and to 
        preserve the jury's independence.
        \item ``There is little doubt that post-verdict investigation into 
        juror misconduct would in some instances lead to the invalidation of 
        verdicts reached after irresponsible or improper juror behavior. It is 
        not at all clear, however, that the jury system could survive such 
        efforts to perfect it.''\footnote{Casebook p. 1184.}
        \item The legislative history of the Federal Rules of Evidence suggest 
        that Congress explicitly rejected a version of the rules ``that would 
        have allowed jurors to testify on juror conduct during deliberations, 
        including juror intoxication.''
        \item Petitioners argue that the refusal to hold an evidentiary 
        hearing at which jurors could testify violates the Sixth Amendment 
        guarantee to a competent jury. The court held that nonjuror evidence 
        and the inadmissibility of juror testimony rendered a post-verdict 
        evidentiary hearing unnecessary.
    \end{enumerate}
    \item Justice Marshall, dissenting:
    \begin{enumerate}
        \item The Federal Rules of Evidence (here, 606(b)) does not preclude 
        juror testimony on conduct before or after deliberations. Because the 
        allegations of misconduct here involved juror activity beyond 
        deliberation, it should have been admitted.
        \item Justice Marshall has a different reading of the legislative 
        history.
        \item Regarding the internal/external influence distinction, as ``a 
        common sense matter, drugs and alcohol \emph{are} outside influences 
        on jury members.
    \end{enumerate}
\end{enumerate}

\subsection{\emph{Spurlin v. Gen. Motors}}

\begin{enumerate}
    \item A school bus crashed in Alabama when its brakes failed. Several 
    suits against General Motors were consolidated. The jury awarded \$70,000 
    in each wrongful death case. The court granted General Motors's motions 
    for judgment n.o.v. and, in the alternative, a new trial on the grounds 
    that the evidence did not support the verdict.  \item On the judgment 
    n.o.v.:
    \begin{enumerate}
        \item The \emph{Boeing v. Shipman} standard for reviewing judgments 
        n.o.v. is that the court should consider \emph{all} evidence in light 
        of all reasonable inferences favorable to the party opposed to the 
        motion. If there is substantial evidence opposed to the motion, it 
        must be denied.\footnote{Casebook p. 1191.}
        \item In this case, witness testimony established substantial evidence 
        that the braking system on the bus was not reasonably safe and thus 
        that GM had breached its duty as a manufacturer.
        \item Judgment n.o.v. reversed.
    \end{enumerate}
    \item On the motion in the alternative for new trial:
    \begin{enumerate}
        \item Seventh Amendment guarantees strong protections for jury 
        verdicts.
        \item District Courts should not grant motions for new trials unless 
        the jury verdict is at least ``against the \emph{great} weight of the 
        evidence.''\footnote{Casebook p. 1195.}
        \item The evidence in this case was ``at best conflicting.'' In such 
        cases, courts are not free to set aside jury verdicts.
        \item Grant of new trial overturned.
    \end{enumerate}
\end{enumerate}

\subsection{Entering and Enforcing Judgments}

\begin{enumerate}
    \item \end{enumerate}


