\section{Overview}


\begin{enumerate}
    \item \textbf{Introduction}.
    \begin{enumerate}
        \item Themes throughout the class:
        \begin{enumerate}
            \item Procedure as \textbf{policy}: how does procedure express values 
            about justice?
            \item Procedure as \textbf{strategy}: how do actors use process 
            strategically?
            \item Procedure and \textbf{power}: whom do rules benefit? Does access 
            to rulemakers matter? Why have rules at all? How do state vs. federal 
            issues come into play?
        \end{enumerate}
        \item Goals of the FRCP: \textbf{just, speedy, and inexpensive} 
        determination.
        \item You can't sleep on your rights. \emph{Kubrick}.
        \item \textbf{Due process analysis}: was there deprivation? Was there 
        \textbf{notice and opportunity to be heard}? \emph{Fuentes}. Usually a 
        hearing is needed before deprivation, and balance the parties' 
        interests. \emph{Doehr}. 
    \end{enumerate}
    \item \textbf{Jurisdiction}.
    \begin{enumerate}
        \item \textbf{Personal jurisdiction}.
        \begin{enumerate}
            \item Territorial power framework---outdated. \emph{Pennoyer}.
            \item New test: ``\textbf{minimum contacts} consistent with 
            traditional notions of fair play and substantial justice.'' 
            \emph{International Shoe}.
            \item \textbf{General jurisdiction}: defendant has substantial 
            enough contacts with the state such that any dispute can be 
            litigated there. \emph{Goodyear} (foreign-manufactured tires 
            caused deaths of Americans).
            \item \textbf{Specific jurisdiction}: PJ is based on contacts 
            related to the specific dispute.
            \item \textbf{Purposeful availment}: benefiting from the state's 
            laws. \emph{World-Wide Volkswagen} (car blowing up in Oklahoma).
            \item \textbf{Purposeful direction}: intending to sell products in 
            the forum state. \emph{J. McIntyre} (tort claim against British 
            manufacturer).
            \item \textbf{Long-arm statutes}: authorize courts within 
            constitutional limits to exercise jurisdiction over people outside 
            their borders. \emph{Bensusan} (restaurant website).
            \item \textbf{Rule 4(k)}: PJ analysis in federal courts. 
            \begin{enumerate}
                \item If the claim arises under \emph{state} law, apply the 
                state long-arm statute.
                \item Under \textbf{federal} law: is there a federal long-arm 
                statute? If so, apply it and check whether the defendant has 
                minimum contacts with the US.  (If there is no personal 
                jurisdiction in any state, you can sue in any district court.) 
                If not, apply the state long-arm statute. 
            \end{enumerate}
            \item \textbf{Physical presence} establishes PJ. \emph{Burnham} 
            (divorce dispute). \textbf{Property presence} does not. 
            \emph{Shaffer} (shares of Greyhound).
            \item Plaintiff must take adequate steps to ensure \textbf{notice and 
            opportunity to be heard}---``notice reasonably calculated, under 
            all the circumstances, to apprise interested parties of the 
            pendency of the action and afford them an opportunity to present 
            their objections.'' \emph{Mullane} (judicial settlement of a 
            common trust fund).
            \item \textbf{Forum selection clauses} are enforceable in contract 
            of adhesion. \emph{Carnival Cruise Lines}.
        \end{enumerate}
        \item \textbf{Subject matter jurisdiction}.
        \begin{enumerate}
            \item Sources: Article III, federal jurisdictional statutes, 
            state long-arm statutes.
            \item \textbf{Cannot be waived} for structural reasons. Parties 
            and courts can raise SMJ issues at any time, as SCOTUS did 
            \emph{sua sponte} in \emph{Mottley} (railroad pass contract 
            dispute).
            \item \textbf{Federal question jurisdiction}.
            \begin{enumerate}
                \item Authorized under Article III and 28 U.S.C. \S\ 1331.
                \item Any federal ``ingredient'' is sufficient.
                \item FQSMJ is available even if there is no federal cause of 
                action in cases that \textbf{``implicate significant federal 
                issues''}. \emph{Grable} (IRS property seizure; 
                ``kaleidoscope'' and ``welcome mat'').
                \item The \textbf{well pleaded complaint rule} holds that 
                anticipated federal defense does not establish FQSMJ. 
                \emph{Mottley} (railroad pass contract dispute).
            \end{enumerate}
            \item \textbf{Diversity jurisdiction.}
            \begin{enumerate}
                \item Article III and 28 U.S.C. \S\ 1332.
                \item DSMJ exists when parties are from diverse states and the 
                \textbf{amount in controversy} is above \$75,000.
                \item \textbf{Complete diversity}: no plaintiff can be from 
                the same state as any defendant. \emph{Mas} (peeping landlord).
                \item \textbf{Corporations} are citizens of their 
                \textbf{state of incorporation} and where their 
                \textbf{``nerve center''} is located. \emph{Hertz} (class 
                action in California).
            \end{enumerate}
            \item \textbf{Supplemental jurisdiction}.
            \begin{enumerate}
                \item A claim without jurisdiction can \textbf{ride the 
                coattails} of a claim \emph{with} valid jurisdiction.
                \item 28 U.S.C. \S\ 1367.
                \item Analysis:
                \begin{itemize}
                    \item Is there a claim with valid jurisdiction?
                    \item Do the supplemental claims arise from \textbf{same 
                    case or controversy}? \S\ 1367(a) and \textbf{Gibbs} 
                    (mining contracts---``common nucleus of operative fact'').
                    \item Is the original claim based on diversity? If yes, 
                    there is no supplemental jurisdiction for claims against 
                    third-party defendants if the claim would destroy complete 
                    diversity. 1367(b) and \emph{Kroger} (crane, 
                    electrocution).
                \end{itemize}
            \end{enumerate}
            \item \textbf{Removal}.
            \begin{enumerate}
                \item State $\rightarrow$ to federal. 28 U.S.C. \S\ 1441 
                (allowing removal) and 1446 (procedure).
                \item Only defendants can remove and all defendants must 
                consent.
                \item Removal does not expand SMJ. \emph{Caterpillar} 
                (employment contracts; no SMJ for private contract disputes).
                \item Plaintiff is \textbf{master of the complaint} and free 
                to bring action in state \emph{or} federal court.
            \end{enumerate}
            \item \textbf{Venue}.
            \begin{enumerate}
                \item Which district can you sue in? 28 U.S.C. \S\ 1390 
                (scope), 1391 (venue generally), 1404 (transfer), 1406 
                (dealing with improper venue). 
                \item Waivable.
                \item Transfer is available between districts.
            \end{enumerate}
            \item \textbf{\emph{Forum non conveniens}}.
            \begin{enumerate}
                \item Is there a more convenient forum where the case should 
                be adjudicated?
                \item Differences in substantive law are insufficient for FNC 
                unless the law in the target forum is egregiously bad.
                \item Successful FNC motions result in dismissal. \emph{Piper} 
                (plane crash in Scotland).
            \end{enumerate}
        \end{enumerate}
    \end{enumerate}
    \item \textbf{Governing law}.
    \item \textbf{Pleading}.
    \item \textbf{Joinder}.
    \item \textbf{Discovery}.
    \item \textbf{Summary judgment}.
    \item \textbf{Jury trials}.
    \item \textbf{Appeals}.
    \item \textbf{Preclusion}.
\end{enumerate}




