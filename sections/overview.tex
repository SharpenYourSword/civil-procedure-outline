\section{Overview}

\subsection{Introduction}

\begin{enumerate}
    \item Themes throughout the class:
    \begin{enumerate}
        \item Procedure as \textbf{policy}: how does procedure express values 
        about justice?
        \item Procedure as \textbf{strategy}: how do actors use process 
        strategically?
        \item Procedure and \textbf{power}: whom do rules benefit? Does access 
        to rulemakers matter? Why have rules at all? How do state vs. federal 
        issues come into play?
    \end{enumerate}
    \item Goals of the FRCP: \textbf{just, speedy, and inexpensive} 
    determination.
    \item You can't sleep on your rights. \emph{Kubrick}.
    \item \textbf{Due process analysis}: was there deprivation? Was there 
    \textbf{notice and opportunity to be heard}? \emph{Fuentes}. Usually a 
    hearing is needed before deprivation, and balance the parties' interests. 
    \emph{Doehr}.  \end{enumerate}

\subsection{Jurisdiction}.

\begin{enumerate}
    \item \textbf{Personal jurisdiction}.
    \begin{enumerate}
        \item Territorial power framework---outdated. \emph{Pennoyer}.
        \item New test: ``\textbf{minimum contacts} consistent with 
        traditional notions of fair play and substantial justice.'' 
        \emph{International Shoe}.
        \item \textbf{General jurisdiction}: defendant has substantial enough 
        contacts with the state such that any dispute can be litigated there. 
        \emph{Goodyear} (foreign-manufactured tires caused deaths of 
        Americans).
        \item \textbf{Specific jurisdiction}: PJ is based on contacts related 
        to the specific dispute.
        \item \textbf{Purposeful availment}: benefiting from the state's laws. 
        Factors: state's interest in adjudicating the dispute, plaintiff's 
        interest in convenient and effective relief, interstate judicial 
        system's interest in efficient resolution, and states' shared interest 
        in substantive social policy.\footnote{Casebook p. 191.}
        \emph{World-Wide Volkswagen} (car blowing up in Oklahoma).
        \item \textbf{Purposeful direction}: intending to sell products in the 
        forum state. \emph{J. McIntyre} (tort claim against British 
        manufacturer).
        \item \textbf{Long-arm statutes}: authorize courts within 
        constitutional limits to exercise jurisdiction over people outside 
        their borders. \emph{Bensusan} (restaurant website).
        \item \textbf{Rule 4(k)}: PJ analysis in federal courts.  
        \begin{enumerate}
            \item If the claim arises under \emph{state} law, apply the state 
            long-arm statute.
            \item Under \textbf{federal} law: is there a federal long-arm 
            statute? If so, apply it and check whether the defendant has 
            minimum contacts with the US.  (If there is no personal 
            jurisdiction in any state, you can sue in any district court.) If 
            not, apply the state long-arm statute.  \end{enumerate}
        \item \textbf{Physical presence} establishes PJ. \emph{Burnham} 
        (divorce dispute). \textbf{Property presence} does not.  
        \emph{Shaffer} (shares of Greyhound).
        \item Plaintiff must take adequate steps to ensure \textbf{notice and 
        opportunity to be heard}---``notice reasonably calculated, under all 
        the circumstances, to apprise interested parties of the pendency of 
        the action and afford them an opportunity to present their 
        objections.'' \emph{Mullane} (judicial settlement of a common trust 
        fund).
        \item \textbf{Forum selection clauses} are enforceable in contract of 
        adhesion. \emph{Carnival Cruise Lines}.
    \end{enumerate}
    \item \textbf{Subject matter jurisdiction}.
    \begin{enumerate}
        \item Sources: Article III, federal jurisdictional statutes, state 
        long-arm statutes.
        \item \textbf{Cannot be waived} for structural reasons. Parties and 
        courts can raise SMJ issues at any time, as SCOTUS did \emph{sua 
        sponte} in \emph{Mottley} (railroad pass contract dispute).
        \item \textbf{Federal question jurisdiction}.
        \begin{enumerate}
            \item Authorized under Article III and 28 U.S.C. \S\ 1331.
            \item Any federal ``ingredient'' is sufficient.
            \item Even if there is no federal cause of action, FQSMJ is 
            available in cases that \textbf{``implicate significant federal 
            issues''}. \emph{Grable} (IRS property seizure; ``kaleidoscope'' 
            and ``welcome mat'').
            \item The \textbf{well pleaded complaint rule}: an anticipated 
            federal defense does not establish FQSMJ. \emph{Mottley} 
            (railroad pass contract dispute).
        \end{enumerate}
        \item \textbf{Diversity jurisdiction.}
        \begin{enumerate}
            \item Article III and 28 U.S.C. \S\ 1332.
            \item DSMJ exists when parties are from diverse states and the 
            \textbf{amount in controversy} is above \$75,000.
            \item \textbf{Complete diversity}: no plaintiff can be from the 
            same state as any defendant. \emph{Mas} (peeping landlord).
            \item \textbf{Corporations} are citizens of their \textbf{state of 
            incorporation} and where their \textbf{``nerve center''} is 
            located. \emph{Hertz} (class action in California).
        \end{enumerate}
        \item \textbf{Supplemental jurisdiction}.
        \begin{enumerate}
            \item A claim without jurisdiction can \textbf{ride the coattails} 
            of a claim \emph{with} valid jurisdiction.
            \item 28 U.S.C. \S\ 1367.
            \item Analysis:
            \begin{itemize}
                \item Is there a claim with valid jurisdiction?
                \item Do the supplemental claims arise from \textbf{same case 
                or controversy}? \S\ 1367(a) and \textbf{Gibbs} (mining 
                contracts---``common nucleus of operative fact'').
                \item Is the original claim based on diversity? If yes, there 
                is no supplemental jurisdiction for claims \emph{by 
                plaintiffs} against third-party defendants if the claim would 
                destroy complete diversity. Goal is to disallow jurisdiction 
                over defendants that would otherwise be unavailable. \S\ 
                1367(b) and \emph{Kroger} (crane, electrocution).
                \item Courts are not required to grant supplemental 
                jurisdiction. \S\ 1367(c).
            \end{itemize}
        \end{enumerate}
        \item \textbf{Removal}.
        \begin{enumerate}
            \item State $\rightarrow$ to federal. 28 U.S.C. \S\ 1441 (allowing 
            removal) and 1446 (procedure).
            \item Removal confers venue on the district court. 1390(c).
            \item Unavailable in diversity cases where a defendant is a 
            citizen of the state.
            \item Only defendants can remove and all defendants must consent.
            \item Removal does not expand SMJ. \emph{Caterpillar} (employment 
            contracts; no SMJ for private contract disputes).
            \item Plaintiff is \textbf{master of the complaint} and free to 
            bring action in state \emph{or} federal court.
        \end{enumerate}
        \item \textbf{Venue}.
        \begin{enumerate}
            \item Which district can you sue in? 28 U.S.C. \S\ 1390 (scope), 
            1391 (venue generally), 1404 (transfer--apply law of original 
            venue), 1406 (dealing with improper venue--apply law of new 
            venue). Courts can dismiss \emph{or} transfer.
            \item Waivable.
            \item Transfer is available between districts.
        \end{enumerate}
        \item \textbf{\emph{Forum non conveniens}}.
        \begin{enumerate}
            \item Is there a more convenient forum where the case should be 
            adjudicated?
            \item Differences in substantive law are insufficient for FNC 
            unless the law in the target forum is egregiously bad.
            \item Successful FNC motions result in dismissal. \emph{Piper} 
            (plane crash in Scotland).
        \end{enumerate}
    \end{enumerate}
\end{enumerate}

\subsection{Governing law}.

\begin{enumerate}
    \item \textbf{General federal common law}: ``The common law is not a 
    \textbf{brooding omnipresence} in the sky, but the articulate voice of 
    some sovereign or quasi sovereign that can be 
    identified~.~.~.~''---Holmes.
    \item \textbf{\emph{Swift}}: RDA (28 U.S.C. \S\ 1652) means federal courts 
    should apply \textbf{federal} substantive common law.
    \item \textbf{\emph{Erie}}: RDA means federal courts should apply 
    \textbf{state} substantive common law. No more federal general common law.
    \item REA (28 U.S.C. \S\S\ 2071--77) authorizes FRCP----``shall not 
    abridge, enlarge, or modify any substantive right'' (2072).
    \item \textbf{Outcome determinative test}: if choice of law would 
    significantly affect the outcome, apply state substantive law.  
    \emph{York} (statute of limitations in a fraud case).
    \item \textbf{\emph{Byrd} balancing}: strong federal interests can weigh 
    in favor of applying state law. \emph{Byrd} (employment status of a worker 
    injured while connecting power lines).
    \item A rule is procedural if it \textbf{really regulates procedure}.  It 
    can be substantial without being substantive. \emph{Sibbach} (compulsory 
    medical exam).
    \item \textbf{\emph{Hanna} tests}: when to apply federal procedural rules.
    \begin{enumerate}
        \item What is the source of the procedural rule?
        \begin{enumerate}
            \item Congressional rule, i.e., FRCP or statute (``\emph{Hanna} 
            holding''):
            \begin{itemize}
                \item Is the federal rule \textbf{pertinent}? I.e., is there 
                an unavoidable conflict with a state procedural rule? If no, 
                apply the federal rule. If yes:
                \item Is the federal rule \textbf{valid}?\footnote{No FRCP has 
                ever been found to be invalid, thought the Court has 
                interpreted them narrowly to preserve their validity.}
                \begin{itemize}
                    \item Is it \textbf{constitutional}? I.e., is it 
                    ``rationally capable of classification as procedural?''
                    \item Is it consistent with the \textbf{Rules Enabling 
                    Act}? I.e., does it ``really regulate procedure'' 
                    (\emph{Sibbach}) without abridging substantive rights?
                \end{itemize}
                \item $\rightarrow$ If the rule is invalid, apply the state 
                rule. Otherwise, apply the federal rule.
            \end{itemize}
        \end{enumerate}
        \begin{enumerate}
            \item Judge-made rule (``\emph{Hanna} dictum''):
            \begin{itemize}
                \item Is the difference between the state and federal rules \textbf{outcome determinative}? If not, apply the federal rule. If yes:
                \item Does the difference implicate the twin aims of 
                \emph{Erie?}
                \begin{enumerate}
                    \item Does the difference encourage \textbf{forum 
                    shopping}?
                    \item Does the difference cause \textbf{inequitable 
                    administration of the law}. Can you explain to your client 
                    why being in federal vs. state court makes a difference?
                \item If the difference \emph{does} implicate both the twin 
                aims, apply the state rule. Otherwise, apply the federal rule.
                \end{enumerate}
            \end{itemize}
        \end{enumerate}
    \end{enumerate}
    \item Federal substantive common law still applies under rare 
    circumstances. \emph{Clearfield Trust} (stolen WPA paycheck).
\end{enumerate}

\subsection{Pleading}.

\begin{enumerate}
    \item Four parts (FRCP 8 and 10):
    \begin{enumerate}
        \item \textbf{Parties}: who are we?
        \item \textbf{PJ, SMJ, Venue}: how are we here?
        \item \textbf{Subject of the case}: why are we here?
        \item \textbf{Remedy sought}: what do we want?
    \end{enumerate}    
    \item \textbf{Defenses} (12):
    \begin{itemize}
        \item (b) Defenses by motion. Must be made before responsive pleading.
        \item (g) Motions can be joined. The following are \textbf{waived} if 
        not brought with the first motion: (2) PJ, (3) venue, (4) insufficient 
        process, (5) insufficient service of process.
        \item (h)(2) 12(b)(6), compulsory joinder, and defenses are 
        \textbf{not waived} and can be raised in pleadings, in 12(c) motions, 
        or at trial.
        \item h(3) SMJ can be raised at any time.
    \end{itemize}
    \item \textbf{Motion to dismiss for failure to state a claim} (12)(b)(6):
    \begin{enumerate}
        \item Defendant makes the motion.
        \item Accept the plaintiff's allegations as true.
        \item View the evidence in the light most favorable to the plaintiff.
        \item Deny unless ``clear that no relief could be granted under any 
        set of facts that could be proved consistent with the complaint.'' 
        \emph{Swierkiewicz}.
        \item Liberal policy of leave to amend in response to a 12(b)(6) 
        motion.
    \end{enumerate}
    \item Original pleading requirements were liberal: \textbf{``short and 
    plain statement''}. 8(a).
    \item \emph{Conley}: do not grant 12(b)(6) ``unless it appears beyond 
    doubt that the plaintiff can prove no set of facts in support of his claim 
    which would entitle him to relief.''
    \item \emph{Swierkiewicz}: only need to plead a legally cognizable claim. 
    No need to produce evidence.
    \item \emph{Twombly}: a pleading must be not merely \emph{conceivable} but 
    \emph{plausible.}
    \begin{enumerate}
        \item Pleading must contain a short and plain statement (Rule 8).
        \item The court accepts facts as true.
        \item The court draws all reasonable inferences in favor of the 
        plaintiff.
        \item The court determines whether the complain is \emph{plausible on 
        its face}, meaning it must be more than a mere possibility and not 
        merely consistent with the defendant's liability.
        \item Plaintiffs must ''nudge[] their claims across the line from 
        conceivable to plausible.''
    \end{enumerate}
    \item \textbf{Pleading in the alternative}: can plead multiple 
    inconsistent claims---spaghetti on the wall. \emph{McCormick} (drunk 
    driving death).
    \item \textbf{Rule 11} covers attorney duties and sanctions.  \emph{Zuk} 
    (sanctions for legally and factually inadequate claims in a university 
    copyright dispute).
    \item \textbf{Responsive pleadings} must explicitly affirm or deny each 
    element of a claim. \emph{Zielinksi} (forklift jousting).
\end{enumerate}

\subsection{Joinder, Counterclaims, Crossclaims, Impleader}.

\begin{enumerate}
    \item Each claim requires supp. J or independent SMJ.
    \item Supp. J always applies except in diversity cases with specific 
    circumstances. \S\ 1367, \emph{Gibbs}, \emph{Kroger}.
    \item \textbf{Joinder of claims} (18(a)): any pleader can join claims. 
    Each requires supp. J. 
    \item \textbf{Joinder of parties}:
    \begin{enumerate}
        \item \textbf{Permissive} (20): (a)(1) \textbf{multiple 
        plaintiffs} (same T\&O and common Q), (a)(2) plaintiff can name 
        \textbf{multiple co-defendants} (same requirements). Each party must 
        meet jurisdiction requirements. D's can pool claims to meet amount in 
        controversy.
        \item \textbf{Compulsory} (19) when:
        \begin{enumerate}
            \item there can be no relief (19(a)(1)(A)), or
            \item it would prejudice the absentee (19(a)(1)(B)(i)), or
            \item it would cause one of the current parties to be exposed to 
            multiple or inconsistent obligations (but not necessarily 
            inconsistent \emph{judgments}) (19(a)(1)(B)(ii)).
        \end{enumerate}
        \item If an SMJ or PJ problem prevents joinder, court can decide to 
        proceed or dismiss. 19(b) and \emph{Helzberg} (jewelry store lease).
    \end{enumerate}
    \item Two types of \textbf{counterclaims}:
    \begin{enumerate}
        \item \textbf{Compulsory}: same T\&O. Forfeited if not raised. 13(a), 
        \emph{Jones v. Ford} (racial discrimination and car loans).
        \item \textbf{Permissive}: any non-compulsory counterclaim. It can be 
        completely unrelated. \emph{May be} within supp. J if it has a ``loose 
        factual connection'' to the T\&O. 13(b), \emph{Jones v. Ford}.
    \end{enumerate}
    \item A \textbf{third-party defendant} can counterclaim against the 
    original defendant or the original plaintiff. 14(a).
    \item \textbf{Crossclaims} are claims between co-defendants or 
    co-plaintiffs. Same T\&O. 13(g).
    \item \textbf{Impleader} (14(a) and \emph{Banks}):
    \begin{enumerate}
        \item Can implead for liability to plaintiff, but not for independent 
        claims.
        \item Can't be used to suggest different defendants.
        \item Can join a claim in addition to the impleaded claim.
        \item Supplemental jurisdiction applies if same T\&O.
        \item Third-party defendants can assert \textbf{claims of their own} 
        (and supplemental jurisdiction will apply), including:
        \begin{enumerate}
            \item Counterclaims against the third-party plaintiff.
            \item Crossclaims against other third-party defendants.
            \item Counterclaims against the primary plaintiff if (a) same T\&O 
            or (b) if the primary plaintiff asserted a claim directly against 
            the third-party defendant.
            \item Impleader claims against others 
            not already in the suit.
        \end{enumerate}
    \end{enumerate}
    \item \textbf{Intervention} (24):
    \begin{enumerate}
        \item Intervention requires independent SMJ.
        \item \textbf{Intervention of right} (24(a)):
        \begin{enumerate}
            \item ``claims an interest relating to the \textbf{property or 
            transaction}'' at issue;
            \item disposing of the action would ``impair or impede the 
            movant's ability to protect its interest''; and
            \item the interest is not adequately represented by the existing 
            parties.
        \end{enumerate}
        \item \textbf{Permissive} (24(b)): Can intervene with a claim or 
        defense with a common Q. Court had discretion.
    \end{enumerate}
    \item \textbf{Supp. J} over counterclaims: for 
    \emph{compulsory} claims, yes (because they arise from the same 
    T\&O); for \emph{permissive} claims, probably not (because they are 
    unrelated), but maybe if there is a ``loose factual connection'' 
    (\emph{Jones v. Ford}).
\end{enumerate}

\subsection{Discovery}.

\begin{enumerate}
    \item Generally broad and flexible, like the pleading and joinder rules.
    \item Interrogatories, requests for production/inspection, depositions.
    \item Process: informal investigation, discovery plan (26(f)), initial 
    mandatory disclosures (26(a)(1)(A)), depositions (27, 28, 30--32), 
    interrogatories (33), production (34).
    \item Evidence is relevant if ``(a) it has any tendency to make a fact 
    more or less probable than it would be without the evidence; and (b) the 
    fact is of consequence in determining the action.'' F.R.Evid. 401.
    \item Privileges/work product (26(b)(3) and \emph{Hickman} [sunk tugboat]).
    \item Physical/mental exams (35).
    \item Requests for admission (36).
    \item Motions to compel/sanctions (37).
\end{enumerate}

\subsection{Summary judgment}.

\begin{enumerate}
    \item \textbf{Failure to state a claim} (12(b)(6)): tests the sufficiency 
    of \emph{only} the allegations (no facts). Assuming they're true, is there 
    a valid claim? \textbf{Discovery-worthy?}
    \item \textbf{Motion for summary judgment} (56): Any party can raise on 
    any claim or defense. Tests factual allegations and legal contentions. 
    Granted if ``there is not genuine dispute as to any material fact and the 
    movant is entitled to judgment as a matter of law.'' \emph{Adickes} 
    (restaurant discrimination), \emph{Celotex} (asbestos; the defendant can 
    win an SJ motion if there is no evidence that the plaintiff can make its 
    case, but the moving party \textbf{must show with facts} that the other 
    party can't make it's case---a statement is not enough). 
    \textbf{Trial-worthy?}
\end{enumerate}

\subsection{Jury trials}.

\begin{enumerate}
    \item \textbf{JMOL/Renewed JMOL} (50): a reasonable jury could not find 
    for one party based on the evidence. \textbf{Jury-worthy?}
    \item \textbf{New trial} (59): (1) procedural errors or (2) against 
    \textbf{``great weight''}. \emph{Spurlin} (bus brake failure; evidence was 
    ``at best conflicting'').
\end{enumerate}

\subsection{Appeals}.

\begin{enumerate}
    \item \textbf{Final judgment rule} (state \S\ 1257, federal \S\ 1291, 
    interlocutory \S\ 1292) and \textbf{collateral order exception} 
    (\emph{Digital}, appealing from a decision it viewed as final).
    \item Any dismissal is final except for jurisdiction, venue, or rule 19. 
    41(b).
    \item Appellant must have \textbf{preserved the issue}.
    \item 3 standards of review:
    \begin{enumerate}
        \item \textbf{Clear error/abuse of discretion}: judge decisions and 
        bench trials. 52(a)(6).
        \item \textbf{De novo}: pure question of law.
        \item \textbf{Complete absence of proof}: jury trials.
    \end{enumerate}
\end{enumerate}

\subsection{Preclusion}.

\begin{enumerate}
    \item \textbf{Claim preclusion}: the same plaintiff cannot relitigate a 
    \emph{claim} after it's been decided. Meant to encourage P to bring all claims in 
    A1. Same rationale for compulsory counterclaims.
    \begin{enumerate}
        \item Same parties.
        \item Final judgment on the merits (yes: judgment after trial affirmed 
        on appeal; no: PJ, SMJ, venue; gray area: 12(b)(6), SJ, failure to 
        prosecute, dismissal under sanction---which count as adjudications on 
        the merits in federal court [41(b)]).
        \item Same cause of action as A1 (same transaction or occurrence).
        \item Narrow exceptions: agreement/statute, egregious judgment in A1, 
        lack of jurisdiction over A1 (\emph{Staats}).        
    \end{enumerate}
    \textbf{Issue preclusion}: cannot relitigate an \emph{issue} after it's 
    been decided.
    \begin{enumerate}
        \item Same issue.
        \item Actually litigated and decided.
        \item Full and fair opportunity to litigate.
        \item Necessary to the judgment.
        \item Applies \textbf{only to parties in the original suit} 
        (\emph{Taylor}).
    \end{enumerate}
    \item \textbf{Defensive non-mutual collateral estoppel}: shield. A 
    defendant can prevent a plaintiff from relitigating a claim that he had 
    previously asserted and lost against another defendant. 
    \emph{Blonder-Tongue} (patentee whose patent was found invalid cannot 
    relitigate the validity of the patent against another defendant). Gives 
    plaintiffs the incentive to join all potential defendants in the first 
    action.
    \item \textbf{Offensive non-mutual collateral estoppel}: sword. A 
    plaintiff seeks to prevent a defendant from relitigating an issue the 
    defendant had lost in an earlier trial. May be unfair because the 
    defendant may not have had an incentive to vigorously litigate the issue 
    in the earlier case. It encourages a ``wait-and-see'' strategy on 
    plaintiffs' part. The court has discretion to disallow preclusion if (1) 
    the plaintiff could have easily joined A1 and (2) it's not somehow unfair 
    to the defendant. \emph{Parklane} (Parklane cannot relitigate issues about 
    a merger because it had already lost on those issues against the SEC).  
\end{enumerate}
