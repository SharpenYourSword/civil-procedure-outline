\section{Class Actions}

\begin{enumerate}
    \item Class actions can involve either plaintiff classes (common) or
    defendant classes (rare).
    \item Binding determinations can be made upon class members who are
    absent, unnamed, and sometimes unnotified.
    \item ``~.~.~.~in almost all class actions the attorney's financial
    investment, ideological stake in the outcome, and potential to influence
    the conduct of the case is much greater than that of the named class
    representative.''\footnote{Casebook p. 799 n. 1.d.}
\end{enumerate}

\subsection{FRCP 23: Class Actions}

\begin{enumerate}
    \item Rule 23 was revised in 1966 with three goals in
    mind:\footnote{Casebook p. 797.}
    \begin{enumerate}
        \item Define cases where the benefits of a class suit outweigh the
        disadvantages.
        \item Specify that all class suits are binding and define the scope of
        their preclusive effect.
        \item Ensure maximum advantage and fair representation for absent
        class members.
    \end{enumerate}
    \item The rule's specific provisions are:
    \begin{itemize}
        \item (a) Requirements applicable to all class actions.
        \begin{itemize}
            \item (1) \textbf{Numerosity}: the class must be so numerous that 
            joinder is impracticable.
            \item (2) \textbf{Commonality}: there must be questions of law or 
            fact common to the class (a ``not particularly stringent 
            requirement''\footnote{Casebook p. 798 n. 1.b.}).
            \item (3) \textbf{Typicality}: the claims or defenses of the 
            representative party must be typical of those of the class as a 
            whole.
            \item (4) \textbf{Fair and adequate protection of the interests of 
            the class}: the named parties must represent the entire class's 
            interest---e.g., it must avoid conflicts of interest, and the 
            class must not include groups with ``sharply differing 
            interests.''footnote{Casebook p. 799.} The representation by the 
            class attorney must also be adequate.
        \end{itemize}
        \item (b) Types of class actions.
        \begin{itemize}
            \item (1) Class treatment is allowed when (A) individual suits would
            result in incompatible standards of conduct for the non-class or
            (B) individual suits would impair the ability of those who have
            not brought individual suits, e.g., in ``limited fund'' suits
            where the fund is insufficient to adequately cover the number of
            possible individual claims. Generally limited to suits seeking
            injunctive or declaratory relief.
            % todo: examples of 23(b)(1)(A) and (B) suits?
            \item (2) Class treatment is allowed when the party opposing the
            class has ``acted or refused to act on grounds that apply
            generally to the class.'' Civil rights suits are the most common.
            \item (3) Class treatment is allowed when questions common to the
            class ``predominate'' over questions affecting individual class
            members, and class action must be ``superior'' to other methods of
            adjudication. Notice to class members is mandatory and members
            must have the option of opting out of the class (unlike (b)(1) and
            (2) actions).
        \end{itemize}
    \end{itemize}
\end{enumerate}

\subsection{\emph{Chandler v. Southwest Jeep-Eagle, Inc.}}

To establish class certification, a plaintiff must meet all four requirements 
of 23(a) and all of the requirements of one of the three subsections of 23(b).

\begin{enumerate}
    \item Chandler sued Southwest Jeep-Eagle over ``misrepresentations and 
    unfair and deceptive practices in connection with Southwest's standard 
    retail installment contract.'' Chandler sought class certification on two 
    counts.\footnote{Casebook p. 814.}
    \item The court reasoned that to establish class certification, Chandler 
    must pass a two-part test. First, he must meet all requirements of 23(a). 
    Second, he must meet one of the requirements of 23(b)---in this case, 
    23(b)(3).
    \item The court found that Chandler's claims met all four elements of 
    23(a)---numerosity, commonality, typicality, and adequacy of 
    representation.\footnote{Casebook pp. 816--19.} It also held that Chandler 
    met both elements of 23(b)---a predominating common question of law or 
    fact and a showing that a class action is superior to other methods of 
    litigation.\footnote{Casebook pp. 819--21.}
    \item The court granted Chandler's motion for class certification.
\end{enumerate}

\subsection{\emph{Wal Mart Stores, Inc. v. Dukes}}

\begin{enumerate}
    \item Female employees of Wal Mart brought a Title VII action seeking 
    injunctive and declaratory relief, back pay, and punitive damages.
    \item The district court granted class certification under 23(b). The 
    Ninth Circuit substantially affirmed.
    \item Justice Scalia:
    \begin{enumerate}
        \item Class certification was inconsistent with 23(a) because only a 
        corporate policy could implement the widespread discrimination that 
        would satisfy the rule's requirement of a common question of law or 
        fact, and no such policy existed.
        \item Class certification was inconsistent with 23(b)(2) because the 
        rule prevents monetary relief that is not incidental to the injuntive 
        or declaratory relief. Claims for back pay must be evaluated 
        individually. To hold otherwise would abridge Wal Mart's substantive 
        rights.
    \end{enumerate}
\end{enumerate}
