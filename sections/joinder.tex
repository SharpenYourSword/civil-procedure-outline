\section{Joinder}

\subsection{Joinder of Claims, Counterclaims, Cross-claims}

\subsubsection{FRCP 18: Joinder of Claims}

\begin{itemize}
    \item (a) A party can join as many claims, counterclaims, and cross-claims as it has against another party.
    \item (b) A party may join two claims even though one of them is contingent on the disposition of the other.
\end{itemize}

\subsubsection{FRCP 42: Consolidation; Separate Trials}

\begin{itemize}
    \item (a) If actions involve a ``common question of law or fact'':
    \begin{itemize}
        \item (1) Join all matters at issue.
        \item (2) Consolidate the actions.
        \item (3) Issue other orders to avoid cost or delay.
    \end{itemize}
    \item (b) Court can order separate trials.
\end{itemize}

\subsubsection{FRCP 13: Counterclaim and Crossclaim}

\begin{itemize}
    \item (a) Compulsory counterclaim
    \item (b) Permissive counterclaim
    \item ... % todo: finish outline and expand
\end{itemize}

\subsubsection{\emph{Jones v. Ford Motor Credit Co.}}

\begin{enumerate}
    \item Plaintiffs sued Ford Credit, individually and as class representatives, for racial discrimination under the Equal Credit Opportunity Act. 
    \item Ford asserted counterclaims (1) for the plaintiffs' unpaid car loans and (2) conditional counterclaims against any class members in default of their loans.
    \item District court held that Ford's counterclaims were not compulsory counterclaims and declined to exercise jurisdiction over the counterclaims.
    \item Two types of counterclaims:
    \begin{enumerate}
        \item \textbf{Compulsory}: arising from same transaction or occurrence; forfeited if not raised.
        \item \textbf{Permissive}: any claim not arising from the same transaction or occurrence.
    \end{enumerate}
    \item Appellate court here agreed that Ford's counterclaims were permissive.
    \item Compulsory counterclaims \emph{do} establish supplemental jurisdiction.
    \item Third Circuit rejected the view that independent jurisdiction is required for all permissive counterclaims.
    \item Now, permissive counterclaims need only have a ``loose factual connection'' to the original facts---a broader requirement than the \emph{Gibbs} CNOF test.\footnote{Casebook p. 718.}
    \item The court held that (1) supplemental jurisdiction (under 28 U.S.C. § 1367) may be available for permissive counterclaims and (2) the court should not exercise its discretion to not grant supplemental jurisdiction until it has ruled on the plaintiffs' motion for class certification.
\end{enumerate}

\subsubsection{\emph{Fairview Park v. Al Monzo}}

\begin{enumerate}
    \item Fairview sued Robinson Township, Al Monzo, and Maryland Casualty. Al Monzo crossclaimed against Robinson Township, and Robinson Township counterclaimed against Al Monzo.
    \item Court dismissed Fairview's claim against Robinson Township on a statutory basis.
    \item The court then dismissed Al Monzo's crossclaim against Fairview because of a lack of independent jurisdiction (since both were PA residents).
    \item The court entered judgment for Fairview against Al Monzo. 
    \item On appeal, Al Monzo argued that ``it could not be divested of jurisdiction by the Township's dismissal as a primary defendant.''\footnote{Casebook p. 725} 
    \item The appellate court relies on the principle that ``jurisdiction which has once been attached is not lost by subsequent events.''
    \item Since the claim against Robinson was dismissed \emph{not on jusridictional grounds}, Al Monzo's crossclaim can remain.
\end{enumerate}

\subsubsection{Parties and Standing}

\begin{enumerate}
    \item Standing means a party has a claim that can survive a motion to dismiss.
    \item Standing requirements stem from the ``case or controversy'' requirement of Article III.
    \item Rules have been criticized as obscure and complex.
    \item State courts generally follow different and simpler standing rules.
\end{enumerate}

\subsubsection{FRCP 20: Permissive Joinder of Parties}

% \begin{enumerate}
%     \item todo
% \end{enumerate}

\subsubsection{\emph{Kedra v. City of Philadelphia}}

\begin{enumerate}
    \item Plaintiffs brought action under the Civil Rights Act and the Constitution against city of Philadelphia and multiple police officers and officials for brutality, etc.
    \item Defendants first contested the mother's prosecution of the case on behalf of her minor sons. Court rejected this as frivolous.
    \item Defendants second argued that there was an improper joinder of defendant parties because the events, which occurred over fourteen or fifteen months, do not arise from the same transaction, occurrence, or series of transactions or occurrences. Court found that the events \emph{did} arise from the same occurrences, so joinder was proper under FRCP 20(a).
    \item There was also the question of whether joinder of defendants would prejudice some of them. Court held that it would be better to address this question after discovery.
\end{enumerate}

\subsubsection{FRCP 19: Required Joinder of Parties}

% \begin{enumerate}
%     \item todo
% \end{enumerate}

\subsubsection{\emph{Temple v. Synthes Corp.}}

\begin{enumerate}
    \item Temple sued Synthes in district court and the doctor in state court. The district court ordered Temple under FRCP 19 to join the doctor or face dismissal. The appellate court agreed.
    \item Here, the Supreme Court reversed because the doctor and Synthes were jointly and severally liable, which historically are \emph{permissive} parties but not \emph{necessary} parties.
\end{enumerate}

\subsubsection{\emph{Helzberg's Diamond Shops v. Valley West Des Moines Shopping Center}}

\begin{enumerate}
    \item Helzberg's lease with Valley West stipulated that no more than two other full jewelry shops could open in the mall. When the mall signed a lease with a fourth jewelry shop, Lord's, Helzberg sought injunctive relief against Valley West.
    \item Valley West moved to dismiss because Helzberg had failed to join Lord's under FRCP 19. Denied.
    \item It was not feasible for Lord's to be joined because Lord's was not subject to the court's personal jurisdiction. Under rule 19, the court was then required to determine whether Lord's was indispensable. It decided that it was not, because Lord's absence would not prejudice it because all of its rights under its lease are maintained.
    \item ``In sum, it is generally recognized that a person does not become indispensable to an action to determine rights under a contract simply because that person's rights or obligations under an entirely separate contract will be affected by the result of the action.''\footnote{Casebook p. 749.}
\end{enumerate}

\subsubsection{FRCP 14(a)--(b): Third-Party Practice}

\begin{itemize}
    \item (a) When a Defending Party May Bring in a Third Party.
    \begin{itemize}
        \item (1) A defending party may bring in a third party, at which point the defending partybecomes a ``third-party plaintiff.'' It must obtain the court's leave to file a complaint against a third party more than 14 days after serving its original answer.
        \item (2) The third-party defendant:
        \begin{itemize}
            \item (A) Must assert rule 12 defenses.
            \item (B) Must assert 13(a) counterclaims, may assert 13(b) counterclaims, and may assert 13(g) crossclaims against other third-party defendants.
            \item (C) May assert any of the third-party plaintiff's defenses against the primary plaintiff.
            \item (D) May assert claims [counterclaims?] against the primary plaintiff arising from the same subject matter as the plaintiff's claim against the third-party plaintiff.
        \end{itemize}
        \item (3) Primary plaintiff may assert claims against third-party defendant arising from the same subject matter as the plaintiff's claims against the third-party plaintiff. The third-party defendant must assert rule 12 defenses and rule 13(a) counterclaims, and may assert rule 13(b) counterclaims and rule 13(g) crossclaims.
        \item (4) Any party may move to strike, sever, or try separately the third-party claim.
        \item (5) Third-party defendants can assert this rule against any nonparty.
        \item (6) [Admiralty/maritime jurisdictions]
    \end{itemize}
    \item (b) When a plaintiff may bring in a third party: same rules as for defending parties.
\end{itemize}

\subsubsection{\emph{Banks v. City of Emeryville}}

\begin{enumerate}
    \item 
\end{enumerate}

\subsubsection{FRCP 24: Intervention}

\begin{enumerate}
    \item (a) Intervention of right.
    \item (b) Permissive intervention.
    \item (c) Notice and pleading required.
\end{enumerate}

\subsubsection{\emph{Atlantis Dev. Corp. v. United States}}

\begin{enumerate}
    \item 
\end{enumerate}
