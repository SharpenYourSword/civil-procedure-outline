\section{Joinder, Counterclaims, and Crossclaims}

\begin{enumerate}
    \item \textbf{Joinder of claims}: once a party has made a claim against 
    another party, he can join any other claim to the suit. It is 
    \textbf{never required} and the joined claim must have \textbf{independent 
    subject matter jurisdiction}. Rule 18(a).
    \item \textbf{Joinder of parties}:
    \begin{enumerate}
        \item \textbf{Permissive} (Rule 20): (a) \textbf{multiple plaintiffs} have the 
        option to join together (if the claims arise from the same transaction 
        or occurrence and the plaintiffs share a common question of law or 
        fact) or (b) a plaintiff can name \textbf{multiple co-defendants} 
        (under the same requirements as multiple plaintiffs: T\&O and common 
        question).
        \begin{enumerate}
            \item Jurisdiction requirements must be independently satisfied, 
            with the one exception that multiple plaintiffs can pool the 
            amount of their claims to meet the amount in controversy 
            requirement.
        \end{enumerate}
        \item \textbf{Compulsory} (Rule 19):
        \begin{enumerate}
            \item \textbf{Necessary} parties: without them, there can be no 
            relief, or it would prejudice the absentee, or it would prejudice 
            the current parties. Necessary parties must be joined if possible, 
            but the suit continues if they can't. 
            \item \textbf{Indispensable} parties: same as necessary but where 
            joinder is impossible because of jurisdictional problems. The 
            indispensable parties must be joined if possible, and 
            the \textbf{action must be dropped} if they can't. Rule 19(b) and 
            \emph{Helzberg}.
        \end{enumerate}
    \end{enumerate}
    \item There are two types of \textbf{counterclaims}:
    \begin{enumerate}
        \item \textbf{Compulsory}: it arises from the same transaction and 
        occurrence of the other party's claim. The party forfeits that claim 
        if it does not assert it (unless the counterclaim would require an 
        additional party over whom the court cannot get personal 
        jurisdiction). Rule 13(a). \emph{Is} within supplemental jurisdiction. 
        \emph{Jones v. Ford.}
        \item \textbf{Permissive}: any counterclaim that is not compulsory. It 
        need not be related to the same transaction or occurrence. \emph{May 
        be} within supplemental jurisdiction if it has a ``loose factual 
        connection'' to the original events. \emph{Jones v. Ford.}
    \end{enumerate}
    \item Plaintiffs can counter-counter-claim (and that claim can be 
    compulsory).
    \item A \textbf{third-party defendant} can counterclaim against the 
    original defendant or the original plaintiff (but only if the original 
    plaintiff first made a claim against the third-party defendant). Rule 
    14(a).
    \item \textbf{Supplemental jurisdiction} over counterclaims: for 
    \emph{compulsory} claims, yes (because they arise from the same 
    transaction or occurrence); for \emph{permissive} claims, probably not 
    (because they are unrelated).
    \item \textbf{Crossclaims} are claims between co-defendants or 
    co-plaintiffs. Must have arisen from the same transaction or occurrence as 
    the original action or counterclaim (13(g)). The party making the claim 
    must request \textbf{actual relief} rather than just raise a defense. 
    Never compulsory.
    \item \textbf{Impleader} (14(a)):
    \begin{enumerate}
        \item A defendant can \textbf{implead} a \textbf{third-party 
        defendant} he believes is liable to him (e.g., indemnity).
        \item The claim must be derivative of the original claim, i.e., the 
        third-party \emph{plaintiff} can't claim he is not at all liable to 
        the primary plaintiff.
        \item Plaintiffs against whom counterclaims are made can implead those 
        who are liable to them. Rule 14(b).
        \item Service to the third-party defendant can be made within the 
        \textbf{100-mile bulge} of the courthouse.
        \item Supplemental jurisdiction applies.
        \item Primary plaintiffs can assert claims against third-party 
        defendants.
        \item Third-party defendants can assert \textbf{claims of their own}, 
        including:
        \begin{enumerate}
            \item Counterclaims against the third-party plaintiff.
            \item Crossclaims against other third-party defendants.
            \item Counterclaims against the primary plaintiff if (a) it arises 
            from the same transaction or occurrence as the plaintiff's 
            original claim or (b) if the primary plaintiff asserted a claim 
            directly against the third-party defendant.
            \item Impleader claims against others not already in the suit.
        \end{enumerate}
        \item If the primary claim is dismissed, the court has discretion 
        whether to hear the third-party claims.
    \end{enumerate}
    \item \textbf{Intervention}: parties can enter the lawsuit of their own 
    initiative.
    \begin{enumerate}
        \item Intervention requires \textbf{independent subject matter 
        jurisdiction}.
        \item \textbf{Intervention of right} (24(a)): The court's permission is not 
        required. The party can intervene if it:
        \begin{enumerate}
            \item ``claims an interest relating to the \textbf{property or 
            transaction}'' at issue;
            \item disposing of the action would ``impair or impede the 
            movant's ability to protect its interest''; and
            \item the interest is not adequately represented by the existing 
            parties. 
        \end{enumerate}
        \item \textbf{Permissive intervention} (24(b)): The court has discretion on 
        whether to allow the intervention. The party can intervene if it ``has 
        a claim or defense that shares with the main action a \textbf{common 
        question of law or fact}.''
    \end{enumerate}
\end{enumerate}

\subsection{FRCP 18: Joinder of Claims}

\begin{itemize}
    \item (a) A party can join as many claims, counterclaims, and cross-claims 
    as it has against another party.
    \item (b) A party may join two claims even though one of them is 
    contingent on the disposition of the other.
\end{itemize}

\subsection{FRCP 42: Consolidation; Separate Trials}

\begin{itemize}
    \item (a) If actions involve a ``common question of law or fact'':
    \begin{itemize}
        \item (1) Join all matters at issue.
        \item (2) Consolidate the actions.
        \item (3) Issue other orders to avoid cost or delay.
    \end{itemize}
    \item (b) Court can order separate trials.
\end{itemize}

\subsection{FRCP 13: Counterclaim and Crossclaim}

\begin{itemize}
    \item (a) Compulsory counterclaims.
    \item (b) Permissive counterclaims.
    \item ~.~.~.~
\end{itemize}

\subsection{Counterclaims and Supplemental Jurisdiction: \emph{Jones v. Ford 
Motor Credit Co.}}

Compulsory counterclaims \emph{are} within supplemental jurisdiction. 
Permissive counterclaims \emph{may be} within supplemental jurisdiction if it 
has a ``loose factual connection'' to the original events.

\begin{enumerate}
    \item Plaintiffs sued Ford Credit, individually and as class 
    representatives, for racial discrimination under the Equal Credit 
    Opportunity Act.
    \item Ford asserted counterclaims (1) for the plaintiffs' unpaid car loans 
    and (2) conditional counterclaims against any class members in default of 
    their loans.
    \item The district court held that Ford's counterclaims were not compulsory 
    counterclaims and declined to exercise jurisdiction over the 
    counterclaims.
    \item Two types of counterclaims:
    \begin{enumerate}
        \item \textbf{Compulsory}: arising from same transaction or 
        occurrence; forfeited if not raised.
        \item \textbf{Permissive}: any claim not arising from the same 
        transaction or occurrence.
    \end{enumerate}
    \item Appellate court here agreed that Ford's counterclaims were 
    permissive.
    \item Compulsory counterclaims \emph{do} establish supplemental 
    jurisdiction.
    \item Third Circuit rejected the view that independent jurisdiction is 
    required for all permissive counterclaims.
    \item Now, permissive counterclaims need only have a ``loose factual 
    connection'' to the original facts---a broader requirement than the 
    \emph{Gibbs} CNOF test.\footnote{Casebook p. 718.}
    \item The court held that (1) supplemental jurisdiction (under 28 U.S.C. 
    \S 1367) may be available for permissive counterclaims and (2) the court 
    should not exercise its discretion to not grant supplemental jurisdiction 
    until it has ruled on the plaintiffs' motion for class certification.
\end{enumerate}

\subsection{Jurisdiction over Crossclaims: \emph{Fairview Park v. Al Monzo}}

Once the original claim has been dismissed, the court can maintain its 
jurisdiction to hear crossclaims.

\begin{enumerate}
    \item Fairview sued Robinson Township, Al Monzo, and Maryland Casualty. Al 
    Monzo crossclaimed against Robinson Township, and Robinson Township 
    counterclaimed against Al Monzo.
    \item Court dismissed Fairview's claim against Robinson Township on a 
    statutory basis.
    \item The court then dismissed Al Monzo's crossclaim against Fairview 
    because of a lack of independent jurisdiction (since both were PA 
    residents).
    \item The court entered judgment for Fairview against Al Monzo. 
    \item On appeal, Al Monzo argued that ``it could not be divested of 
    jurisdiction by the Township's dismissal as a primary 
    defendant.''\footnote{Casebook p.  725}
    \item The appellate court relied on the principle that ``jurisdiction 
    which has once been attached is not lost by subsequent events.''
    \item Since the claim against Robinson was dismissed \emph{not on 
    jusridictional grounds}, Al Monzo's crossclaim can remain.
\end{enumerate}

\subsection{Parties and Standing}

\begin{enumerate}
    \item Standing means a party has a claim that can survive a motion to 
    dismiss.
    \item Standing requirements stem from the ``case or controversy'' 
    requirement of Article III.
    \item Rules have been criticized as obscure and complex.
    \item State courts generally follow different and simpler standing rules.
\end{enumerate}

\subsection{FRCP 20: Permissive Joinder of Parties}

% \begin{enumerate}
%     \item todo
% \end{enumerate}

\subsection{Series of Transactions or Occurrences: \emph{Kedra v. City of 
Philadelphia}}

An extended series of events can meet the Rule 20(a) requirement of ``the same 
transaction, occurrence, or series of transactions or occurrences.''

\begin{enumerate}
    \item The plaintiffs brought action under the Civil Rights Act and the 
    Constitution against city of Philadelphia and multiple police officers and 
    officials for brutality, etc.
    \item The defendants first contested the mother's prosecution of the case 
    on behalf of her minor sons. The court rejected this as frivolous.
    \item Second, the defendants second that there was an improper joinder of 
    defendant parties because the events, which occurred over fourteen or 
    fifteen months, did not arise from the same transaction, occurrence, or 
    series of transactions or occurrences.
    \item The court found that the events \emph{did} arise from the same 
    occurrences, so joinder was proper under FRCP 20(a).
    \item There was also the question of whether joinder of defendants would 
    prejudice some of them. Court held that it would be better to address this 
    question after discovery.
\end{enumerate}

\subsection{FRCP 19: Required Joinder of Parties}

% \begin{enumerate}
%     \item todo
% \end{enumerate}

\subsection{Permissive and Necessary Parties: \emph{Temple v. Synthes Corp.}}

Parties who are jointly and severally liable for a tort are permissive but not 
compulsory for the purposes of joinder of parties.

\begin{enumerate}
    \item Temple sued Synthes in district court and the doctor in state court. 
    The district court ordered Temple under FRCP 19 to join the doctor or face 
    dismissal. The appellate court agreed.
    \item Here, the Supreme Court reversed because the doctor and Synthes were 
    jointly and severally liable, which historically are \emph{permissive} 
    parties but not \emph{necessary} parties.
\end{enumerate}

\subsection{Necessary vs. Indispensable Parties: \emph{Helzberg's Diamond 
Shops v. Valley West Des Moines Shopping Center}}

Joinder of a party is not compulsory if the absence of that party would not 
prejudice any of the current parties or the absentee.

\begin{enumerate}
    \item Helzberg's lease with Valley West stipulated that no more than two 
    other full jewelry shops could open in the mall. When the mall signed a 
    lease with a fourth jewelry shop, Lord's, Helzberg sought injunctive 
    relief against Valley West.
    \item Valley West moved to dismiss because Helzberg had failed to join 
    Lord's under FRCP 19. Denied.
    \item It was not feasible for Lord's to be joined because Lord's was not 
    subject to the court's personal jurisdiction. Under Rule 19, the court was 
    then required to determine whether Lord's was indispensable. It decided 
    that it was not, because Lord's absence would not prejudice it because all 
    of its rights under its lease are maintained.
    \item ``In sum, it is generally recognized that a person does not become 
    indispensable to an action to determine rights under a contract simply 
    because that person's rights or obligations under an entirely separate 
    contract will be affected by the result of the action.''\footnote{Casebook 
    p. 749.}
\end{enumerate}

\subsection{FRCP 14(a)--(b): Third-Party Practice}

\begin{itemize}
    \item (a) When a Defending Party May Bring in a Third Party.
    \begin{itemize}
        \item (1) A defending party may bring in a third party, at which point 
        the defending partybecomes a ``third-party plaintiff.'' It must obtain 
        the court's leave to file a complaint against a third party more than 
        14 days after serving its original answer.
        \item (2) The third-party defendant:
        \begin{itemize}
            \item (A) Must assert rule 12 defenses.
            \item (B) Must assert 13(a) counterclaims, may assert 13(b) 
            counterclaims, and may assert 13(g) crossclaims against other 
            third-party defendants.
            \item (C) May assert any of the third-party plaintiff's defenses 
            against the primary plaintiff.
            \item (D) May assert claims [counterclaims?] against the primary 
            plaintiff arising from the same subject matter as the plaintiff's 
            claim against the third-party plaintiff.
        \end{itemize}
        \item (3) Primary plaintiff may assert claims against third-party 
        defendant arising from the same subject matter as the plaintiff's 
        claims against the third-party plaintiff. The third-party defendant 
        must assert rule 12 defenses and rule 13(a) counterclaims, and may 
        assert rule 13(b) counterclaims and rule 13(g) crossclaims.
        \item (4) Any party may move to strike, sever, or try separately the 
        third-party claim.
        \item (5) Third-party defendants can assert this rule against any 
        nonparty.
        \item (6) [Admiralty/maritime jurisdictions]
    \end{itemize}
    \item (b) When a plaintiff may bring in a third party: same rules as for 
    defending parties.
\end{itemize}

\subsection{\emph{Banks v. City of Emeryville}}

\begin{enumerate}
    \item % TODO 755
\end{enumerate}

\subsection{FRCP 24: Intervention}

\begin{enumerate}
    \item (a) Intervention of right.
    \item (b) Permissive intervention.
    \item (c) Notice and pleading required.
\end{enumerate}

\subsection{\emph{Atlantis Dev. Corp. v. United States}}

\begin{enumerate}
    \item % TODO 766
\end{enumerate}
