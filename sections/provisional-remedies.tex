\section{Provisional Remedies}

\subsection{\emph{Kubrick}: Statute of Limitations}

\begin{enumerate}
    \item When does the statute of limitations clock begin? \textbf{\emph{United States v. Kubrick}} Plaintiff was rendered partially deaf from neomycin treatment at the VA. He discovered the possibility of malpractice only after the two year statute of limitations had expired. The issue is whether the claim accrues when the plaintiff is aware of the existence and cause of his injury or when he is also aware of the possibility of malpractice. The Supreme Court (White) holds the former (Stevens dissenting).
    \item Rule vs. standard:
    \begin{enumerate}
        \item White: there's a clear rule here.
        \item Stevens: a rule is unnecessary---all we need is a looser standard that can be applied on a case-by-case basis.
    \end{enumerate}
    \item Why does the statute of limitations exist? (see CB 54-55)
    \begin{enumerate}
        \item Protect against the ``cloud of litigation''
        \item Protect against ``stale claims''
        \item Keep the plaintiff from sitting on his rights
    \end{enumerate}
\end{enumerate}

Themes throughout the class:

\begin{enumerate}
    \item Procedure as \emph{policy}: how does procedure express values about justice?
    \item Procedure as \emph{strategy}: how do actors use process strategically?
    \item Procedure and \emph{power}: whom do rules benefit? Does access to rulemakers matter? Why have rules at all? How do state vs. federal issues come into play?
\end{enumerate}

\subsection{Due Process Requirements}

\begin{enumerate}
    \item Fifth Amendment: ``No person shall be ... deprived of life, liberty, or property, without due process of law''
    \item Fourteenth Amendment: ``No \textbf{State} shall ... deprive any person of life, liberty, or property, without due process of law''
    \item Remedies:
    \begin{enumerate}
        \item \emph{Plenary}: Usually awarded at the end of a lawsuit. Usual types: compensatory and punitive damages, injunctions, and declaratory judgments.
        \item \emph{Provisional}: Can be awarded at any time while a lawsuit is pending. Usual types: attachment (seizure of property), temporary restraining orders, preliminary injunctions. They are ``designed to stabilize the situation pending the final disposition of the case or to provide security to the plaintiff so that if she succeeds in obtaining judgment she will be able to enforce it effectively.''\footnote{Casebook p. 46.}
    \end{enumerate}
\end{enumerate}

\subsection{\emph{Fuentes}: Writs of Replevin}

\begin{enumerate}
    \item What are the due process requirements for ex parte decisions?
    \item Do statutes that allow writs of replevin only upon ex parte application and posting of bond violate the Fourteenth Amendment?
    \item \emph{Fuentes v. Shevin} In multiple consolidated cases, a creditor was granted, ex parte, a writ of replevin for the property of a debtor in default (which was allowed by statutes in Florida and Pennsylvania). The court found that these statutes violate the Due Process Clause (Fourteenth Amendment). Absent extraordinary circumstances, due process requires notice and hearing before deprivation.
    \begin{enumerate}
        \item Justice White: Parties whose rights are to be affected are entitled to be heard; and in order that they may enjoy that right they must must first be notified. The right to be heard is a basic part of procedural fairness.
    \end{enumerate}
    \item Key due process protections: \textbf{notice} and \textbf{opportunity to be heard} (in a meaningful way)
\end{enumerate}

Similar cases:

\begin{enumerate}
    \item \emph{Mitchell v. W.T. Grant}: A similar statute in Louisiana was upheld on the grounds that (1) the applicant for the writ must declare the facts in a certified petition or affidavit, and (2) the showing must be made to a judge, not merely a court official.
    \item \emph{North Georgia Finishing, Inc. v. Di-Chem, Inc.}: A similar Georgia statute was struck down because (1) the affidavit can be filed by the petitioner's attorney, who need not have any direct knowledge of the facts of the dispute, and (2) the writ is issuable by a court clerk, not a judge.
\end{enumerate}

The minimum constitutional requirements for valid ex parte prejudgment and seizure appear to be:

\begin{enumerate}
    \item An application grounded in facts.
    \item Issued by a judge, not a court official.
    \item A speedy hearing.
    \item Only applicable to a limited range of transactions\footnote{See California's version of these statues, Casebook p. 82 top.}.
\end{enumerate}

\subsection{\emph{Doehr}: Prejudgment Attachment of Real Estate}

\begin{enumerate}
    \item Does prejudgment attachment of real estate (a) without prior notice or hearing, (b) without extraordinary circumstances, and (c) without a bond satisfy the Due Process Clause in the Fourteenth Amendment? \emph{Doehr}: no.
    \item \emph{Connecticut v. Doehr}: Under a CT statute, DiGiovanni won a \$75,000 prejudgment attachment on Doehr’s home in conjunction with a civil action for assault and battery. The court unanimously held that the CT statute would allow deprivation for cases where the defendant's property claim would fail to convince a jury. Without exigent circumstances, a preattachment hearing is required.
    \item Concurrences:
    \begin{enumerate}
        \item Marshall, Stevens, O'Connor, White: Bonds are also necessary in all cases.
        \item Rehnquist, Blackmun: Liens can serve a useful purpose (e.g., for laborers to enforce their interests over delinquent landowners). Also, the terms ``bond'' and ``exigent circumstances'' are overly vague.
    \end{enumerate}
    \item \emph{Matthews} rules provide a test for whether deprivation meets due process requirements:
    \begin{enumerate}
        \item Consideration of the private interests that the deprivation will affect.
        \item Examination of the risk of erroneous deprivation, and the safeguards in place.
        \item Attention to the interest of the party seeking the judgment remedy (in \emph{Mathews} originally, this was the government; here it's the private plaintiff).\footnote{Casebook p. 85.}
    \end{enumerate}
\end{enumerate}
