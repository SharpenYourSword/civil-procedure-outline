\section{Pleading}

\subsection{FRCP 8}

\begin{enumerate}
    \item The general rules for constructing pleadings and responses.
\end{enumerate}

\subsection{FRCP 10}

\begin{enumerate}
    \item Formal guidelines for pleadings.
\end{enumerate}

% \subsection{\emph{Access Now}}
% 
% \begin{enumerate}
%     \item 
% \end{enumerate}
% 
% \subsection{\emph{Swierkiewicz v. Sorema, N.A.}}
% 
% \begin{enumerate}
%     \item 
% \end{enumerate}
% 
% \subsection{\emph{Bell Atlantic v. Twombly}}
% 
% \begin{enumerate}
%     \item todo
% \end{enumerate}
% 
% \subsection{\emph{Ashcroft v. Iqbal}}
% 
% \begin{enumerate}
%     \item todo
% \end{enumerate}

\subsection{\emph{McCormick v. Kopmann}}

\begin{enumerate}
    \item McCormick, widow of a man killed in a collision with Kopmann, makes two claims that are in question on appeal:
    \begin{enumerate}
        \item Count I: wrongful death against Kopmann.
        \item Count IV: dram shop claim against the Huls.
    \end{enumerate}
    \item Defendants moved for directed verdicts. Denied. Jury returned verdict for \$15,000 against Kopmann and found the Huls not guilty. Kopmann moved for judgment notwithstanding the verdict or for a new trial; both were denied. 
    \item Kopmann appealed, arguing:
    \begin{enumerate}
        \item Counts I and IV are mutually exclusive. The court agreed, but held that this does not prevent them from being pleaded together. The Illinois Civil Practice Act (modeled on FRCP 8(e)(2), allows alternative pleading where the plaintiff is ``genuinely in doubt as to what the facts are and what the evidence will show.''\footnote{Caseboko p. 663.}
        \item Allegations of intoxication in Count IV count as binding admissions. Court held that ``he is not `admitting' anything other than his uncertainty.''\footnote{Casebook p. 664.}
        \item Prima facie case in Count IV means plaintiff was contributorily negligent regarding count one. The court found that the plaintiff did exercise due care and was not intoxicated.
    \end{enumerate}
    \item Affirmed.
    \item Plaintiff can plead claims that are fundamentally inconsistent.
    \item Pleadings are not binding admissions. They're hypotheses subject to proof and disproof.
\end{enumerate}

\subsection{FRCP 11}

\begin{enumerate}
    \item 11(b) requires pleadings to:
    \begin{enumerate}
        \item not have any improper purpose,
        \item contain arguments supported by law, or nonfrivolous arguments for modifying law,
        \item have evidentiary support (or likely evidentiary support) for factual claims, and
        \item have evidentiary support for denials of factual claims.
    \end{enumerate}
    \item 11(c) describes sanctions:
    \begin{enumerate}
        \item Law firms are jointly responsible.
        \item Motions for sanctions must be made separately.
        \item Courts can order parties to show that they have not violated 11(b).
        \item Sanctions must be limited to what's necessary to deter the actor or others.
    \end{enumerate}
\end{enumerate}

\section{\emph{Zuk v. E. Penn. Psychiatric Institute}}

\begin{enumerate}
    \item Zuk sued EPPI for copyright infringement. District court dismissed under FRCP 12(b)(6) (motion to dismiss), and found Zuk and his counsel liable for \$15,000 in sanctions and counsel fees. Zuk settled his liability and his counsel, Lipman, appealed.
    \item EPPI moved for sanctions ``on the grounds essentially that appellant had failed to conduct an inquiry into the facts reasonable under the circumstances and into the law.''\footnote{Casebook p. 671.} The court imposed sanctions based on FRCP 11 and 28 U.S.C. § 1927.
    \item Lipman had no liability under the Copyright Act.
    \item There was no bad faith on Lipman's part. Therefore, the lower court abused its discretion in awarding sanctions based on 28 U.S.C. § 1927.
    \item The lower court did not identify exactly how the sanctions were based on the FRCP and USC rules. The FRCP sanctions were appropriate, because the original plaintiffs' research was deficient both factually and legally, but it is not possible to review them further. Remanded to consider the type and amount of sanctions specific to the FRCP violation.
    \item ``...we conclude that it was in error to invoke without comment a very severe penalty.''\footnote{Casebook p. 678.}
\end{enumerate}

\subsection{Responding to the Complaint: FRCP 12}

\begin{enumerate}
    \item Under 12(b), a party can assert the following defenses by motion.
    \begin{enumerate}
        \item Lack of subject-matter jurisdiction;
        \item lack of personal jurisdiction;
        \item Improper venue;
        \item Insufficient process;
        \item Insufficient service of process;
        \item Failure to state a claim upon which relief can be granted; and
        \item Failure to join a party under Rule 19.
    \end{enumerate}
\end{enumerate}

\subsection{\emph{Zielinski v. Philadelphia Piers}}

\begin{enumerate}
    \item Facts:
    \begin{enumerate}
        \item February 9, 1953: Sandy Johnson crashed a forklift into Frank Zielinski, causing injuries.
        \item February 10, 1953: Carload Contractors, Inc. sent an accident report to its insurance company.
        \item April 28, 1953: Zielinski filed complaint against Philadelphia Piers, Inc., arguing that (1) PPI owned the forklift that Johnson was operating and (2) Johnson was acting as an employee of PPI.
        \item April 29, 1953: complaint was forwarded to the insurance company.
        \item June 12, 1953: PPI's general manager responded to interrogatories 1 through 5.
        \item August 19, 1953: Sandy Johnson testified in a deposition that he was an employee of PPI.
        \item September 27, 1955: Pre-trial conference held where Zielinski learned that the work of moving freight on the piers had been sold to Carload Contractors, Inc., and that Johnson became an employee of Carload.
        \item October 21, 1955: From the answers to supplementary interrogatories, plaintiff learned that the accident report had been submitted to the insurance company.
    \end{enumerate}
    \item In the original complaint, Zielinski claimed that PPI owned the forklift and that Johnson was an employee of PPI. Initially, PPI simply responded, ``Defendant...denies the averments of paragraph 5.''\footnote{Casebook p. 686.}
    \item PPI's response was inadequate. It meant to only deny the claim of employee negligence, but it should have also addressed the question of whether PPI owned and operated the forklift. PPI's response led Zielinski to believe that PPI was the owner and operator of the forklift, but in fact it had been sold to CCI. Zielinski sued the wrong company, but by the time he learned his mistake (September 27, 1955), the statute of limitations had run.
    \item FRCP 8(b) requires responses to claims to explicitly affirm or deny each element of a claim.
    \item The court held that the doctrine of equitable estoppel requires that the case go forward against PPI with the record showing that PPI owned and operated the forklift---even though that wasn't actually the case.
\end{enumerate}

\subsection{FRCP 15}

\begin{enumerate}
    \item todo
\end{enumerate}

\subsection{\emph{Worthington v. Wilson}}

\begin{enumerate}
    \item todo
\end{enumerate}
