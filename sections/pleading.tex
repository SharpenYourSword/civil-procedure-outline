\section{Pleading}

\subsection{FRCP 8: General Rules of Pleading}

\begin{itemize}
    \item (a) Pleading must contain:
    \begin{itemize}
        \item (1) Short and plain statement of jurisdiction.
        \item (2) Short and plain statement of claim.
        \item (3) Demand for relief.
    \end{itemize}
    (b) Defenses; Admissions and Denials:
    \begin{itemize}
        \item (1) General response:
        \begin{itemize}
            \item (A) Short and plain statement of defense to each claim.
            \item (B) Admit or deny allegations.
        \end{itemize}
        \item (2) Denial must respond to substance of allegation.
        \item (3) General denial will deny everything. Otherwise, specific 
        allegations must be separately admitted or denied.
        \item (4) Denial in part requires admission of the part that's true 
        and denial of the rest.
        \item (5) Defense must acknowledge lack of knowledge where it exists, 
        which has the effect of a denial.
        \item (6) Failure to deny = admission.
    \end{itemize}
\end{itemize}

\subsection{FRCP 10: Form of Pleadings}

\begin{itemize}
    \item (a) Caption/names of parties.
    \item (b) Paragraphs must be numbered; independent statements must be 
    separated.
    \item (c) Adoption by reference.
\end{itemize}

% \subsection{\emph{Access Now}}
% 
% \begin{enumerate}
%     \item todo
% \end{enumerate}
% 
% \subsection{\emph{Swierkiewicz v. Sorema, N.A.}}
% 
% \begin{enumerate}
%     \item todo
% \end{enumerate}
% 
% \subsection{\emph{Bell Atlantic v. Twombly}}
% 
% \begin{enumerate}
%     \item todo
% \end{enumerate}
% 
% \subsection{\emph{Ashcroft v. Iqbal}}
% 
% \begin{enumerate}
%     \item todo
% \end{enumerate}

\subsection{\emph{McCormick v. Kopmann}}

\begin{enumerate}
    \item McCormick, widow of a man killed in a collision with Kopmann, makes 
    two claims that are in question on appeal:
    \begin{enumerate}
        \item Count I: wrongful death against Kopmann.
        \item Count IV: dram shop claim against the Huls.
    \end{enumerate}
    \item Defendants moved for directed verdicts. Denied. Jury returned 
    verdict for \$15,000 against Kopmann and found the Huls not guilty. 
    Kopmann moved for judgment notwithstanding the verdict or for a new trial; 
    both were denied.  \item Kopmann appealed, arguing:
    \begin{enumerate}
        \item Counts I and IV are mutually exclusive. The court agreed, but 
        held that this does not prevent them from being pleaded together. The 
        Illinois Civil Practice Act (modeled on FRCP 8(e)(2), allows 
        alternative pleading where the plaintiff is ``genuinely in doubt as to 
        what the facts are and what the evidence will 
        show.''\footnote{Caseboko p. 663.}
        \item Allegations of intoxication in Count IV count as binding 
        admissions. Court held that ``he is not `admitting' anything other 
        than his uncertainty.''\footnote{Casebook p. 664.}
        \item Prima facie case in Count IV means plaintiff was contributorily 
        negligent regarding count one. The court found that the plaintiff did 
        exercise due care and was not intoxicated.
    \end{enumerate}
    \item Affirmed.
    \item Plaintiff can plead claims that are fundamentally inconsistent.
    \item Pleadings are not binding admissions. They're hypotheses subject to 
    proof and disproof.
\end{enumerate}

\subsection{\emph{Zuk v. E. Penn. Psychiatric Institute}}

\begin{enumerate}
    \item Zuk sued EPPI for copyright infringement. District court dismissed 
    under FRCP 12(b)(6) (motion to dismiss), and found Zuk and his counsel 
    liable for \$15,000 in sanctions and counsel fees. Zuk settled his 
    liability and his counsel, Lipman, appealed.
    \item EPPI moved for sanctions ``on the grounds essentially that appellant 
    had failed to conduct an inquiry into the facts reasonable under the 
    circumstances and into the law.''\footnote{Casebook p. 671.} The court 
    imposed sanctions based on FRCP 11 and 28 U.S.C. § 1927.
    \item Lipman had no liability under the Copyright Act.
    \item There was no bad faith on Lipman's part. Therefore, the lower court 
    abused its discretion in awarding sanctions based on 28 U.S.C. § 1927.
    \item The lower court did not identify exactly how the sanctions were 
    based on the FRCP and USC rules. The FRCP sanctions were appropriate, 
    because the original plaintiffs' research was deficient both factually and 
    legally, but it is not possible to review them further. Remanded to 
    consider the type and amount of sanctions specific to the FRCP violation.
    \item ``~.~.~.~we conclude that it was in error to invoke without comment a 
    very severe penalty.''\footnote{Casebook p. 678.}
\end{enumerate}

\subsection{FRCP 11: Signing, Representations, Sanctions}

\begin{itemize}
    \item (a) Attorney (or self-represented party) must sign all papers.
    \item (b) Representations must:
    \begin{itemize}
        \item (1) Not have any improper purpose.
        \item (2) Make claims warranted by law or make nonfrivolous arguments 
        for changing the law.
        \item (3) Have evidentiary support (or probably support) for factual 
        contentions.
        \item (4) Make denials of factual contentions warranted by evidence, 
        reasonable belief, or lack of information.
    \end{itemize}
    \item (c) Sanctions:
    \begin{itemize}
        \item (1) After notice and opportunity to be heard, court may sanction 
        an attorney, firm, or party for 11(b) violations. Law firms must be 
        held jointly responsible.
        \item (2) Motions for sanctions must be made separately. The party has 
        21 days from the date of service to withdraw the challenged paper. The 
        court can award expenses to the prevailing party.
        \item (3) On its own initiative, a court can ask a party to show why 
        it hasn't violated 11(b).
        \item (4) Sanctions must be limited to what is necessary for 
        deterrence. Sanctions can be nonmonetary or monetary.
        \item (5) Limits on monetary sanctions:
        \begin{itemize}
            \item Courts cannot sanction a represented party for violating 
            11(b)(2) (i.e., lawyers are responsible for legal contentions).
            \item Courts cannot impose monetary sanctions unless it issues an 
            11(c)(3) show-cause order before dismissal or settlement.
        \end{itemize}
        \item (6) A sanction order must include an explanation.
    \end{itemize}
    \item (d) Rule 11 does not apply to discovery motions.
\end{itemize}

\subsection{FRCP 12: Defenses and Objections}

\subsection{\emph{Zielinski v. Philadelphia Piers}}

\begin{enumerate}
    \item Facts:
    \begin{enumerate}
        \item February 9, 1953: Sandy Johnson crashed a forklift into Frank 
        Zielinski, causing injuries.
        \item February 10, 1953: Carload Contractors, Inc. sent an accident 
        report to its insurance company.
        \item April 28, 1953: Zielinski filed complaint against Philadelphia 
        Piers, Inc., arguing that (1) PPI owned the forklift that Johnson was 
        operating and (2) Johnson was acting as an employee of PPI.
        \item April 29, 1953: complaint was forwarded to the insurance 
        company.
        \item June 12, 1953: PPI's general manager responded to 
        interrogatories 1 through 5.
        \item August 19, 1953: Sandy Johnson testified in a deposition that he 
        was an employee of PPI.
        \item September 27, 1955: Pre-trial conference held where Zielinski 
        learned that the work of moving freight on the piers had been sold to 
        Carload Contractors, Inc., and that Johnson became an employee of 
        Carload.
        \item October 21, 1955: From the answers to supplementary 
        interrogatories, plaintiff learned that the accident report had been 
        submitted to the insurance company.
    \end{enumerate}
    \item In the original complaint, Zielinski claimed that PPI owned the 
    forklift and that Johnson was an employee of PPI. Initially, PPI simply 
    responded, ``Defendant~.~.~.~denies the averments of paragraph 
    5.''\footnote{Casebook p. 686.}
    \item PPI's response was inadequate. It meant to only deny the claim of 
    employee negligence, but it should have also addressed the question of 
    whether PPI owned and operated the forklift. PPI's response led Zielinski 
    to believe that PPI was the owner and operator of the forklift, but in 
    fact it had been sold to CCI. Zielinski sued the wrong company, but by the 
    time he learned his mistake (September 27, 1955), the statute of 
    limitations had run.
    \item FRCP 8(b) requires responses to claims to explicitly affirm or deny 
    each element of a claim.
    \item The court held that the doctrine of equitable estoppel requires that 
    the case go forward against PPI with the record showing that PPI owned and 
    operated the forklift---even though that wasn't actually the case.
\end{enumerate}

% \subsection{FRCP 15}
% 
% \begin{enumerate}
%     \item todo
% \end{enumerate}

\subsection{\emph{Worthington v. Wilson}}

\begin{enumerate}
    \item Facts:
    \begin{enumerate}
        \item February 25, 1989: Worthington was arrested by two police 
        officers.
        \item February 25, 1991: Worthington filed a complaint against the 
        Village of Peoria Heights in county court against ``three unknown 
        named police officers.'' The Village removed the action to the 
        District Court for the Central District of Illinois.
        \item June 17, 1991: Worthington amended his complaint to identify the 
        two police officers by name.
    \end{enumerate}
    \item The statute of limitations was two years, which had expired when 
    Worthington filed his amended complaint. Defendants argued that the 
    amended complaint cannot related back to the filing date of the original 
    complaint under FRCP 15(c).
    \item In response, Worthington made three arguments:
    \begin{enumerate}
        \item Relation back is governed by an Illinois statute, not 15(c).
        \item The 15(c) requirements are satisfied.
        \item He should not be punished for omitting the officers' names in 
        the original complaint because the Peoria Heights Police Department 
        had withheld that information.
    \end{enumerate}
    \item In \emph{Schiavone v. Fortune} (1986), the Supreme Court held that 
    parties to be brought in by amendment must receive notice of the action 
    before the expiration of the statue of limitations period.
    \item December 1, 1991: FRCP 15 was amended in direct response to 
    \emph{Schiavone}. The new rule allowed complaints to be amended at any 
    time to correct misnamed defendants as long as the defendant was aware of 
    the action within 120 days of the filing of the original complaint. Notice 
    is no longer required.
    \item Under the new version of FRCP 15(c), Worthington's amendment was 
    acceptable.
    \item Defendants also argued that 15(c) does not apply because the defect 
    in Worthington's original complaint was due to lack of knowledge, not a 
    ``mistake.'' The 7th Circuit had previously held that lack of knowledge of 
    the proper defendant does not involve a mistake. The district court here 
    disagreed, but out of deference agreed with defendants and granted their 
    motion to dismiss.
\end{enumerate}



