\section{Pleading}

\subsection{FRCP 8}

\begin{enumerate}
    \item The general rules for constructing pleadings and responses.
\end{enumerate}

\subsection{FRCP 10}

\begin{enumerate}
    \item Formal guidelines for pleadings.
\end{enumerate}

% \subsection{\emph{Access Now}}
% 
% \begin{enumerate}
%     \item 
% \end{enumerate}
% 
% \subsection{\emph{Swierkiewicz v. Sorema, N.A.}}
% 
% \begin{enumerate}
%     \item 
% \end{enumerate}
% 
% \subsection{\emph{Bell Atlantic v. Twombly}}
% 
% \begin{enumerate}
%     \item todo
% \end{enumerate}
% 
% \subsection{\emph{Ashcroft v. Iqbal}}
% 
% \begin{enumerate}
%     \item todo
% \end{enumerate}

\subsection{\emph{McCormick v. Kopmann}}

\begin{enumerate}
    \item McCormick, widow of a man killed in a collision with Kopmann, makes two claims that are in question on appeal:
    \begin{enumerate}
        \item Count I: wrongful death against Kopmann.
        \item Count IV: dram shop claim against the Huls.
    \end{enumerate}
    \item Defendants moved for directed verdicts. Denied. Jury returned verdict for \$15,000 against Kopmann and found the Huls not guilty. Kopmann moved for judgment notwithstanding the verdict or for a new trial; both were denied. 
    \item Kopmann appealed, arguing:
    \begin{enumerate}
        \item Counts I and IV are mutually exclusive. The court agreed, but held that this does not prevent them from being pleaded together. The Illinois Civil Practice Act (modeled on FRCP 8(e)(2), allows alternative pleading where the plaintiff is ``genuinely in doubt as to what the facts are and what the evidence will show.''\footnote{Caseboko p. 663.}
        \item Allegations of intoxication in Count IV count as binding admissions. Court held that ``he is not `admitting' anything other than his uncertainty.''\footnote{Casebook p. 664.}
        \item Prima facie case in Count IV means plaintiff was contributorily negligent regarding count one. The court found that the plaintiff did exercise due care and was not intoxicated.
    \end{enumerate}
    \item Affirmed.
\end{enumerate}

\section{\emph{Zuk v. E. Penn. Psychiatric Institute}}

\begin{enumerate}
    \item Zuk sued EPPI for copyright infringement. District court dismissed under FRCP 12(b)(6) (motion to dismiss), and found Zuk and his counsel liable for \$15,000 in sanctions and counsel fees. Zuk settled his liability and his counsel, Lipman, appealed.
    \item EPPI moved for sanctions ``on the grounds essentially that appellant had failed to conduct an inquiry into the facts reasonable under the circumstances and into the law.''\footnote{Casebook p. 671.} The court imposed sanctions based on FRCP 11 and 28 U.S.C. § 1927.
    \item Lipman had no liability under the Copyright Act.
    \item There was no bad faith on Lipman's part. Therefore, the lower court abused its discretion in awarding sanctions based on 28 U.S.C. § 1927.
    \item The lower court did not identify exactly how the sanctions were based on the FRCP and USC rules. The FRCP sanctions were appropriate, because the original plaintiffs' research was deficient both factually and legally, but it is not possible to review them further. Remanded to consider the type and amount of sanctions specific to the FRCP violation.
    \item ``...we conclude that it was in error to invoke without comment a very severe penalty.''\footnote{Casebook p. 678.}
\end{enumerate}

\subsection{FRCP 11}

\begin{enumerate}
    \item todo
\end{enumerate}

% \subsection{Responding to the Complaint: FRCP 12}
% 
% \begin{enumerate}
%     \item todo
% \end{enumerate}
% 
% \subsection{\emph{Zielinski v. Philadelphia Piers}}
% 
% \begin{enumerate}
%     \item todo
% \end{enumerate}
% 
% \subsection{FRCP 15}
% 
% \begin{enumerate}
%     \item todo
% \end{enumerate}
% 
% \subsection{\emph{Worthington v. Wilson}}
% 
% \begin{enumerate}
%     \item todo
% \end{enumerate}
