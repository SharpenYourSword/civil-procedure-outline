\section{Preclusion}

\begin{enumerate}
    \item \textbf{Res judicata} (claim preclusion): a claim (and related 
    claims) cannot be relitigated after final judgment.\footnote{Casebook p.  
    1224.}
    \item \textbf{Collateral estoppel} (issue preclusion): an issue of fact or 
    law cannot be relitigated after final judgment.
    \item When filing a claim, failing to file a related claim can prevent the 
    claimant from litigating that claim in the future.
\end{enumerate}

\subsection{Consequences of Final Judgment: \emph{Federated Department Stores, 
Inc. v. Moitie}}

Once a claim reaches final judgment, parties cannot raise other claims arising 
from the same transaction or occurrence. Res judicata overrides competing 
policy concerns. It is usually worth keeping a case alive on appeal.

\begin{enumerate}
    \item The government brought an antitrust suit against Federated 
    Department Stores and others. Seven plaintiffs filed civil actions, 
    including Moitie in state court (\emph{Moitie I}) and Brown in the 
    Northern District of California (\emph{Brown I}). The civil claims 
    followed the govenment's claims almost verbatim, although Moitie referred 
    only to state law.
    \item \emph{Moitie I} was removed to district court. All civil claims were 
    directed to the same federal judge.
    \item The district court rejected all of the civil claims for failure to 
    allege an ``injury'' to their ``business or property'' under \S\ 4 of the 
    Clayton Act.\footnote{15 U.S.C. \S\ 15.}
    \item Five of the plaintiffs appelaed in the Ninth Circuit. The lawyer 
    representing Moitie and Brown, however, chose to refile in state court 
    (\emph{Moitie II} and \emph{Brown II}).
    \item \emph{Moitie II} and \emph{Brown II} were removed to federal court.  
    The court found that they were ``in many respects identical'' to the 
    earlier claims. It dismissed them under res judicata.\footnote{Casebook p.  
    1225.}
    \item While \emph{Moitie II} and \emph{Brown II} were pending appeal, the 
    Supreme Court decided \emph{Reiter v. Sonatone Corp.}, holding that 
    retailers \emph{could} allege an ``injury'' to their ``business or 
    property'' under \S\ 4 of the Clayton Act, and accordingly the Ninth 
    Circuit reversed the five cases pending appeal. The Ninth Circuit also 
    reversed the dismissals of \emph{Moitie II} and \emph{Brown II} on the 
    same grounds, even though it violated a strict interpretation of res 
    judicata, becuase ``the doctrine of res judicata must give way to `public 
    policy' and `simple justice.'''\footnote{Casebook p. 1226.}
    \item Justice Rehnquist:
    \begin{enumerate}
        \item ``...such an unwarranted departure from res judicata is 
        unwarranted. Indeed, the decision below is all but foreclosed by our 
        prior case law.''\footnote{Casebook p. 1227.}
        \item ``The doctrine of res judicata serves vital public interests 
        beyond any individual judge's determination of the equities in a 
        particular case.''\footnote{Casebook p. 1229.}
        \item Reversed.
    \end{enumerate}
    \item Justice Blackmun, concurring:
    \begin{enumerate}
        \item There may be cases where policy concerns override res judicata.
        \item \emph{Brown II} should not even been allowed in federal court 
        because \emph{Brown I} was res judicata.
    \end{enumerate}
    \item Justice Brennan, dissenting:
    \begin{enumerate}
        \item Agree with Blackmun that \emph{Brown I} was res judicata.
    \end{enumerate}
\end{enumerate}

\subsection{Claim Preclusion: \emph{Davis v. DART}}

A plaintiff must bring all causes of action related to the same claim. Any 
related actions he fails to bring are barred from future suits.

\begin{enumerate}
    \item 2001: Davis and Johnson alleged race discrimination and retaliation 
    under Title VII and violations of the First and Fourteenth Amendments 
    under 42 U.S.C. \S\ 1983 against Dallas Area Rapid Transit and its Chief 
    of Police. They originally brought the claim in state court and it was 
    removed to Texas district court (\emph{Davis I}). The district court 
    dismissed the claims with prejudice.
    \item 2002: Davis and Johnson brought another action in district court 
    alleging similar (but not identical) claims. The court granted summary 
    judgment for the defendants on the grounds that (1) they failed to raise 
    an issue of fact about whether their nonselection for promotion was 
    racially motivated and (2) res judicata from \emph{Davis I} precluded the 
    remaining claims.
    \item The Fifth Circuit identified four factors for barring claims under 
    res judicata:
    \begin{enumerate}
        \item Identical parties.
        \item Prior judgment from a court of competent jurisdiction.
        \item Prior judmgent that was final and on the merits.
        \item Same cause of action in both suits.
    \end{enumerate}
    \item Only the fourth factor was disputed here. The standard of review for 
    the Fifth Circuit was (1) whether the barred claims were part of the same 
    cause of action (``same nucleus of operative facts'') and (2) whether 
    Davis and Johnson could have advances the barred claims in \emph{Davis 
    I}.\footnote{Casebook p. 1233.}
    \item The court here found that the claims in both cases ``originate from 
    the same continuing course of allegedly discriminatory conduct'' and that 
    the claims could have been brought together (despite the plaintiff's 
    argument that their pending EEOC claim prevented bringing the full action 
    in court).
\end{enumerate}

\subsection{\emph{Staats v. County of Sawyer}}

\begin{enumerate}
    \item % TODO
\end{enumerate}

\subsection{\emph{Levy v. Kosher Overseers of Am.}}

\begin{enumerate}
    \item % TODO
\end{enumerate}

\subsection{\emph{Jacobs v. CBS}}

\begin{enumerate}
    \item % TODO
\end{enumerate}

\subsection{\emph{Taylor v. Sturgell}}

\begin{enumerate}
    \item % TODO
\end{enumerate}

\subsubsection{\emph{Parklane Hosiery v. Shore}}

\begin{enumerate}
    \item % TODO
\end{enumerate}
