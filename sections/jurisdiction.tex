\section{Jurisdiction}

\begin{enumerate}
    \item \textbf{Territorial jurisdiction}: jurisdiction over cases arising in or 
    involving people residing within a defined territory.
    \item \textbf{Personal jurisdiction}: a court's power to bring a person into its 
    adjudicative process.
    \item \textbf{Subject matter jurisdiction}: the court's power to decide a 
    particular type of case.
    \item The Due Process Clause (Fourteenth Amendment) governs jurisdictional 
    questions.
    \item Parties must raise objections to jurisdiction at the beginning of a suit.
\end{enumerate}

\subsection{Personal Jurisdiction}

\subsubsection{Territorial Power: \emph{Pennoyer v. Neff}}

Under the territorial power framework of personal jurisdiction, a court cannot 
establish jurisdiction over someone beyond its physical borders.

\begin{enumerate}
    \item Mitchell sued Neff in Oregon state court for nonpayment of legal 
    fees rendered in 1862--1863. Neff was nowhere to be found, so Mitchell 
    published notice of the suit in the \emph{Pacific Christian Advocate}.
    \item Neff did not appear, so the court granted a default judgment for 
    Mitchell.
    \item Neff's property in Oregon was attached and then sold at auction to 
    Mitchell, who then sold it to Pennoyer.
    \item Eight years later, Neff turned up and sued Pennoyer to recover the 
    property.
    \item Justice Field:
    \begin{enumerate}
        \item The basis for personal jurisdiction is a state's territorial 
        power. States are all-powerful within their borders and powerless 
        beyond.
        \item Service by publication was insufficient to establish personal 
        jurisdiction over a non-resident in an in personam suit (thought it 
        would suffice for in personam suits). Thus, the original judgment 
        against Neff was void.
        \item Doctrine of \textbf{collateral attack}: a judgment void when 
        rendered is void forever.
    \end{enumerate}
    \item \emph{Milliken v. Meyer}: domicile within a state is sufficient to 
    establish jurisdiction.
\end{enumerate}

\subsubsection{Jurisdiction Over Out-of-State Drivers: \emph{Hess v. 
Pawloski}}

States can implement statutes requiring out-of-state drivers to give implied 
consent to personal jurisdiction within that state.

\begin{enumerate}
    \item Hess, a Pennsylvania resident, ``negligently and wantonly drove a 
    motor vehicle on a public highway in Massachusetts,'' causing injury to 
    Pawloski.\footnote{274 U.S. 352, 353 (1927).}
    \item Pawloski brought a negligence suit in Massachusetts state court.  
    Hess contested personal jurisdiction. Denied. Hess appealed on Fourteenth 
    Amendment grounds.
    \item Justice Butler:
    \begin{enumerate}
        \item In earlier cases (e.g., \emph{Kane v. New Jersey}), the Court 
        upheld the constitutionality of statutes requiring out-of-state 
        drivers to appoint an agent to receive service of process.
        \item States can legitimately require drivers to appoint similar agent 
        implicitly, and these kinds of statutes do not not constitute 
        discrimination against non-residents.
        \item Therefore, it is consistent with due process for states to 
        require out-of-state drivers to implicitly appoint an agent to receive 
        process, thereby establishing jurisdiction over those drivers if civil 
        actions arise.
        \item Affirmed.
    \end{enumerate}
\end{enumerate}

\subsubsection{Minimum Contacts: \emph{International Shoe Co. v. Washington}}

\enquote{But now that the capias ad respondendum has given way to personal 
service of summons or other form of notice, due process requires only that in 
order to subject a defendant to a judgment in personam, if he be not present 
within the territory of the forum, he have certain \enquote{\textbf{minimum 
contacts with it such that the maintenance of the suit does not offend 
`traditional notions of fair play and substantial 
justice.}}}\footnote{Casebook p. 179.}
% TODO: replace nested quotes with enquote throughout
\begin{enumerate}
    \item The State of Washington sued International Shoe to recover unpaid 
    contributions to the state unemployment compensation fund.
    \item Justice Stone:
    \begin{enumerate}
        \item International Shoe argued first that the Washington statute 
        imposed an unconstitutional burden on interstate commerce. The court 
        rejected International Shoe's argument on the basis that ``it is no 
        longer debatable'' that the commerce clause gives Congress broad power 
        to regulate interstate commerce.
        \item Second, International Shoe argued that merely soliciting orders 
        within a state ``does not render the seller amenable to suit within 
        the state.'' Historically, physical presence within a state was a 
        prerequisite for jurisdiction in in personam cases (\emph{Pennoyer}). 
        But now, \textbf{minimum contacts} are sufficient.
        \item ``Presence'' is a symbolic term that can refer to business 
        activities within a territory.  It can refer to activities that give 
        rise to the liabilities at issue in the suit. Moreover, since an 
        entity enjoys certain benefits and protections from a state's laws, it 
        also has an obligation to that state.
        \item Washington was entitled to recover the unpaid contributions.
    \end{enumerate}
    \item Justice Black, concurring: states have a constitutional power to tax 
    and sue corporations that do business in the state's territory. The test 
    of ``fair play and substantial justice,'' however, is ``confusing'' and 
    gives the court the unwarranted power to strike down any legislation it 
    might see as violating ``natural justice.''
\end{enumerate}

\subsubsection{General and Specific Jurisdiction}

\begin{enumerate}
    \item \textbf{General jurisdiction}: The defendant has substantial enough 
    contacts with a state that any dispute can be litigated in that state, 
    regardless of whether the dispute arises from those contacts.
    \item \textbf{Specific jurisdiction}: Jurisdiction is based on contacts related to 
    the specific dispute.\footnote{Casebook p. 186.}
\end{enumerate}

\subsubsection{Purposeful Availment: \emph{World-Wide Volkswagen Corp. v. Woodson}}

Selling a product that ends up in another state is insufficient to establish 
personal jurisdiction over the seller, because the seller has not purposefully 
availed itself of the benefits of doing business in the forum state.

\begin{enumerate}
    \item The Robinsons bought an Audi in New York. It caught fire in an
    accident in Oklahoma. They brought a products liability action in Oklahoma 
    state court against every link in the distribution chain. The regional 
    distributor (World-Wide Volkswagen) and retailer (Seaway) contested Oklahoma's 
    jurisdiction.
    \item Justice White: \begin{enumerate}
        \item To establish jurisdiction, defendants must have minimum contacts 
        with the forum state and jurisdiction must not violate ``traditional 
        notions of fair play and substantial justice'' (\emph{International 
        Shoe}).
        \item Elements of ``fair play and substantial justice'' 
        include:\footnote{Casebook p. 190.}
        \begin{enumerate}
            \item The forum State's interest in adjudicating the dispute.
            \item The convenience of the venue for the plaintiff.
            \item The interstate judicial system's interest in efficient 
            resolution.
            \item States' shared interest in fundamental policy goals.
        \end{enumerate}
        \item Jurisdictional rules have been relaxed since \emph{Pennoyer}, 
        but the Constitution nonetheless privileges state sovereignty.  \item 
        ``Petitioners [World-Wide and Seaway] carry on no activity whatsoever 
        in Oklahoma.''\footnote{Casebook p. 192.}
        \item Petitioners could not reasonably predict being haled into court 
        in Oklahoma. Corporations can ``purposely avail'' themselves of the 
        benefits of conducting activity in the forum state---but there is no 
        such availment here.\footnote{Casebook p. 194.} Were it otherwise, 
        ``[e]very seller of chattels would in effect appoint the chattel his 
        agent for service of process.''\footnote{Casebook p. 193.}
        \item No contacts, so no jurisdiction.
    \end{enumerate}
    \item Justice Brennan, dissenting: The Court reads \emph{International 
    Shoe} too narrowly. The seller and dealer purposefully injected their 
    product ``into the stream of interstate commerce,'' thus establishing 
    minimum contacts.\footnote{Casebook p. 196.}
    \item Justices Marshall, dissenting: The dealer and seller chose to become 
    part of a global marketing and servicing network. Cars derive their value 
    from being mobile. The dealer and seller received economic advantage from 
    the ability to draw revenue from Oklahoma.
    \item Justice Blackmun, dissenting: It is confusing why the distributor 
    and seller are getting sued here. Also, cars are mobile by nature.
\end{enumerate}

\subsubsection{Purposeful Availment II: \emph{Burger King Corp. v. Rudzewicz}}
% long arm or long-arm? make consistent throughout
Establishing a franchise relationship with an out-of-state corporation is 
sufficient to constitute purposeful availment.

\begin{enumerate}
    \item Rudcewicz and MacShara operated a Burger King franchise in Michigan. 
    They fell behind on their payments. Burger King, based in Florida, 
    sued them in Florida district court for breaching their franchise 
    obligations and for trademark infringement.
    \item The defendants moved to dismiss for lack of personal jurisdiction. 
    The district court denied the motion, holding that the Florida long arm statute 
    established jurisdiction over disputes arising from the franchise 
    agreement. The district court then found for Burger King.
    \item The Eleventh Circuit reversed, holding that the defendants did not 
    have reasonable notice and were ``financially unprepared'' for litigation 
    in a Florida forum.\footnote{Casebook p. 213.}
    \item Justice Brennan:
    \begin{enumerate}
        \item The ``fair warning'' requirement is satisfied if the defendant 
        ``\textbf{purposefully avails} itself of the privilege of conducting activities 
        in the forum state, thus invoking the benefit of the protection of its 
        laws.''\footnote{Casebook p. 215.}
        \item Once minimum contacts are established, the court may consider 
        other factors:\footnote{Casebook p. 217; cf. \emph{World-Wide 
        Volkswagen.}}
        \begin{enumerate}
            \item ``the burden on the defendant''
            \item ``the forum state's interest in adjudicating the dispute''
            \item ``the plaintiff's interest in obtaining the most efficient 
            resolution of controversies''
            \item the ``shared interest of the several States in furthering 
            fundamental substantive social policies''
        \end{enumerate}
        \item Rudcewicz established a substantial relationship with the 
        Florida headquarters and did not show how jurisdiction would be 
        fundamentally unfair. Reversed.
    \end{enumerate}
    \item Justice Stevens, dissenting: it is fundamentally unfair to require a 
    franchisee to submit to jurisdiction in the district of the franchisor. 
    There is a huge disparity in bargaining power, and franchises almost 
    always limit their activities to local markets.
    \item In \emph{Calder} (the \emph{National Enquirer} libel case), the 
    Court developed the ``effects test,'' under which activities ``purposefully 
    directed'' at a state can establish jurisdiction.
    \begin{enumerate}
        \item Bradt on \emph{Calder}: 
        California actress Shirley Jones (of ``Partridge Family'' fame) sued 
        the National Enquirer for libel based on a story it published 
        essentially calling her a drunk. Jones sued in California state 
        court. The Enquirer, which is based in Florida, contested 
        jurisdiction on the ground that it did not have minimum contacts with 
        California. The reporter had only gone to CA once, and everything 
        else was essentially done in Florida. The Supreme Court held that 
        there \emph{was} personal jurisdiction over the Enquirer in California 
        on the ground that it ``expressly aimed'' its conduct toward 
        California. This is an interesting spin on purposeful availment 
        commonly referred to as the ``effects test''---if a defendant commits 
        an intentional tort aimed at the forum state and causes harm in the 
        forum state, there is specific jurisdiction over the defendant in 
        cases arising out of that harm in the forum state. In \emph{Calder}, 
        the Court found that the Enquirer had aimed its conduct at California 
        because (a) it knew that's where Jones lived and worked and would 
        therefore suffer the brunt of the injury, and (b) because California 
        was the largest state for circulation of the Enquirer, so it knew the 
        harm in that state would be significant. 
    \end{enumerate}
    \item Purposeful direction is distinct from purposeful availment, which is 
    the quid pro quo the Court recognized in \emph{International Shoe}: the 
    privilege of enjoying ``the benefits and protections of the laws'' of a 
    state gives rise to obligations, including to the procedure of responding 
    to a suit in that state.\footnote{Casebook p. 238.}
\end{enumerate}

\subsubsection{Defenses to Personal Jurisdiction}

\begin{enumerate}
    \item Default and collaterally attack (\emph{Pennoyer}).
    \item Appear in court, move to dismiss, and appeal if you lose 
    (\emph{International Shoe}).
\end{enumerate}

\subsubsection{General and Specific Jurisdiction: \emph{Goodyear Dunlop Tires 
Operations, S.A. v. Brown}} % bluebook: operations?
 
A ``stream of commerce'' connection to the forum state is insufficient to 
establish personal jurisdiction.

\begin{enumerate}
    \item A bus accident outside Paris killed two 13-year-old boys from North 
    Carolina. The parents sued Goodyear USA and three of its foreign 
    subsidiaries in North Carolina state court.
    \item The subsidiaries moved to dismiss for lack of personal jurisdiction. The North Carolina 
    trial court denied the motion, the appellate court affirmed, and the state 
    Supreme Court denied review.
    \item Justice Ginsburg:
    \begin{enumerate}
        \item \textbf{General jurisdiction} exists when an actor has 
        ``continuous and systematic'' affiliations with the forum 
        state.\footnote{Supplement p. 4.}
        \item \textbf{Specific jurisdiction} exists when the specific cause of 
        action is connected to the forum state.
        \item A limited ``stream of commerce'' connection to the forum state 
        is insufficient to establish personal jurisdiction.
        \item Reversed.
    \end{enumerate}
\end{enumerate}

\subsubsection{Product Distribution: \emph{J. McIntyre Machinery, Ltd. v. Nicastro}}

Distributing products indirectly to the forum state is insufficient to 
establish personal jurisdiction without purposeful direction or purposeful 
availment.

\begin{enumerate}
    \item Nicastro injured his hand while using a machine that J. McIntyre, an 
    English corporation, manufactured. Nicastro sued J. McIntyre in New Jersey 
    state court. 
    \item The New Jersey Supreme Court held that \enquote{New Jersey's courts can 
    exercise jurisdiction over a foreign manufacturer of a product so long as 
    the manufacturer \enquote{knows or reasonable should know that its products are 
    distributed through a nationwide distribution 
    system.}}\footnote{Supplement p. 15.}
    \item Justice Kennedy: J. McIntyre did not purposefully direct its 
    activities at New Jersey, nor did it purposefully avail itself of the 
    privilege of the benefits and protections of New Jersey's laws. Reversed.
    \item Justice Breyer, concurring: this case can be decided on precedent 
    alone. There is no need to update jurisdictional rules to account for 
    developments in modern commerce.
    \item Justice Ginsburg, dissenting: J. McIntyre used a distributor to 
    shield itself from liability, but it clearly intended to market its 
    products to the United States as a whole.
\end{enumerate}

\subsubsection{Long-Arm Statutes}

\begin{enumerate}
    \item ``A court will not find in personam jurisdiction unless there is 
    statutory authorization for the exercise of that 
    jurisdiction.''\footnote{Casebook p. 241.}
    \item Long-arm statutes enable states to assert jurisdiction beyond their 
    borders.
    \item Some extend jurisdiction to the full extent that the Constitution 
    allows. Others specify the circumstances in which states can extend 
    jurisdiction.
    \item Detailed long-arm statutes can make the law more predictable.
    \item Federal courts usually follow the long-arm statutes of the states in 
    which they sit.
    \item If the state long-arm statute does not extend to the full 
    constitutional limits of in personam jurisdiction, the only question 
    before the court is statutory (e.g., in \emph{Bensusan}, below).
\end{enumerate}

\subsubsection{Online Activity and Long-Arm Statutes: \emph{Bensusan Rest. Corp 
v. King}}

Digital activity complicates jurisdictional questions, but long-arm statutes 
still place enforceable limits on personal jurisdiction.

\begin{enumerate}
    \item Bensusan and King both owned cabarets named ``Blue Note'' in New 
    York City and Columbia, MO, respectively. King built a website using the 
    name. Bensusan sued in the Southern District of New York for compensatory 
    damages, punitive damages, costs, attorney's fees, and to enjoin King from 
    using the name.
    \item The New York long-arm statute\footnote{N.Y.C.P.L.R. \S\ 320(a).} required that 
    the defendant must have been physically present in the state when he 
    committed the act in question.
    \item The district court rejected Bensusan's complaint under FRCP 12(b)(2) 
    for lack of personal jurisdiction. The Second Circuit affirmed, holding
    that King was not physically present and did not meet any of the other 
    statutory requirements for establishing jurisdiction.
    \item Federal court cannot be an authoritative source of state law---it 
    can only infer what New York courts would do in a comparable case.
    \item In \emph{Inset Systems, Inc. v. Instruction Set, Inc.}, a 
    Connecticut district court held that advertising online could potentially 
    reach ``as many as 10,000 Internet users within Connecticut'' and was 
    ``continuously available.'' Therefore, the advertiser had ``purposefully 
    availed itself of the privileges of doing business in 
    Connecticut.''\footnote{Casebook p. 250.}
    \item The \textbf{Zippo test}: in \emph{Zippo Manufacturing Co. v. Zippo 
    Dot Com, Inc.}, Zippo (the lighter company) sued Zippo (the publishing 
    company) in Pennsylvania district court for state and federal trademark 
    claims. The court established the Zippo test for determining jurisdiction 
    in Internet cases. The test relies on a scale of commercial activity. On 
    one side of the scale are sites that exist exclusively to do business 
    online.  If a defendant knowingly enters into a contract with residents of 
    a foreign jurisdiction, then personal jurisdiction is proper. On the other 
    side of the scale are sites that passively post information. Personal 
    jurisdiction cannot be asserted for those sites. In the middle are sites 
    in which the user can exchange information with a host computer. In those 
    cases, the ``level of interaction'' and the ``commercial nature of the 
    exchange'' are the grounds for determining the exercise of jurisdiction.  
    \end{enumerate}

\subsubsection{FRCP 4(k): Territorial Limits of Effective Service}

\begin{itemize}
    \item (1) Framework for exercising jurisdiction over defendants in 
    general.
    \begin{itemize}
        \item (A) The state's long-arm statute governs absent a contrary 
        federal rule or statute.
        \item (B) ``Bulge'' jurisdiction: third-party defendants are subject 
        to in personam jurisdiction if they can be served within 100 miles of 
        where the summons was issued.
        \item (C) Courts can exercise any in personam jurisdiction that a 
        federal statute authorizes.
    \end{itemize}
    \item (2) Framework for exercising jurisdiction in federal courts.
\end{itemize}

\subsubsection{Challenging the \emph{Pennoyer} Framework for Quasi In Rem Actions: \emph{Shaffer v.  
Heitner}}
% TODO: roman type for quasi in rem throughout
Jurisdiction in quasi in rem actions must be based on the minimum
contacts test (\emph{International Shoe}), not the territorial power
framework (\emph{Pennoyer}). Owning stock in a corporation incorporated in the 
forum state does not meet the minimum contacts requirement.

\begin{enumerate}
    \item Heitner owned one share of stock in Greyhound Corp., a
    Delaware corporation with its primary place of business in Phoenix.
    Heitner sued 28 of Greyhound's officers and directors in Delaware state court for 
    damages in a case involving antitrust and criminal contempt violations in 
    Oregon. Under a Delaware statute, he filed for an order of sequestration 
    against
    the property of the officers. The property consisted of Greyhound
    options and 82,000 shares of common stock.
    \item The defendants argued that the sequestration statute did not satisfy 
    due process and that the property seized could not be attached in
    Delaware. Denied.
    \item Justice Marshall:
    \begin{enumerate}
        \item \emph{Quasi in rem} jurisdiction has traditionally been based on 
        physical presence (\emph{Pennoyer}), not minimum contacts.
        \item The \emph{Pennoyer} framework included a few exceptions 
        (marriage, foreign corporations doing business in a state, etc.).
        \item Modern realities expanded the \emph{Pennoyer} framework (e.g., 
        \emph{Hess}) without fundamentally changing it.
        \item The \emph{International Shoe} framework supplanted 
        \emph{Pennoyer} for \emph{in personam} cases. No similar conceptual 
        revision has occurred for \emph{in rem} cases, though lower courts 
        have moved strongly in that direction.
        \item Key break from \emph{Pennoyer}: asserting jurisdiction over a 
        thing is a ``customarily elliptical way'' of asserting jurisdiction 
        over the interests of a person in a thing.
        \item \emph{Pennoyer} led to odd situations where property served as 
        the basis for jurisdiction in causes of action completely unrelated to 
        the property (like the present case). It was illogical, though 
        \emph{Pennoyer} permitted it, to assert jurisdiction indirectly, via 
        property, if direct assertion of personal jurisdiction would not be 
        allowed.
        \item There are no good historical reasons to cling to 
        \emph{Pennoyer}.
        \item We should therefore use the \emph{International Shoe} test for 
        all assertions of state court jurisdiction.
        \item Appellants' holdings did not constitute minimum contacts with 
        Delaware---thus, Delaware did not have jurisdiction.
        \item Reversed.
    \end{enumerate}
    \item Justice Powell, concurring:
    \begin{enumerate}
        \item Property that is ``indisputably and permanently'' within a state 
        (e.g., real estate) might pass the \emph{International Shoe} test. The 
        court should reserve judgment on that issue.
    \end{enumerate}
    \item Justice Stevens, concurring:
    \begin{enumerate}
        \item Fair notice requires warning that a particular activity will 
        open the actor to the jurisdiction of a foreign sovereign. Buying 
        stock includes no such warning.
        \item Agree with Powell.
        \item There are other longstanding methods of asserting jurisdiction 
        based on territory that should not be discounted.
    \end{enumerate}
    \item Justice Brennan, concurring in part and dissenting in part:
    \begin{enumerate}
        \item Delaware explicitly did not enact a law basing \emph{quasi in 
        rem} jurisdiction over shareholders on a minimum contacts test. For 
        the court to invalidate this imaginary statute is a pure ``advisory 
        opinion.''
    \end{enumerate}
\end{enumerate}

\subsubsection{Physical Presence and Personal Jurisdiction: \emph{Burnham v. 
Superior Court of Cal.}}

Presence within a state is sufficient to establish personal jurisdiction.

\begin{enumerate}
    \item Burnham and his wife decided to separate. Before his wife moved to 
    California, the couple agreed to divorce on grounds of ``irreconcilable 
    differences.'' After she left, however, Burnham filed for divorce on 
    grounds of ``desertion.''
    \item His wife brought suit in California. Some months later, Burnham was 
    on a business trip in California, where he was served with court summons 
    and a divorce petition.
    \item Burnham filed a motion to quash on the argument that his brief 
    contacts with California did not meet the requirements to establish 
    jurisdiction. The Superior Court denied the motion and the Court of Appeal 
    denied mandamus relief.
    \item Justice Scalia:
    \begin{enumerate}
        \item The question is whether physical presence is enough to establish 
        jurisdiction or whether the person must also have minimum contacts.
        \item There has never been a case that suggests in-state service is 
        insufficient to establish personal jurisdiction.
        \item The \emph{Pennoyer} territorial power framework had been 
        broadened over the 20th century. \emph{International Shoe} established 
        a different standard.
        \item Burnham sought to establish that presence in the forum state is no 
        longer sufficient to establish jurisdiction. This was entirely wrong. 
        The \emph{International Shoe} test was developed by analogy to the 
        ``physical presence'' test, and it would be ``perverse'' to use the 
        \emph{Shoe} test to undermine it.
        \item \emph{Shaffer} involved an absent defendant, and it held that 
        the defendant's contacts must include property related to the 
        litigation to establish jurisdiction (or, in a different light, that 
        quasi in rem and in personam are really one and the same). There was 
        no absent defendant in the present case.
        \item In response to Brennan's concurrence: (1) Brennan proposes a 
        standard based on ``contemporary notions of due process''---but this 
        is hopelessly subjective. (2) Brennan argues that the concept of 
        transient jurisdiction---of presence within a state creating a 
        ``reasonable expectation'' of being subject to suit---is based on 
        fairness. Really, though, it's based on the same traditions that 
        Brennan tries to dismiss. ``Justice Brennan's long journey is a 
        circular one.''
        \item Affirmed.
    \end{enumerate}
    \item Justice White, concurring: it would be unworkable to decide in each case whether service 
        was delivered fairly. The rule should stand as-is.
    \item Justice Brennan, concurring:
    \begin{enumerate}
        \item A rule does not comport with due process simply because of its 
        pedigree.
        \item \emph{Shaffer} established that all rules, regardless of 
        pedigree, must comport with modern understandings of due process.
        \item More than a century of the rule's existence gives defendants 
        ample notice that their presence within a state can subject them to 
        that state's jurisdiction.
        \item By visiting a state, a defendant avails himself of that state's 
        benefits. Without the transient jurisdiction rule, the actor would 
        have the full benefit of access to the state's courts as a plaintiff 
        while remaining immune from the same courts' jurisdiction. This 
        asymmetry would be unfair.
    \end{enumerate}
    \item Justice Stevens, concurring: the other justices' opinions are overly broad.
\end{enumerate}

\subsubsection{\emph{Mullane v. Central Hanover Bank \& Trust}}

If it's easy to serve notice by mail or in person, notice by publication does 
not satisfy due process requirements.  ``An elementary and fundamental 
requirement of due process in any proceeding which is to be accorded finality 
is \textbf{notice reasonably calculated}, under all the circumstances, to 
apprise interested parties of the pendency of the action and afford them an 
\textbf{opportunity to present their objections}.''

\begin{enumerate}
    \item A New York law required a judicial settlement of a common trust 
    fund. In strict compliance with the statute, the Central Hanover Bank and 
    Trust gave notice to beneficiaries by publication in a local newspaper.  
    \item Mullane was appointed special guardian for parties with an interest 
    in the fund. He argued that notice by publication violated the Fourteenth 
    Amendment's requirement for notice of judicial proceedings.
    \item The NY Court of Appeals rejected Mullane's argument.
    \item Justice Jackson:
    \begin{enumerate}
        \item The only notice to beneficiaries appeared in a local newspaper, 
        in strict compliance with the NY statute.
        \item At the time of the first investment, the trust had contacted 
        each person by mail.
        \item It doesn't matter whether this proceeding was in rem, 
        quasi in rem, or in personam. In all cases, courts have 
        the right to protect claimants' rights to notice and hearing and 
        determine claimants' interests.
        \item Notice by publication is chancy at best compared to notice in 
        person---``we are unable to regard this as more than a feint.''
        \item It's not necessary to put huge effort into finding unknown 
        claimants. Notice by publication is fine if it's not reasonably easy 
        to make contact.
        \item For parties with known contact information, notice by mail is 
        the minimum requirement. The NY statute requiring a minimum of notice 
        by publication in all cases is unconstitutional.
    \end{enumerate}
\end{enumerate}

\subsubsection{Service by Mail: \emph{Jones v. Flowers}}

When notice served by certified mail is returned undelivered, the government 
must take additional reasonable steps (e.g., service by regular mail or 
posting at the physical address) to satisfy due process.

\begin{enumerate}
    \item The plaintiff, Jones, was delinquent on his property taxes. The 
    state sent two certified letters to his address over the course of two 
    years, both of which were returned as ``unclaimed.'' Just before the 
    property was to be auctioned, the state also published a notice of public 
    sale in a newspaper. The house was sold to Flowers.
    \item Jones sued Flowers and the Commissioner, arguing that failure to 
    provide notice of the tax sale violated due process. The trial court 
    granted summary judgment in favor of the defendants. The Arkansas Supreme 
    Court affirmed.
    \item Justice Roberts:
    \begin{enumerate}
        \item Due process does not require that the property owner receive 
        actual notice, but it does require a reasonable attempt 
        (\emph{Mullane}).
        \item A person who actually wanted to inform someone about an 
        impending tax sale would surely take extra steps if a certified letter 
        of notice was returned unclaimed.
        \item The state could have resent the letter by regular mail or posted 
        it physically at the address. (It should not be required, though, to 
        hunt for Jones's contact information.)
        \item Reversed.
    \end{enumerate}
    \item Justice Thomas, dissenting: process requirements must be determined ex ante. They should not be dependent on the outcome of the first attempt.
\end{enumerate}

\subsubsection{Contractual Consent to Jurisdiction: \emph{Carnival Cruise 
Lines, Inc. v. Shrute}}

Forum selection clauses in contracts of adhesion are enforceable. Plaintiffs 
have a high burden of proof to show that a forum is so inconvenient that it 
violates due process.

\begin{enumerate}
    \item The plaintiffs, the Shutes, purchased tickets through a travel agent 
    for a seven-day cruise. The cruise line, Carnival, sent the defendants the 
    tickets by mail. The tickets included a contract that named Florida as the 
    forum state for any litigation regarding the contract. By purchasing the 
    tickets, the Shutes agreed to the terms of the contract.
    \item During the cruise, Mrs. Shute slipped on a mat and injured herself.  
    The Shutes sued for negligence in the Western District of Washington. 
    Carnival argued that (1) the forum selection clause in the contract 
    required the Shutes to bring suit in Florida, and (2) Carnival did not 
    have sufficient contacts with Washington to allow its courts to exercise 
    personal jurisdiction.
    \item The district court granted Carnival's motion to dismiss on the 
    grounds that Carnival had insufficient contacts with Washington to 
    exercise personal jurisdiction.
    \item The Ninth Circuit reversed on the grounds that Mrs. Shute would not 
    have been injured but for Carnival's solicitation of business in 
    Washington. It further held that the forum selection clause could not be 
    enforced because (1) it was not freely bargained for, (2) the Shutes were 
    physically and financially incapable of pursuing litigation in Florida, 
    and (3) the clause violated the Limitation of Vessel Owner's Liability 
    Act.
    \item Issues before the Supreme Court:
    \begin{enumerate}
        \item Was the forum selection clause enforceable?
        \item Was it too inconvenient for the Shutes to pursue litigation in 
        Florida?
        \item Did the forum selection clause violate the Limitation of Vessel 
        Owner's Liability Act?
    \end{enumerate}
    \item Justice Blackmun:
    \begin{enumerate}
        \item In their briefs, the Shutes conceded that they received adequate 
        notice of the forum selection clause.
        \item Some forum selection clauses might not be enforceable, e.g., if 
        they were established through ``fraud or overreaching.''
        \item In \emph{The Bremen}, the court upheld the validity of a forum 
        selection clause in international admiralty between two commercial 
        actors. The Ninth Circuit applied \emph{The Bremen} in this case to 
        hold that the forum selection clause was unenforceable because the 
        parties had not negotiated it. The Supreme Court (Blackmun here) 
        reasoned that the Shutes (individuals) did not negotiate with Carnival 
        (a large corporation).
        \item However, the lack of bargaining does not automatically 
        invalidate the contract. There are plenty of reasons why a 
        non-negotiated forum selection clause would be reasonable: (1) to 
        avoid litigation in every single passenger's different forum, (2) to 
        dispel confusion about the proper forum, and (3) to reduce fares 
        resulting from the limited fora. Thus, the clause is enforceable.
        \item Re Florida as an inconvenient forum: Shutes have not satisfied 
        the burden of proof to show heavy inconvenience.
        \item Re violation of the Limitation of Vessel Owner's Liability Act: 
        there is no evidence that Congress intended to avoid having a 
        plaintiff travel to a distant forum in order to litigate.
        \item The forum selection clause was enforceable. Reversed.
    \end{enumerate}
    \item Justice Stevens, dissenting:
    \begin{enumerate}
        \item Only the most meticulous passenger will be aware of the forum 
        selection clause.
        \item Passengers will not be able to evaluate the contract until they 
        agree to it by purchasing a non-refundable ticket. Negotiation is 
        logically impossible.
        \item The forum selection clause \emph{is} null and void under the 
        Limitation of Vessel Owner's Liability Act.
        \item This is a contract of adhesion. The Shutes did not know or 
        consent to all of its terms.
        \item Forum selection clauses are not enforceable if they were not 
        freely bargained for.
        \item The forum selection clause makes it more difficult for the 
        Shutes to recover damages for the slip-and-fall, which is contrary to 
        public policy.
    \end{enumerate}
    \item Before \emph{Carnival}, forum selection clauses in form contracts 
    were disfavored. Now they're found in virtually every consumer contract.
\end{enumerate}

\subsection{Subject Matter Jurisdiction and Venue}

\begin{enumerate}
    \item Federal courts have \textbf{limited jurisdiction}. Cases must be 
    within their limited subject matter jurisdiction. State courts have 
    \textbf{general jurisdiction}.
    \item The primary sources of limits on subject matter jurisdiction are 
    Article III of the Constitution, federal jurisdictional statutes, and 
    state long-arm statutes.
    \item Unlike personal jurisdiction, \textbf{subject matter jurisdiction 
    cannot be waived}, because waiver would upset the structural balance 
    between state and federal courts.
    \item Federal courts must find that they have subject matter jurisdiction 
    before they can decide any question on the merits.\footnote{Casebook p.  
    448.}
    \item Courts can (and must) raise subject matter objections sua sponte.
\end{enumerate}

\subsubsection{Federal Question Jurisdiction}

\begin{enumerate}
    \item Article III authorizes the federal judiciary: ``...The judicial 
    power shall extend to all cases, in law and equity, \textbf{arising under 
    this Constitution, the laws of the United States}, and treaties made, or 
    which shall be made, under their authority;---to all cases affecting 
    ambassadors, other public ministers and consuls;---to all cases of 
    admiralty and maritime jurisdiction;---to controversies to which the 
    United States shall be a party;---to controversies between two or more 
    states;---between a state and citizens of another state;---\textbf{between 
    citizens of different states};---between citizens of the same state 
    claiming lands under grants of different states, and between a state, or 
    the citizens thereof, and foreign states, citizens or subjects....''
    \item Federal judicial powers are \textbf{enumerated}, just like federal 
    legislative powers. But unlike congressional power, the scope of the 
    judiciary is narrowly construed.\footnote{Casebook p. 372 n.  2.}
    \item 28 U.S.C \S\ 1331 establishes federal question jurisdiction: ``The 
    district courts shall have original jurisdiction of all civil actions 
    arising under the Constitution, laws, or treaties of the United States.''
    \item States have concurrent jurisdiction over cases based on federal law 
    unless Congress has provided for exclusive jurisdiction.\footnote{Casebook 
    p. 373.}
    \item Federal courts have concurrent jurisdiction over state law cases as 
    long as there is subject matter jurisdiction. Diversity is typically the 
    basis for federal subject matter jurisdiction, though federal questions 
    are also often involved.\footnote{Casebook p. 373--374.}
\end{enumerate}

\paragraph{Well Pleaded Complaint: \emph{Louisville \& Nashville RR Co. v.  
Mottley\\\\}}

\textbf{Well pleaded complaint rule}: ``Plaintiff may not anticipate a federal defense by the defendant in her complaint and use that defense as a basis for federal jurisdiction.''\footnote{Casebook p. 378.}

\begin{enumerate}
    \item Facts:
    \begin{enumerate}
        \item September 1871: plaintiffs won a judgment against the railroad 
        company that awarded them free rail passes for life.
        \item June 29, 1906: Congress passed a statute forbidding free passes 
        or free transportation.
        \item January 1, 1907: the railroad company stopped honoring the 
        plaintiffs' free passes.
    \end{enumerate}
    \item The Mottleys brought a contract action against the railroad in 
    federal court, alleging that (1) the congressional statute does not cover 
    their kind of pass and (2) if the statute did cover their kind of pass, it 
    would violate the Fifth Amendment.
    \item The Supreme Court held that there was no subject matter question in 
    this case. Both parties were citizens of Kentucky, so there was no 
    diversity jurisdiction. The plaintiff's claim was based solely on a 
    private contract, and although the plaintiffs anticipated a federal 
    defense from the defendants, the court held that \textbf{an anticipated 
    federal defense is not sufficient to establish federal question 
    jurisdiction}.
\end{enumerate}

\paragraph{The Kaleidoscope and the Welcome Mat: \emph{Grable \& Sons Metal 
Prods. v. Darue Eng. \& Manuf.}}

A federal cause of action is not always required to establish federal question 
jurisdiction. Courts can choose to hear state law cases if they ``implicate 
significant federal issues.''

\begin{enumerate}
    \item 1994: The IRS seized Grable's land in Michigan to satisfy its tax 
    delinquency. It notified Grable by mail and sold the property to Darue 
    under a quitclaim deed.
    \item 1999: Grable brought a quiet title action against Darue in state 
    court, claiming that Darue's title was invalid because the IRS failed to 
    notify Grable with personal service as required by statute.
    \item Darue removed to federal court on the basis that Grable's claim 
    depended on an interpretation of federal tax law. The district court 
    declined to remand to state court. It granted summary judgment in favor of 
    Darue, holding that the IRS had substantially (if not literally) complied 
    with the statute.
    \item The Sixth Circuit affirmed the summary judgment. It held that 
    jurisdiction existed because (1) the title claim raised an issue of 
    federal law and (2) the claim implicated a substantial federal interest in 
    construing federal tax law.
    \item The question before the Supreme Court was whether a federal cause of 
    action is always required to establish federal question jurisdiction.
    \item Justice Souter:
    \begin{itemize}
        \item Federal jurisdiction exists for state law claims if they 
        ``implicate significant federal issues.''\footnote{Casebook p. 390.}
        \item District courts can refuse to exercise jurisdiction if the cause 
        of action is not based on a federal question and there is no diversity 
        of citizenship.
        \item Justice Cardozo: \enquote{a request to exercise federal-question 
        jurisdiction over a state action calls for a \enquote{common-sense 
        accommodation of judgment to [the] kaleidoscopic situations} that 
        present a federal issue, in \enquote{a selective process which picks 
        the substantial causes out of the web and lays the other ones 
        aside.}}\footnote{Casebook p. 391.}
        \item ``...the question is, does a state-law claim necessarily raise a 
        stated federal issue, actually disputed and substantial, which a 
        federal forum may entertain without disturbing any congressionally 
        approved balance of federal and state judicial 
        responsibilities.''\footnote{Casebook p. 392.}
        \item The Court saw the absence of a federal cause of action ``not as 
        a missing federal door key, always required, but as a missing welcome 
        mat, required in the circumstances...''\footnote{Casebook p. 394.}
        \item Affirmed.
    \end{itemize}
    \item Justice Thomas, concurring:
    \begin{itemize}
        \item Justice Holmes in \emph{Merrill Dow} argued in his dissent that 
        a federal cause of action should be necessary, not merely sufficient.  
        Thomas argued that the current rule is unclear, and under the right 
        circumstances he would be ``willing to consider'' adopting the clearer 
        Holmes rule.\footnote{Casebook p. 395--96.}
    \end{itemize}
\end{enumerate}

\subsubsection{Diversity Jurisdiction}

\begin{enumerate}
    \item Article III \S\ 2 defines federal judicial jurisdiction to include 
    ``all Cases...between Citizens of different States...''
    \item 28 U.S.C. \S\ 1332 authorizes diversity jurisdiction. Federal courts 
    have original jurisdiction over all cases where the amount in controversy 
    is greater than \$75,000 and the parties meet the diversity requirements.
    \item One reason for establishing diversity jurisdiction was to protect 
    out-of-state parties from prejudice in state courts.
    \item Unlike subject matter jurisdiction, which did not receive permanent 
    congressional authorization until 1875, Congress immediately granted 
    diversity jurisdiction.
    \item A plaintiff can aggregate multiple claims against a defendant to 
    meet the amount in controversy requirement---however, a plaintiff cannot 
    aggregate multiple claims against multiple defendants.
\end{enumerate}

\paragraph{Complete Diversity: \emph{Mas v. Perry}}

\textbf{Complete diversity} requires that none of the plaintiffs can be from 
the same state as any of the defendants.

\begin{enumerate}
    \item Mr. and Mrs. Mas were graduate assistants at Louisiana State 
    University. They discovered that their landlord, Perry, had been spying on 
    them through two-way mirrors for several months.
    \item The plaintiffs sued Perry in district court in Louisiana. Perry 
    moved to dismiss for lack of jurisdiction on the grounds that (1) the 
    plaintiffs failed to prove diversity jurisdiction (because Mr. and Mrs.  
    Mas lived in Louisiana) and (2) the jurisdictional amount of damages was 
    lacking. The district court rejected the motion.
    \item The Fifth Circuit held:
    \begin{enumerate}
        \item Mr. Mas was a French resident, so there was diversity 
        jurisdiction for his claim under 28 U.S.C. \S\ 1332(a)(2) (``alienage 
        jurisdiction'').
        \item Mrs. Mas was a domiciliary of Mississippi, so there was 
        diversity jurisdiction for her claim under 28 U.S. \S\ 1332(a)(1). 
        (Domicile is distinct from residency. Although Mrs. Mas resided in 
        Louisiana, she was a domiciliary of Mississippi, so diversity was 
        preserved. Domicile is established when a person intends to remain in 
        a state indefinitely.)
        \item The amount in controversy well exceeded the threshold for 
        federal jurisdiction (then \$10,000). The \textbf{\emph{St. Paul 
        Mercury} rule} (or the ``legal certainty'' rule) requires defendants 
        opposing jurisdiction to prove a ``legal certainty'' that the 
        plaintiff could not recovery more than the statutory threshold (now 
        \$75,000, then \$10,000).
        \item Affirmed.
    \end{enumerate}
\end{enumerate}

\paragraph{The Corporate Nerve Center: \emph{Hertz Corp. v. Friend}}

A corporation is a citizen where it is incorporated and where its ``officers 
direct, control, and coordinate the corporation's activities.''

\begin{enumerate}
    \item The plaintiffs brought an class action claim against Hertz in 
    California state court. Hertz requested removal to federal court. The 
    plaintiffs argued that diversity jurisdiction was lacking because Hertz 
    was a California citizen.
    \item 28 U.S.C. \S\ 1332(c)(1) provides that a corporation is a citizen of 
    the state where it is incorporated \emph{and} the state where it has its 
    principle place of business.
    \item Hertz argued that its principle place of business was New 
    Jersey.\footnote{It is incorporated in Delaware.} The district court 
    disagreed, relying on a Ninth Circuit precedent that defines a 
    corporation's principle place of business as the state where the amount of 
    business it conducts is ``significantly larger'' than other states. It 
    remanded to the state courts. The Ninth Circuit affirmed.
    \item The Second Circuit rule was that a corporation's principle place of 
    business is where its ``nerve center'' is located.\footnote{Supplement pp.  
    39--40.}. Courts have interpreted the test with increasing complexity. The 
    Supreme Court here sought to establish a simple rule. Expanding on the 
    ``nerve center'' approach, it held that the principle place of business is 
    ``the place where a corporation's officers direct, control, and coordinate 
    the corporation's activities.''\footnote{Supplement p. 41.}
    \item Under the Supreme Court's rule, Hertz's principle place of business 
    was not California. Reversed.
\end{enumerate}

\subsection{Supplemental Jurisdiction}

\begin{enumerate}
    \item Supplemental jurisdiction allows a claim without jurisdiction to 
    join a claim with valid jurisdiction.
    \item 28 U.S.C. \S\ 1367, passed in 1990, codified the federal common law 
    rules of pendent and ancillary jurisdiction into the single concept of 
    supplemental jurisdiction.
    \item Supplemental jurisdiction is \textbf{discretionary}.\footnote(28 
    U.S.C. \S\ 1367(c).
    \item Evaluating supplemental jurisdiction cases:
    \begin{enumerate}
        \item Is there a claim with valid original jurisdiction?
        \item Do supplemental claims form part of the same case or 
        controversy?  28 U.S.C. \S\ 1367(a).
        \item Are the supplemental claims within the \S\ 1367(b) rules? A 
        supplemental claim against a third-party defendant is not allowed if 
        (1) the original basis for jurisdiction is diversity and (2) the 
        supplemental claim would destroy diversity. \S\ 1367(b) exists to 
        prevent plaintiffs from bringing suit against diverse defendants and 
        waiting for them to implead a non-diverse defendant who is the real 
        target of the action.\footnote{See \emph{Owen v. Kroger}.}
        \item Can the court decide not to exercise supplemental jurisdiction?  
        \S\ 1367(c). % TODO
    \end{enumerate}
    \item \emph{Gibbs} is preserved in \S\ 1367(a) as the ``same case or 
    controversy.''
    \item \emph{Kroger} is preserved in \S\ 1367(b) as the limitation on 
    supplemental jurisdiction for claims against third-party defendants where 
    the claim would destroy diversity jurisdiction. % TODO: confirm
\end{enumerate}

\paragraph{Common Nucleus of Operative Fact: \emph{United Mine Workers of 
Am. v. Gibbs}} % TODO: bluebook abbrevs?

\begin{enumerate}
    \item Facts:
    \begin{enumerate}
        \item Spring 1960: The Tennessee Consolidated Coal company laid off 
        100 miners belonging to the United Mine Workers Local 5881.
        \item Summer 1960: Grundy Company, a subsidiary of Consolidated, hired 
        Gibbs to serve as superintendent for the opening of a new mine that 
        would employ members of the Southern Labor Union. Grundy also awarded 
        him a contract to haul the mine's coal to the railroad.
        \item August 15, 1960: UMW Local 5881 workers forcibly prevented the 
        opening of the new mine, believing that Consolidated had offered them 
        the jobs, not SLU. Gibbs lost his job as superintendent and was unable 
        to perform his hauling contract.
    \end{enumerate}
    \item Gibbs sued UMW in district court under the Labor Management 
    Relations Act, which established federal question jurisdiction. He also 
    made a state law claim for interference with his employment and hauling 
    contracts.
    \item The district court rejected the LMRA claim and the hauling contract 
    claim, but it awarded damages for Gibbs for the employment interference 
    claim.
    \item The key question before the Supreme Court was whether the district 
    court ``properly entertained'' jurisdiction of the state law 
    claim.\footnote{Casebook p. 410.}
    \item The Court held that federal courts can exercise supplemental [then, 
    pendent] jurisdiction over state law claims if they arise from a ``common 
    nucleus of operative fact'' with a federal claim.\footnote{Casebook p.  
    411.}
\end{enumerate}

\paragraph{Supplemental Jurisdiction and Third-Party Defendants: \emph{Owen 
Equip. \& Erection Co. v. Kroger}}

There is no supplemental jurisdiction for plaintiffs' claims against 
third-party defendants if the claim would destroy diversity jurisdiction. 28 
U.S.C. \S\ 1367(b).
% TODO: confirm

\begin{enumerate}
    \item James Kroger was electrocuted when the crane he was walking next to 
    came too close to a power line.
    \item Mrs. Kroger (Iowa) brought suit in district court in Nebraska 
    against the Omaha Public Power District (Nebraska) for negligence based on 
    diversity jurisdiction. OPPD impleaded Owen under FRCP 14(a) as a 
    third-party defendant. Kroger amended her complaint to assert a new claim 
    against Owen directly. OPPD left the case when the court granted its 
    motion for summary judgment, leaving only Kroger's direct claim against 
    Owen.
    \item On the third day of the trial, Owen revealed it was also from Iowa, 
    eliminating diversity jurisdiction. It moved to dismiss for lack of 
    jurisdiction. The district court denied the motion and the jury found for 
    Kroger. The Eight Circuit affirmed.
    \item Justice Stewart: Kroger's claim against Owen was independent from 
    its claim against OPPD, even though they arose from a common nucleus of 
    operative fact. Therefore, there was no supplemental jurisdiction, and 
    there was no diversity jurisdiction because both parties were from Iowa.  
    Reversed.
    \item Justice White, dissenting: Kroger did not deliberately circumvent 
    the diversity requirement. Efficiency considerations weigh in favor of 
    keeping the case in federal court.
\end{enumerate}

\subsection{Removal}

\begin{enumerate}
    \item 28 U.S.C. \S\ 1441 allows a defendant to remove a case from state to 
    federal court.
    \item Only defendants can remove.
    \item Only state $\rightarrow$ federal. No federal $\rightarrow$ state.
    \item Removal does not expand federal subject matter jurisdiction.
    \item The defendant is the \textbf{master of her complaint}. She is free 
    to bring her action in state court.
\end{enumerate}

\paragraph{Removal and Subject Matter Jurisdiction: \emph{Caterpillar Inc. v. Williams}}

Removal does not establish subject matter jurisdiction.

\begin{enumerate}
    \item The plaintiffs alleged that they entered into oral employment 
    contracts with Caterpillar, which they claim Caterpillar violated when it 
    closed its San Leandro plant and laid off the plaintiffs.
    \item The plaintiffs sued for breach of contract in state court. 
    Caterpillar removed to district court on the grounds that any individual 
    employment contracts were superseded by collective bargaining agreements, 
    for which LMRA preempted state law.
    \item The district court held that removal was proper. The Ninth Circuit 
    reversed.
    \item The Supreme Court affirmed the Ninth Circuit's holding that removal 
    was improper. It reasoned that the plaintiff's claim was based on a 
    private contract dispute for which their was no subject matter 
    jurisdiction in federal court. The well pleaded complaint rule prevented 
    Caterpillar from removing to federal court solely on the basis of a 
    federal defense.
    \begin{enumerate}
        \item There is significant dispute on this point. First, once the 
        federal defense is pleaded as a basis for removal, it is no longer a 
        hypothetical defense. Second, it might be unfair to prevent a 
        defendant who relies on federal law to have a federal forum to 
        determine its federal rights.
    \end{enumerate}
    \item Affirmed.
\end{enumerate}

\subsection{\emph{Forum Non Conveniens} and Transfer of Venue}

\begin{enumerate}
    \item Venue refers ``to the geographic specification of the proper court 
    or courts for the litigation of a civil action that is within the subject 
    matter jurisdiction of the district courts in general.'' 28 U.S.C. \S\ 
    1390(a).
    \item 28 U.S.C. \S\ 1391, ``Venue generally'': determines the federal 
    district where a defendant can be sued.
    \item 28 U.S.C. \S\ 1404, ``Change of venue'': determines when and how a 
    case can be transferred between federal districts---``for convenience...in 
    the interest of justice...''
    \item 28 U.S.C. \S\ 1406: ``Cure of waiver of defects'': how to deal with 
    improper venue.
    \item The primary purpose of venue is to protect the 
    defendant.\footnote{Casebook p. 449.}
    \item Most venue requirements are waivable.
    \item In 2011, Congress abolished the distinction between ``local'' and 
    ``transitory'' actions.\footnote{Supplement p. 50.}
\end{enumerate}

\subsubsection{\emph{Forum Non Conveniens} and Differences in Substantive Law: 
\emph{Piper Aircraft Co. v. Reyno}}

Differences in substantive law are insufficient to dismiss under \emph{forum 
non conveniens} unless the law in the target forum is egregiously bad.

\begin{enumerate}
    \item Six people were killed when a small plane crashed in Scotland. On 
    behalf of the decedents, the plaintiff sued the aircraft manufacturer 
    (Piper) and the propeller manufacturer in a California superior court.
    \item Piper successfully moved for transfer to the Middle District Court 
    of Pennsylvania under 28 U.S.C. §1404(a) (change of venue for 
    convenience). Then, in district court, it moved to dismiss on the grounds 
    of \emph{forum non conveniens}. The district court granted the motion, and 
    the defendants agreed to submit to the jurisdiction of Scottish courts.  
    The court noted several private interest and public policy reasons for 
    moving the venue to Scotland.\footnote{Casebook p. 474--75.}
    % TODO: such as?
    \item The Third Circuit reversed the motion to dismiss, arguing (1) that 
    the district court did not have the authority to review the policy reasons 
    for dismissal and (2) that dismissal is never appropriate when the law of 
    the alternative forum is less favorable to the plaintiff.
    \item The Supreme Court rejected the Third Circuit on both questions:
    \begin{enumerate}
        \item Differences in substantive law between two forums should never 
        be a substantial factor unless the alternative forum is egregiously 
        bad.
        \item It held that the district court did not abuse its discretion in 
        weighing the public and private interests, on the basis that foreign 
        plaintiffs deserve less deference in determining the convenience of a 
        forum.
    \end{enumerate}
\end{enumerate}
