\section{Jurisdiction}

TODO: difference between territorial jurisdiction and personal jurisdiction?

\subsection{\emph{Pennoyer}: Personal Jurisdiction and Territorial Power}

\begin{enumerate}
    \item  In an in personam case, does service by publication to a non-resident defendant establish territorial jurisdiction? \emph{Pennoyer v. Neff}:
Mitchell sues Neff in Oregon state trial court for nonpayment of legal fees rendered in 1862-1863. Neff is nowhere to be found, so Mitchell publishes notice of the suit in a newspaper. Neff does not appear, so the court orders a default judgment. The property is attached and then sold to Mitchell, who sells it to Pennoyer. Eight years later, Neff successfully sues Pennoyer to recover the property. The court (Justice Field) relies on an analytical framework in which the basis for jurisdiction is a state's territorial power. States are all-powerful within their borders, and powerless beyond. It holds that service by publication isn't good enough for in personam suits against a non-resident (though it might be good enough for in rem suits). Thus, the original judgment against Neff was void.
    \item Doctrine of \textbf{collateral attack}: a judgment void when rendered is void forever.
    \item The Due Process Clause (Fourteenth Amendment) governs personal jurisdiction.
\end{enumerate}

\subsection{\emph{Hess}: Jurisdiction Over Out-of-State Drivers}

\begin{enumerate}
    \item Can a state implement a statute that requires out-of-state drivers to give implied consent to jurisdiction within that state? \textbf{\emph{Hess v. Pawloski}} Plaintiff, a Pennsylvania resident, ``negligently and wantonly drove a motor vehicle on a public highway in Massachusetts,'' causing injury to the defendant. In a MA Superior Court, plaintiff contested MA's jurisdiction, which was denied. The Supreme Judicial Court upheld the order. Plaintiff appealed to the Supreme Court on Fourteenth Amendment grounds. The court reasoned that earlier cases (e.g., \emph{Kane v. New Jersey}) have upheld the constitutionality of statutes that require out-of-state drivers to appoint an agent to receive process before using the highway. States can legitimately require to appoint similar agent implicitly, and these kinds of statutes do not not constitute discrimination against non-residents. Therefore, it is consistent with the Due Process Clause for states to require out-of-state drivers to implicitly appoint an agent to receive process, thereby establishing jurisdiction over those drivers if civil actions arise.
\end{enumerate}

\subsection{\emph{International Shoe}: ``Minimum Contacts''}

\begin{enumerate}
    \item Can a state have jurisdiction over an entity if the entity does not have a permanent presence within the state? What constitutes the necessary contact to establish jurisdiction? \textbf{\emph{International Shoe Co. v. Washington}}: The State of Washington sued International Shoe to recover unpaid contributions to the state unemployment compensation fund. Justice Stone wrote the opinion.

    International Shoe argued first that the Washington statute imposes an unconstitutional burden on interstate commerce. The court rejected this on the basis that ``it is no longer debatable'' that the commerce clause gives Congress broad power to regulate interstate commerce.

    Second, it argued that merely soliciting orders within a state ``does not render the seller amenable to suit within the state.'' The court notes that historically, physical presence within a state is a prerequisite for jurisdiction in in personam cases (\emph{Pennoyer}). But now, \textbf{minimum contacts} are sufficient. ``Presence'' is a symbolic term that can refer to business activities within a territory; in other words, it can refer to activities that give rise to the liabilities at issue in the suit. Moreover, since an entity enjoys certain benefits and protections from a state's laws, it also has an obligation to that state.

    The court held that Washington is entitled to recover the unpaid contributions.

\emph{...due process requires only that in order to subject a defendant to a judgment in personam, if he be not present within theterritory of the forum, he have certain \textbf{minimum contacts} with it such that the maintenance of the suit does not offend \textbf{``traditional notions of fair play and substantial justice.''}}\footnote{Casebook p. 179.}
    \item Justice Black, concurring: States have a constitutional power to tax and sue corporations that due business in the state's territory. The test of ``fair play and substantial justice'' is ``confusing'' and gives the court the unwarranted power to strike down any legislation it might see as violating ``natural justice.''
    \item Unlike in \emph{Pennoyer}, \emph{in personam} jurisdiction is based on ``minimum contacts,'' not territorial principles.
\end{enumerate}

\subsubsection{General and Specific Jurisdiction}

\begin{enumerate}
    \item \textbf{General jurisdiction}: The defendant has substantial enough contacts with a state that any dispute can be litigated in that state, regardless of whether the dispute arises from those contacts.
    \item \textbf{Specific jurisdiction}: Contacts to the forum are related to the specific dispute.
\end{enumerate}

\subsection{\emph{World-Wide}: Jurisdiction Over Out-of-State Car Dealers}

\begin{enumerate}
    \item Does selling a car in one state constitute the necessary minimum contact (under the \emph{Shoe} test) to establish in personam jurisdiction over the seller in another state?\textbf{\emph{World-Wide Volkswagen Corp. v. Woodson}}: The Robinsons purchased an Audi in New York. It caught fire in an accident in Oklahoma.    

    Justice White: To establish jurisdiction, defendants must have minimum contacts, and the contacts must not violate ``traditional notions of fair play and substantial justice'' (forum State's interest in adjudicating the dispute; convenient venue for plaintiff; interstate judicial system's interest in efficient resolution; and shared interest in fundamental policy goals). Jurisdictional rules have been relaxed, but the Constitution nonetheless privileges state sovereignty. ``Petitioners (World-Wide and Seaway) carry on no activity whatsoever in Oklahoma.'' Petitioners could not reasonably predict being haled into court in OK. Corporations can ``purposely avail'' themselves of the benefits of conducting activity in the forum state---but there is no such availment here. No contacts, so no jurisdiction.

    Justice Brennan, dissenting: The Court reads \emph{Shoe} too narrowly. The seller and dealer purposefully injected their product ``into the stream of interstate commerce,'' thus establishing minimum contact.

    Justices Marshall and Blackmun, dissenting: The dealer and seller chose to become part of a global marketing and servicing network. Cars derive their value from being mobile. The dealer and seller received economic advantage from the ability to draw revenue from Oklahoma.

    Justice Blackmun, dissenting: Confusing why the distributor and seller are getting sued here. Also, cars are mobile by nature.
    \item Questions from Bradt:
    \begin{enumerate}
        \item Who are the defendants in the case, and which defendants are challenging personal jurisdiction? What is the basis of jurisdiction for the defendants who are not so challenging?
        \item In what states do you think the Robinsons could sue defendants over whom the Oklahoma courts do not have jurisdiction?
        \item Note on page 192 Justice White's statement that even if the forum the plaintiff has chosen will not be abusive to the defendant, the Due Process Clause may bar the plaintiff's choice ``acting as an instrument of interstate federalism.''  Is this consistent with International Shoe?  And what do you think of this statement after reading Burger King?  Should federalism play any role here?
        \item Pay close attention to Justice White's discussion of ``purposeful availment'' on page 194, and Justice Marshall's dissent on that point on pages 197--198. Who has the better of the argument?  Do you think Worldwide and Seaway should have been subject to jurisdiction in Oklahoma in this case?
    \end{enumerate}

\end{enumerate}

\subsection{\emph{Burger King Corp. v. Rudzewicz}}

\begin{enumerate}
    \item TODO: Brief
    \item Questions from Bradt:
    \begin{enumerate}
        \item Is this a case of specific jurisdiction or general jurisdiction?
        \item How did Rudzewicz ``purposefully direct'' his activities toward Florida? Are you persuaded that his actions justify Florida's exercise of jurisdiction over him?
        \item Who has the better of the argument as to whether it's fundamentally fair to subject Rudzewicz to jurisdiction in Florida, Justice Brennan or Justice Stevens?
        \item Why did Seaway and Worldwide VW benefit from the protections of the Due Process Clause while Mr. Rudzewicz didn't? The Court seems to make a lot of Mr. Rudzewicz's business sophistication and the benefits he received from participation in the national market---why weren't those factors persuasive in Volkswagen? Is the work that purposeful availment's doing here justified? (Note that two justices (Burger and Rehnquist) thought that there was no jurisdiction in VW but that there was in BK, and Justice O'Connor (who was on the Court for BK, but not VW), thought there was jurisdiction in BK despite that VW was a controlling opinion.)
    \end{enumerate}
\end{enumerate}

\subsection{\emph{Goodyear Dunlop Tires Operations, S.A. v. Brown}}

\begin{enumerate}
    \item TODO
    \item Bradt questions:
    \begin{enumerate}
        \item Note the Court's definitions of ``general'' and ``specific'' jurisdiction (4). The Court, per Justice Ginsburg, defines general jurisdiction as where the defendant is ''essentially at home.'' According to Justice Ginsburg, what are the ``paradigm places'' where the defendant is ``at home''?
        \item What provision of the North Carolina long-arm statute did the plaintiffs rely on in order to obtain personal jurisdiction over the foreign subsidiaries?
        \item Goodyear USA (the parent corporation) consented to the North Carolina court's jurisdiction over it. Could Goodyear USA, which is based and incorporated in Ohio, have successfully challenged personal jurisdiction in North Carolina? Was there ``general jurisdiction'' over Goodyear in North Carolina? Specific jurisdiction?
        \item The Court notes that some of the foreign subsidiaries' tires made their way to North Carolina (to be used on horse trailers, etc.)? Let's say one of those tires allegedly caused an accident in North Carolina? Do you think there would be specific jurisdiction over the foreign subsidiaries in that case? (Think about that in light of \emph{J. McIntyre.})
    \end{enumerate}
\end{enumerate}

\subsection{\emph{J. McIntyre Machinery, Ltd. v. Nicastro}}

\begin{enumerate}
    \item TODO
    \item Bradt questions:
    \begin{enumerate}
        \item In J.McIntyre, is there a rule that a majority of the Court agrees on regarding the exercise of specific jurisdiction when a defendant places goods in the stream of commerce? What open questions remain?
        \item Do you agree with Justice Kennedy's minimum-contacts analysis? Is he right that J. McIntyre did not purposefully avail itself of New Jersey's market? Or does Justice Ginsburg have the better of this argument?
        \item Where do Justice Kennedy and Justice Breyer agree? Disagree?
        \item Note Justice Kennedy's reasons for rejecting Justice Brennan's framework from Asahi (p. 19--20). Do you find these persuasive?
    \end{enumerate}
\end{enumerate}

\subsection{Purposeful Availment and Purposeful Direction}

\begin{enumerate}
    \item TODO (pp. 237-241)
    \item Bradt on \textbf{\emph{Calder v. Jones}}: California actress Shirley Jones (of ``Partridge Family'' fame) sued the National Enquirer for libel based on a story it published essentially calling her a drunk.  Jones sued in California state court.  The Enquirer, which is based in Florida, contested jurisdiction on the ground that it did not have minimum contacts with California.  The reporter had only gone to CA once, and everything else was essentially done in Florida.  The Supreme Court held that there \emph{was} personal jurisdiction over the Enquirer in California on the ground that it ``expressly aimed'' its conduct toward California.  This is an interesting spin on purposeful availment commonly referred to as the ``effects test''---if a defendant commits an intentional tort aimed at the forum state and causes harm in the forum state, there is specific jurisdiction over the defendant in cases arising out of that harm in the forum state.  In \emph{Calder}, the Court found that the Enquirer had aimed its conduct at California because (a) it knew that's where Jones lived and worked and would therefore suffer the brunt of the injury, and (b) because California was the largest state for circulation of the Enquirer, so it knew the harm in that state would be significant.  
\end{enumerate}

\subsection{Long-arm Statutes}

\begin{enumerate}
    % TODO: italicize all instances of in personam and in rem
    \item ``A court will not find in personam jurisdiction unless there is statutory authorization for the exercise of that jurisdiction.''\footnote{Casebook p. 241.}
    \item Long-arm statutes enable states to assert jurisdiction beyond their borders.
    \item Some extend jurisdiction to the full extent that the Constitution allows. Others specify the circumstances in which states can extend jurisidiction.
    \item Detailed long-arm statutes can make the law more predictable.
    \item Federal courts usually follow the long-arm statutes of the states in which they sit.
    \item If the state long-arm statute does not extend to the full constitutional limits of in personam jurisdiction, the only question before the court is statutory (e.g., in \emph{Bensusan}, below).

\end{enumerate}

\subsection{\emph{Benusan}: Online Activity}

\begin{enumerate}
    \item Does New York have jurisdiction over an out-of-state defendant because of his online activities? \textbf{\emph{Bensusan Restaurant Corp. v. King}}: Bensusan and King both own cabarets in New York City and Columbus, MO, respectively. King build a website, and Bensusan sues for compensatory damages, punitive damages, costs, attorney's fees, and to enjoin King from using the name ``Blue Note.'' The New York long-arm statute (N.Y.C.P.L.R. § 320(a)) requires that the defendant must have been physically present in the state when he committed the act. The court finds that King was not physically present, and does not meet any of the other statutory requirements for jurisdiction.
    \item Federal court cannot be an authoritative source of state law---it can only infer what New York courts would do in a comparable case.
    \item In \emph{Inset Systems, Inc. v. Instruction Set, Inc.}, a Connecticut district court held that advertising online could potentially reach ``as many as 10,000 Internet users within Connecticut'' and is ``continuously available.'' Therefore, the advertiser has ``purposefully availed itself of the privileges of doing business in Connecticut.''
    \item In \emph{Zippo Manufacturing Co. v. Zippo Dot Com, Inc.}, Zippo (the lighter company) sued Zippo (the publishing company). The Pennsylvania district court established the Zippo test for determining jurisdiction in Internet cases. The test relies on a scale of commercial activity. On one side of the scale are sites that exists exclusively to do business online. If a defendant knowingly enters into a contract with residents of a foreign jurisdiction, then personal jurisdiction is proper. On the other side of the scale are sites that passively post information. Personal jurisdiction cannot be asserted for those sites. In the middle are sites in which the user can exchange information with a host computer. In those cases, the ``level of interaction'' and the ``commercial nature of the exchange'' are the basis for determining the exercise of jurisdiction. 
\end{enumerate}

TODO: pp. 259-261 notes 2-5; Rulebook pp. 24-25: FRCP 4(k)

\subsection{\emph{Shaffer v. Heitner}}

\begin{enumerate}
    \item TODO
\end{enumerate}

\subsection{\emph{Burnham v. Superior Court of Calif.}}

\begin{enumerate}
    \item TODO
\end{enumerate}

