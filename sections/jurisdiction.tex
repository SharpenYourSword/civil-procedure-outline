\section{Jurisdiction}

\subsection{\emph{Pennoyer}: Personal Jurisdiction}

\begin{enumerate}
    \item  In an in personam case, does service by publication to a non-resident defendant establish territorial jurisdiction? \emph{Pennoyer v. Neff}:
Mitchell sues Neff in Oregon state trial court for nonpayment of legal fees rendered in 1862-1863. Neff is nowhere to be found, so Mitchell publishes notice of the suit in a newspaper. Neff does not appear, so the court orders a default judgment. The property is attached and then sold to Mitchell, who sells it to Pennoyer. Eight years later, Neff successfully sues Pennoyer to recover the property. The court (Justice Field) relies on an analytical framework in which the basis for jurisdiction is a state's territorial power. States are all-powerful within their borders, and powerless beyond. It holds that service by publication isn't good enough for in personam suits against a non-resident (though it might be good enough for in rem suits). Thus, the original judgment against Neff was void.
    \item Bradt: a judgment void when rendered is void forever.
\end{enumerate}

