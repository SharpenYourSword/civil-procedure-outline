\section{Jurisdiction}

\begin{enumerate}
    \item Territorial jurisdiction: jurisdiction over cases arising in or involving persons residing within a defined territory.
    \item Personal jurisdiction: a court's power to bring a person into its adjudicative process.
\end{enumerate}

\subsection{\emph{Pennoyer}: Personal Jurisdiction and Territorial Power}

\begin{enumerate}
    \item  In an in personam case, does service by publication to a non-resident defendant establish territorial jurisdiction? \emph{Pennoyer v. Neff}:
Mitchell sues Neff in Oregon state trial court for nonpayment of legal fees rendered in 1862-1863. Neff is nowhere to be found, so Mitchell publishes notice of the suit in a newspaper. Neff does not appear, so the court orders a default judgment. The property is attached and then sold to Mitchell, who sells it to Pennoyer. Eight years later, Neff successfully sues Pennoyer to recover the property. The court (Justice Field) relies on an analytical framework in which the basis for jurisdiction is a state's territorial power. States are all-powerful within their borders, and powerless beyond. It holds that service by publication isn't good enough for in personam suits against a non-resident (though it might be good enough for in rem suits). Thus, the original judgment against Neff was void.
    \item Doctrine of \textbf{collateral attack: a judgment void when rendered is void forever.}
    \item Collateral estoppel: binding effect of a judgment on later controversies involving a different claim from that on which the original judgment was based.
    \item The Due Process Clause (Fourteenth Amendment) governs personal jurisdiction.
    \item In \emph{Pennoyer}, personal jurisdiction is based on territorial power. States are all-powerful within their borders and powerless without.
    \item Who benefits from updates to \emph{Pennoyer}?
    \item \emph{Milliken v. Meyer}: domicile alone within a state is sufficient to establish jurisdiction.
\end{enumerate}

\subsection{\emph{Hess}: Jurisdiction Over Out-of-State Drivers}

\begin{enumerate}
    \item Can a state implement a statute that requires out-of-state drivers to give implied consent to jurisdiction within that state? \textbf{\emph{Hess v. Pawloski}} Plaintiff, a Pennsylvania resident, ``negligently and wantonly drove a motor vehicle on a public highway in Massachusetts,'' causing injury to the defendant. In a MA Superior Court, plaintiff contested MA's jurisdiction, which was denied. The Supreme Judicial Court upheld the order. Plaintiff appealed to the Supreme Court on Fourteenth Amendment grounds. The court reasoned that earlier cases (e.g., \emph{Kane v. New Jersey}) have upheld the constitutionality of statutes that require out-of-state drivers to appoint an agent to receive process before using the highway. States can legitimately require to appoint similar agent implicitly, and these kinds of statutes do not not constitute discrimination against non-residents. Therefore, it is consistent with the Due Process Clause for states to require out-of-state drivers to implicitly appoint an agent to receive process, thereby establishing jurisdiction over those drivers if civil actions arise.
\end{enumerate}

\subsection{\emph{International Shoe}: ``Minimum Contacts''}

\textbf{"But now that the capias ad respondendum has given way to personal service of summons or other form of notice, due process requires only that in order to subject a defendant to a judgment in personam, if he be not present within the territory of the forum, he have certain minimum contacts with it such that the maintenance of the suit does not offend `traditional notions of fair play and substantial justice.'"}

\begin{enumerate}
    \item Can a state have jurisdiction over an entity if the entity does not have a permanent presence within the state? What constitutes the necessary contact to establish jurisdiction? \textbf{\emph{International Shoe Co. v. Washington}}: The State of Washington sued International Shoe to recover unpaid contributions to the state unemployment compensation fund. Justice Stone wrote the opinion.

    International Shoe argued first that the Washington statute imposes an unconstitutional burden on interstate commerce. The court rejected this on the basis that ``it is no longer debatable'' that the commerce clause gives Congress broad power to regulate interstate commerce.

    Second, it argued that merely soliciting orders within a state ``does not render the seller amenable to suit within the state.'' The court notes that historically, physical presence within a state is a prerequisite for jurisdiction in in personam cases (\emph{Pennoyer}). But now, \textbf{minimum contacts} are sufficient. ``Presence'' is a symbolic term that can refer to business activities within a territory; in other words, it can refer to activities that give rise to the liabilities at issue in the suit. Moreover, since an entity enjoys certain benefits and protections from a state's laws, it also has an obligation to that state.

    The court held that Washington is entitled to recover the unpaid contributions.

\emph{...due process requires only that in order to subject a defendant to a judgment in personam, if he be not present within theterritory of the forum, he have certain \textbf{minimum contacts} with it such that the maintenance of the suit does not offend \textbf{``traditional notions of fair play and substantial justice.''}}\footnote{Casebook p. 179.}
    \item Justice Black, concurring: States have a constitutional power to tax and sue corporations that due business in the state's territory. The test of ``fair play and substantial justice'' is ``confusing'' and gives the court the unwarranted power to strike down any legislation it might see as violating ``natural justice.''
    \item Unlike in \emph{Pennoyer}, \emph{in personam} jurisdiction is based on ``minimum contacts,'' not territorial principles.
\end{enumerate}

\subsubsection{General and Specific Jurisdiction}

\begin{enumerate}
    \item \textbf{General jurisdiction}: The defendant has substantial enough contacts with a state that any dispute can be litigated in that state, regardless of whether the dispute arises from those contacts.
    \item \textbf{Specific jurisdiction}: Contacts to the forum are related to the specific dispute.
\end{enumerate}

\subsection{\emph{McGee} TODO}

TODO: insurance policies and specific jurisdiction

\subsection{\emph{World-Wide}: Jurisdiction Over Out-of-State Car Dealers}

\begin{enumerate}
    \item Does selling a car in one state constitute the necessary minimum contact (under the \emph{Shoe} test) to establish \emph{in personam} jurisdiction over the seller in another state?\textbf{\emph{World-Wide Volkswagen Corp. v. Woodson}}: The Robinsons purchased an Audi in New York. It caught fire in an accident in Oklahoma.    
    \item Justice White: To establish jurisdiction, defendants must have minimum contacts, and the contacts must not violate ``traditional notions of fair play and substantial justice'' (forum State's interest in adjudicating the dispute; convenient venue for plaintiff; interstate judicial system's interest in efficient resolution; and shared interest in fundamental policy goals). Jurisdictional rules have been relaxed, but the Constitution nonetheless privileges state sovereignty. ``Petitioners (World-Wide and Seaway) carry on no activity whatsoever in Oklahoma.'' Petitioners could not reasonably predict being haled into court in OK. Corporations can ``purposely avail'' themselves of the benefits of conducting activity in the forum state---but there is no such availment here. No contacts, so no jurisdiction.
    \item Justice Brennan, dissenting: The Court reads \emph{Shoe} too narrowly. The seller and dealer purposefully injected their product ``into the stream of interstate commerce,'' thus establishing minimum contact.
    \item Justices Marshall and Blackmun, dissenting: The dealer and seller chose to become part of a global marketing and servicing network. Cars derive their value from being mobile. The dealer and seller received economic advantage from the ability to draw revenue from Oklahoma.
    \item Justice Blackmun, dissenting: Confusing why the distributor and seller are getting sued here. Also, cars are mobile by nature.
    \item Questions from Bradt:
    \begin{enumerate}
        \item Who are the defendants in the case, and which defendants are challenging personal jurisdiction? What is the basis of jurisdiction for the defendants who are not so challenging?
        \item In what states do you think the Robinsons could sue defendants over whom the Oklahoma courts do not have jurisdiction?
        \item Note on page 192 Justice White's statement that even if the forum the plaintiff has chosen will not be abusive to the defendant, the Due Process Clause may bar the plaintiff's choice ``acting as an instrument of interstate federalism.''  Is this consistent with International Shoe?  And what do you think of this statement after reading Burger King?  Should federalism play any role here?
        \item Pay close attention to Justice White's discussion of ``purposeful availment'' on page 194, and Justice Marshall's dissent on that point on pages 197--198. Who has the better of the argument?  Do you think Worldwide and Seaway should have been subject to jurisdiction in Oklahoma in this case?
    \end{enumerate}

\end{enumerate}

\subsection{\emph{Burger King Corp. v. Rudzewicz}}

\begin{enumerate}
    \item TODO: Brief
    \item Rudzewicz fails to show that jurisdiction would be fundamentally unfair.
    \item Justice Stevens, dissenting: it's fundamentally unfair to require a franchisee to submit to the jurisdiction of the franchiser.
    \item Questions from Bradt:
    \begin{enumerate}
        \item Is this a case of specific jurisdiction or general jurisdiction?
        \item How did Rudzewicz ``purposefully direct'' his activities toward Florida? Are you persuaded that his actions justify Florida's exercise of jurisdiction over him?
        \item Who has the better of the argument as to whether it's fundamentally fair to subject Rudzewicz to jurisdiction in Florida, Justice Brennan or Justice Stevens?
        \item Why did Seaway and Worldwide VW benefit from the protections of the Due Process Clause while Mr. Rudzewicz didn't? The Court seems to make a lot of Mr. Rudzewicz's business sophistication and the benefits he received from participation in the national market---why weren't those factors persuasive in Volkswagen? Is the work that purposeful availment's doing here justified? (Note that two justices (Burger and Rehnquist) thought that there was no jurisdiction in VW but that there was in BK, and Justice O'Connor (who was on the Court for BK, but not VW), thought there was jurisdiction in BK despite that VW was a controlling opinion.)
    \end{enumerate}
\end{enumerate}

\subsection{Defenses to Personal Jurisdiction}

\begin{enumerate}
    \item Default and collaterally attack.
    \item Appear in court, move to dismiss, and appeal if you lose.
\end{enumerate}

\subsection{\emph{Goodyear Dunlop Tires Operations, S.A. v. Brown}}

\begin{enumerate}
    \item TODO
    \item Bradt questions:
    \begin{enumerate}
        \item Note the Court's definitions of ``general'' and ``specific'' jurisdiction (4). The Court, per Justice Ginsburg, defines general jurisdiction as where the defendant is ''essentially at home.'' According to Justice Ginsburg, what are the ``paradigm places'' where the defendant is ``at home''?
        \item What provision of the North Carolina long-arm statute did the plaintiffs rely on in order to obtain personal jurisdiction over the foreign subsidiaries?
        \item Goodyear USA (the parent corporation) consented to the North Carolina court's jurisdiction over it. Could Goodyear USA, which is based and incorporated in Ohio, have successfully challenged personal jurisdiction in North Carolina? Was there ``general jurisdiction'' over Goodyear in North Carolina? Specific jurisdiction?
        \item The Court notes that some of the foreign subsidiaries' tires made their way to North Carolina (to be used on horse trailers, etc.)? Let's say one of those tires allegedly caused an accident in North Carolina? Do you think there would be specific jurisdiction over the foreign subsidiaries in that case? (Think about that in light of \emph{J. McIntyre.})
    \end{enumerate}
\end{enumerate}

\subsection{\emph{Asahi}}

TODO

\subsection{\emph{J. McIntyre Machinery, Ltd. v. Nicastro}}

\begin{enumerate}
    \item TODO
    \item Bradt questions:
    \begin{enumerate}
        \item In J.McIntyre, is there a rule that a majority of the Court agrees on regarding the exercise of specific jurisdiction when a defendant places goods in the stream of commerce? What open questions remain?
        \item Do you agree with Justice Kennedy's minimum-contacts analysis? Is he right that J. McIntyre did not purposefully avail itself of New Jersey's market? Or does Justice Ginsburg have the better of this argument?
        \item Where do Justice Kennedy and Justice Breyer agree? Disagree?
        \item Note Justice Kennedy's reasons for rejecting Justice Brennan's framework from Asahi (p. 19--20). Do you find these persuasive?
    \end{enumerate}
\end{enumerate}

\subsection{Purposeful Availment and Purposeful Direction}

\begin{enumerate}
    \item TODO (pp. 237-241)
    \item TODO see Hanson v. Denckla
    \item Bradt on \textbf{\emph{Calder v. Jones}}: California actress Shirley Jones (of ``Partridge Family'' fame) sued the National Enquirer for libel based on a story it published essentially calling her a drunk.  Jones sued in California state court.  The Enquirer, which is based in Florida, contested jurisdiction on the ground that it did not have minimum contacts with California.  The reporter had only gone to CA once, and everything else was essentially done in Florida.  The Supreme Court held that there \emph{was} personal jurisdiction over the Enquirer in California on the ground that it ``expressly aimed'' its conduct toward California.  This is an interesting spin on purposeful availment commonly referred to as the ``effects test''---if a defendant commits an intentional tort aimed at the forum state and causes harm in the forum state, there is specific jurisdiction over the defendant in cases arising out of that harm in the forum state.  In \emph{Calder}, the Court found that the Enquirer had aimed its conduct at California because (a) it knew that's where Jones lived and worked and would therefore suffer the brunt of the injury, and (b) because California was the largest state for circulation of the Enquirer, so it knew the harm in that state would be significant.  
\end{enumerate}

\subsection{Long-arm Statutes}

\begin{enumerate}
    % TODO: italicize all instances of in personam and in rem
    \item ``A court will not find in personam jurisdiction unless there is statutory authorization for the exercise of that jurisdiction.''\footnote{Casebook p. 241.}
    \item Long-arm statutes enable states to assert jurisdiction beyond their borders.
    \item Some extend jurisdiction to the full extent that the Constitution allows. Others specify the circumstances in which states can extend jurisidiction.
    \item Detailed long-arm statutes can make the law more predictable.
    \item Federal courts usually follow the long-arm statutes of the states in which they sit.
    \item If the state long-arm statute does not extend to the full constitutional limits of in personam jurisdiction, the only question before the court is statutory (e.g., in \emph{Bensusan}, below).

\end{enumerate}

\subsection{\emph{Bensusan}: Online Activity}

\begin{enumerate}
    \item Does New York have jurisdiction over an out-of-state defendant because of his online activities? \textbf{\emph{Bensusan Restaurant Corp. v. King}}: Bensusan and King both own cabarets in New York City and Columbia, MO, respectively. King build a website, and Bensusan sues for compensatory damages, punitive damages, costs, attorney's fees, and to enjoin King from using the name ``Blue Note.'' The New York long-arm statute (N.Y.C.P.L.R. § 320(a)) requires that the defendant must have been physically present in the state when he committed the act. The court finds that King was not physically present, and does not meet any of the other statutory requirements for jurisdiction.
    \item Federal court cannot be an authoritative source of state law---it can only infer what New York courts would do in a comparable case.
    \item In \emph{Inset Systems, Inc. v. Instruction Set, Inc.}, a Connecticut district court held that advertising online could potentially reach ``as many as 10,000 Internet users within Connecticut'' and is ``continuously available.'' Therefore, the advertiser has ``purposefully availed itself of the privileges of doing business in Connecticut.''
    \item In \emph{Zippo Manufacturing Co. v. Zippo Dot Com, Inc.}, Zippo (the lighter company) sued Zippo (the publishing company). The Pennsylvania district court established the Zippo test for determining jurisdiction in Internet cases. The test relies on a scale of commercial activity. On one side of the scale are sites that exists exclusively to do business online. If a defendant knowingly enters into a contract with residents of a foreign jurisdiction, then personal jurisdiction is proper. On the other side of the scale are sites that passively post information. Personal jurisdiction cannot be asserted for those sites. In the middle are sites in which the user can exchange information with a host computer. In those cases, the ``level of interaction'' and the ``commercial nature of the exchange'' are the basis for determining the exercise of jurisdiction. 
\end{enumerate}

TODO: pp. 259-261 notes 2-5; Rulebook pp. 24-25: FRCP 4(k)

\subsection{\emph{Shaffer v. Heitner}}

Should \emph{quasi in rem} proceedings be based on the \emph{Pennoyer}
framework of territorial power (which would allow jurisdiction) or the
\emph{International Shoe} framework of minimum contacts consistent with
traditional notions of fair play and substantial justice (which might
not)?

Heitner, the respondent, owned one share of stock in Greyhound Corp., a
Delaware corporation with its primary place of business in Phoenix.
Heitner originally sued 28 of its officers and directors for damages in
a case involving antitrust and criminal contempt violations in Oregon.
Under a Delaware statute, he filed for an order of sequestration against
the property of the officers. The property consisted of Greyhound
options and 82,000 shares of common stock. The defendants (here,
appellants) argue that the sequestration statute did not accord due
process of law and that the property seized could not be attached in
Delaware. Deleware courts rejected this jurisdictional challenge.

Justice Marshall:
\begin{enumerate}
\item
  \emph{Quasi in rem} jurisdiction has traditionally been based on
  physical presence, not minimum contacts (\emph{Pennoyer}).
\item
  The \emph{Pennoyer} framework included a few exceptions (marriage,
  foreign corporations doing business in a state, etc.).
\item
  Modern realities expanded the \emph{Pennoyer} framework (e.g.,
  \emph{Hess}) without fundamentally changing it.
\item
  The \emph{International Shoe} framework supplanted \emph{Pennoyer} for
  \emph{in personam} cases. No similar conceptual revision has occurred
  for \emph{in rem} cases, though lower courts have moved strongly in
  that direction.
\item
  Key break from \emph{Pennoyer}: asserting jurisdiction over a thing is
  a ``customarily elliptical way'' of asserting jurisdiction over the
  interests of a person in a thing.
\item
  \emph{Pennoyer} led to odd situations where property serves as the
  basis for jurisdiction in causes of action that are completely
  unrelated to the property (like the present case). It is illogical,
  though \emph{Pennoyer} permits it, to assert jurisdiction indirectly,
  via property, if direct assertion of personal jurisdiction would not
  be allowed.
\item
  There are no good historical reasons to cling to \emph{Pennoyer}.
\item
  We should therefore use the \emph{International Shoe} test for all
  assertions of state court jurisdiction.
\item
  Appellants' holdings do not constitute minimum contacts with
  Delaware--thus, Delaware does not have jurisdiction.
\end{enumerate}

Justice Powell, concurring:

\begin{enumerate}
\item
  Property that is ``indisputably and permanently'' within a state
  (e.g., real estate) might pass the \emph{International Shoe} test. The
  court should reserve judgment on that issue.
\end{enumerate}

Justice Stevens, concurring in the judgment:

\begin{enumerate}
\item
  Fair notice requires warning that a particular activity will open the
  actor to the jurisdiction of a foreign sovereign. Buying stock
  includes no such warning.
\item
  Agree with Powell.
\item
  There are other longstanding methods of asserting jurisdiction based
  on territory that should not be discounted.
\end{enumerate}

Justice Brennan, concurring in part and dissenting in part:

\begin{enumerate}
\item
  Delaware explicitly did not enact a law basing \emph{quasi in rem}
  jurisdiction over shareholders on a minimum contacts test. For the
  court to invalidate this imaginary statute is a pure ``advisory
  opinion.''
\end{enumerate}

The court unanimously (sans Rehnquist) held that Delaware did not have
jurisdiction over the defendants. The Delaware Supreme Court ruling was
reversed.

Rule: Jurisdiction in \emph{quasi in rem} actions must be based on the minimum
contacts test (\emph{International Shoe}), not the territorial power
framework (\emph{Pennoyer}).

\subsection{\emph{Burnham v. Superior Court of Calif.}}

Is a person's mere presence within a state sufficient to establish that
state's jurisdiction over that person?

The petitioner, Burnham, and his wife decided to separate. Before his
wife moved to California, the couple agreed to divorce on grounds of
``irreconcilable differences.'' After she left, however, the petitioner
filed for divorce on grounds of ``desertion.'' His wife brought suit in
California. Some months later, the petitioner was on a business trip in
California, where he was served with court summons and a divorce
petition.

Petitioner filed a motion to quash on the argument that his brief
contacts with California did not meet the requirements to establish
jurisdiction. The Superior Court denied the motion and the Court of
Appeal denied mandamus relief.

Justice Scalia:

\begin{enumerate}
\item
  The question is whether physical presence is enough to establish
  jurisdiction, or whether the person must also have minimum contacts.
\item
  There has never been a case that suggests in-state service is
  insufficient to establish personal jurisdiction.
\item
  \emph{Pennoyer} broadened over the 20th century; \emph{International
  Shoe} established a different standard.
\item
  The petitioner seeks to establish that presence in the forum state is
  no longer sufficient to establish jurisdiction. This is entirely
  wrong. The \emph{International Shoe} test was developed by analogy to
  the ``physical presence'' test, and it would be ``perverse'' to use
  the \emph{Shoe} test to undermine it.
\item
  \emph{Shaffer} involved an absent defendant, and it held that the
  defendant's contacts must include property related to the litigation
  to establish jurisdiction (or, in a different light, that \emph{quasi
  in rem} and \emph{in personam} are really one and the same). There is
  no absent defendant in the present case.
\item
  In response to Brennan's concurrence: (1) Brennan proposes a standard
  based on ``contemporary notions of due process''---but this is
  hopelessly subjective. (2) Brennan argues that the concept of
  transient jurisdiction---of presence within a state creating a
  ``reasonable expectation'' of being subject to suit--is based on
  fairness. Really, though, it's based on the same traditions that
  Brennan tries to dismiss. ``Justice Brennan's long journey is a
  circular one.''
\end{enumerate}

Justice White, concurring in part:

\begin{enumerate}
\item
  It would be unworkable to decide in each case whether service was
  delivered fairly. The rule should stand as-is.
\end{enumerate}

Justice Brennan, concurring in the judgment:

\begin{enumerate}
\item
  A rule does not comport with due process simply because of its
  pedigree.
\item
  \emph{Shaffer} established that all rules, regardless of pedigree,
  must comport with modern understandings of due process.
\item
  More than a century of the rule's existence gives defendants ample
  notice that their presence within a state can subject them to that
  state's jurisdiction.
\item
  By visiting a state, a defendant avails himself of that state's
  benefits. Without the transient jurisdiction rule, the actor would
  have the full benefit of access to the state's courts as a plaintiff
  while remaining immune from the same courts' jurisdiction. This
  asymmetry would be unfair.
\end{enumerate}

Justice Stevens, concurring in the judgment:

\begin{enumerate}
\item
  The other justices' opinions are overly broad {[}though he doesn't say
  how{]}.
\end{enumerate}

The Superior Court's ruling is affirmed.

Rule: presence within a state is sufficient to establish jurisdiction.

% TODO: make timeline of jurisdiction cases

\subsection{Notice}

\subsubsection{\emph{Mullance v. Central Hanover Bank \& Trust}}

\textbf{``An elementary and fundamental requirement of due process in any proceeding which is to be accorded finality is notice reasonably calculated, under all the circumstances, to apprise interested parties of the pendency of the action and afford them an opportunity to present their objections.''}

\begin{enumerate}
    \item A New York law required a judicial settlement of a common trust fund. In strict compliance with the statute, the Central Hanover Bank and Trust gave notice to beneficiaries by publication in a local newspaper. Does the NY statute requiring, at a minimum, notice by publication for trust beneficiaries violate the Fourteenth Amendment?
    \item The NY Court of Appeals overruled objections that notice by publication here violates the Fourteenth Amendment.
    \item Justice Jackson:
    \begin{enumerate}
        \item The only notice to beneficiaries appeared in a local newspaper, in strict compliance with the NY statute.
        \item At the time of the first investment, the trust had contacted each person by mail.
        \item It doesn't matter whether this proceeding is \emph{in rem}, \emph{quasi in rem}, or \emph{in personam}. In all cases, courts have the right to protect claimants' rights to notice and hearing and determine claimants' interests.
        \item There's not much precedent here.
        \item Notice by publication is chancy at best compared to notice in person---``we are unable to regard this as more than a feint.''
        \item It's not necessary to put huge effort into finding unknown claimants. Notice by publication is fine if it's not reasonably easy to make contact.
        \item For parties with known contact information, notice by mail is the minimum requirement. The NY statute requiring a minimum of notice by publication in all cases is unconstitutional.
    \end{enumerate}
    \item The notice requirements in the New York statute are unconstitutional. Judgment for the appellant (Mullane).
    \item Rules:
    \begin{enumerate}
       \item  The constitutionality of the method of service is independent from the classification of the cause of action as \emph{in rem}, \emph{quasi in rem}, or \emph{in personam}.
        \item If it's easy to serve notice by mail or in person, notice by publication does not satisfy due process requirements.
    \end{enumerate}
\end{enumerate}

\subsubsection{\emph{Jones v. Flowers}}

\begin{enumerate}
    \item When notice of a tax sale is returned undelivered, is the government
    required to take additional reasonable steps to satisfy the Due Process
    Clause?
    \item The plaintiff, Jones, was delinquent on his property taxes. The state
    sent two certified letters to his address over the course of two years,
    both of which were returned as ``unclaimed.'' Just before the property
    was to be auctioned, the state also published a notice of public sale in
    a newspaper. The house was sold to Flowers.
    \item The trial court granted summary judgment in favor of the defendants (the
    Commissioner and Flowers). The Arkansas Supreme Court affirmed.
    \item Justice Roberts:
    \begin{enumerate}
        \item Due process does not require that the property owner receive actual
          notice, but it does require a reasonable attempt (\emph{Mullane}).
        \item A person who actually wanted to inform someone about an impending tax
          sale would surely take extra steps if a certified letter of notice was
          returned unclaimed.
        \item The state could have resent the letter by regular mail or posted it
          physically at the address. (It should not be required, though, to hunt
          for Jones's contact information.)
    \end{enumerate}
    \item Justice Thomas, dissenting:
    \begin{enumerate}
        \item Process requirements must be determined ex ante. They should not be
          dependent on the outcome of the first attempt.
          state should take additional steps to give notice.
    \end{enumerate}
    \item Reversed and remanded.
    \item If notice by mail of an impending tax sale is returned unclaimed, the
\end{enumerate}

\subsection{Consent by Contract}

\subsubsection{\emph{Carnival Cruise Lines, Inc. v. Shrute}}

\begin{enumerate}
    \item The plaintiffs, the Shutes, purchased tickets for a seven-day cruise through a travel agent. The cruise line, Carnival, sent the defendants the tickets by mail. The tickets included a contract that named Florida as the forum state for any litigation regarding the contract. By purchasing the tickets, the Shutes agreed to the terms of the contract.
    \item Mrs. Shute injured herself during the cruise by slipping on a mat. The Shutes sued in Washington for negligence. Carnival argued that (1) the forum selection clause in the contract requires the Shutes to bring suit in Florida, and (2) Carnival did not have sufficient contacts with Washington to allow its courts to exercise personal jurisdiction.
    \item The U.S. District Court for the Western District of Washington granted Carnival's motion to dismiss on the grounds that Carnival had insufficient contacts with Washington to exercise personal jurisdiction. The Court of Appeals reversed on the grounds that Mrs. Shute would not have been injured but for Carnival's solicitation of business in Washington. The Court of Appeals further held that the forum selection clause could not be enforced because (1) it was not freely bargained for, (2) the Shutes are physically and financially incapable of pursuing litigation in Florida, and (3) the clause violates the Limitation of Vessel Owner's Liability Act.
    \item Issues:
    \begin{enumerate}
        \item Is the forum selection clause enforceable?
        \item Is it too inconvenient for the Shutes to pursue litigation in Florida?
        \item Does the forum selection clause violate the Limitation of Vessel Owner's Liability Act?
    \end{enumerate}
    \item Justice Blackmun:
    \begin{enumerate}
        \item In their briefs, the respondents (the Shutes) conceded that they received adequate notice of the forum selection clause.
        \item In \emph{The Bremen}, the court upheld the validity of a forum selection clause in international admiralty between two commercial actors. The Court of Appeals applied \emph{The Bremen} in this case to hold that the forum selection clause was unenforceable because the parties had not negotiated it. The Supreme Court (Blackmun here) argued that the Shutes (individuals) did not negotiate with Carnival (large corporation).
        \item The lack of bargaining does not automatically invalidate the contract, however. There are plenty of reasons why a non-negotiated forum selection clause would be reasonable: (1) avoid litigation in every single passenger's different forum, (2) dispelling confusion about the proper forum, and (3) reduced fares resulting from the limited fora. Thus, the clause is enforceable.
        \item Re Florida as an inconvenient forum: Shutes have not satisfied the burden of proof to show heavy inconvenience.
        \item Re violation of the Limitation of Vessel Owner's Liability Act: no evidence that Congress intended to avoid having a plaintiff travel to a distant forum in order to litigate.
    \end{enumerate}
    \item Justice Stevens, dissenting:
    \begin{enumerate}
        \item Only the most meticulous passenger will be aware of the forum selection clause.
        \item Passengers will not be able to evaluate the contract until they agree to it by purchasing a non-refundable ticket. Negotiation is not possible.
        \item Clause \emph{is} null and void under the Limitation of Vessel Owner's Liability Act.
        \item This is a contract of adhesion. The Shutes did not know or consent to all of its terms.
        \item Forum selection clauses are not enforceable if they were not freely bargained for.
        \item The forum selection clause makes it more difficult for the Shutes to recover damages for the slip-and-fall, which is contrary to public policy.
    \end{enumerate}
    \item Holding: The Court of Appeals is overturned (i.e., the forum selection clause was enforceable; the forum was not inconvenient; and the clause did not violate the Limitation of Vessel Owner's Liability Act).
    \item Rules:
    \begin{enumerate}
        \item Forum selection clauses can be enforced even if they were not freely bargained for.
        \item Plaintiffs have a high burden of proof to show that a forum is so inconvenient that it violated due process.
        \item The Limitation of Vessel Owner's Liability Act does not protect plaintiffs from traveling to distant forums to litigate their disputes.
    \end{enumerate}
\end{enumerate}

\subsection{Subject-Matter Jurisdiction and Venue}

\subsubsection{Federal-Question Jurisdiction}

\begin{enumerate}
    \item todo
\end{enumerate}

\paragraph{\emph{Louisville \& Nashville RR Co. v. Mottley}} % TODO
\paragraph{\emph{Grable \& Sons Metal Prods. v. Darue Eng. \& Manuf.}} % TODO

\subsubsection{Diversity Jurisdiction}

\begin{enumerate}
    \item todo
\end{enumerate}

\paragraph{\emph{Mas v. Perry}} % TODO
\paragraph{\emph{Hertz Corp. v. Friend}} % TODO

\subsubsection{Supplemental Jurisdiction}

\begin{enumerate}
    \item todo
\end{enumerate}

\paragraph{\emph{United Mine Workers of America v. Gibbs}} % TODO
\paragraph{\emph{Owen Equip. \& Erection Co. v. Kroger}} % TODO

\subsubsection{Removal}

\begin{enumerate}
    \item todo
\end{enumerate}

\paragraph{\emph{Caterpillar Inc. v. Williams}} % TODO

\subsubsection{\emph{Forum Non Conveniens} and Transfer of Venue}

\begin{enumerate}
    \item todo
\end{enumerate}

\paragraph{\emph{Piper Aircraft Co. v. Reyno}} % TODO
